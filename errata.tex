\documentclass{article}
\usepackage{amssymb,makeidx}

\begin{document}

\noindent{\bf \large List of Known Errata}
\vspace{.5 cm}

A current list of known errata is maintained on the WWW at \newline{\tt
https://Randall-Holmes.github.io/errata.txt}  The author's e-mail is
\newline {\tt rholmes@boisestate.edu}.  Please tell me about anything you find!

The worst error is:

\begin{description}
\item[p. 128] The object $G$ used in the definition of sums and products of
indexed families of cardinals is not described correctly.  Currently,
the text introduces $G$, incorrectly, as an element of the Cartesian
product of the indexed family $F$ of cardinals.  It is necessary to
stipulate further that the "index set" (the domain) of the indexed
family $F$ of cardinals is a set of singletons; $G$ is then correctly
specified as an element of SI$^{-1}[\prod[F]]$; i.e., SI$\{G\}$, not $G$
itself, belongs to the Cartesian product of $F$.  

It would be even better to start with $G$: ``Let $G$ be an indexed
family of sets.  Let $F$ be the associated indexed family of
cardinals, defined by $F(\{i\})$ = $|G(i)|\ldots$" We could then
define $\prod[F]$ and $\sum[F]$ in the same forms given in the text.
In the proof of K\"onig's theorem on p. 132, the $\cal A$ and $\cal B$
functions are examples of the correct construction of $G$.

\item[p. 132] It should be ${\cal P}_1^2\{B\}$ in the proof of K\"onig's
Theorem, not ${\cal P}^2\{B\}$.

\end{description}

Other errors:

\begin{description}
\item[p. 71, repeated p. 74:] There is an extra parenthesis in the definition
of Cartesian products of indexed families of sets, which might be
initially confusing.

\item[p. 116:] An obvious printer glitch; it should be possible to decipher.

\item[p. 125:] In the last proof, the occurrence of $|A-Y|+|A|$ should be
$|A-Y|+|Y|$.

\item[p. 173:] The statement and proof of a theorem is missing here.  I
assume without proving or even noting the assumption that for any rank
$X$ at or before $Z_0$, $T[X]$ is also a rank.  This is true, and not hard
to prove, but it does need a proof (supplied on my web page).

\item[p. 183:] Both of the occurrences of $T^2\{\Omega\}$ in the proof of the
(correct) Theorem that No is an iterated cut system need to be
replaced with something else; in the first case we need to say that
the ranks are those indexed by elements of $T^2[$Ord$]$ (the image of the
set of ordinals under the $T^2$ operation), and the second instance of
$T^2\{\Omega\}$ should be replaced by the limit of $T^2[$Ord$]$, which is $\Omega$
itself, not $T^2\{\Omega\}$.  The fact that $\lim T^2[$Ord$]$ = $\Omega$ is
discussed in the next chapter.

\item[p. 190:] In the definition of beth numbers, I neglected to stipulate
that each of the collections intersected to form the set of
beth-numbers must contain $\aleph_0$.

\end{description}

\end{document}
