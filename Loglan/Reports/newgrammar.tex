\documentclass[12pt]{book}

\title{A New Systematic Grammar of TLI Loglan}

\author{Randall Holmes}

\begin{document}

\maketitle

\chapter{Introduction}

This is a new grammar of TLI Loglan.   The intention is to present a coherent picture of my provisional adjustments of the language.  The organizational principle is that I follow the structure of the Parsing  Expression Grammar (PEG) which is used to generate the computer grammar.

There will be observations on points of difference with earlier versions of the language as necessary.

This document speaks authoritatively, but not all these proposals have been approved by the TLI Loglan community.  I am writing this in hope of providing support for a consensus that this is the way to proceed.  The membership is welcome to offer criticisms, whether general or of particular points.

The reader can safely ignore comments in footnotes unless already proficient in Loglan or interested in the history of the language.

\chapter{Phonetics and Orthography}

In this chapter, we discuss the rules for writing and pronouncing Loglan, and the way in which a stream of speech sounds or letters is formally resolved into Loglan words (with the proviso that phonetics do not dictate word boundaries between structure words).

\section{Phonetic and orthographic components}


\subsection{Loglan letters and punctuation}

The letters of the Loglan alphabet are the 23 letters of the Roman alphabet excluding {\bf q, w, x}. 

 The foreign letters {\bf q, w, x} can only occur in ``alien text" embedded in Loglan.\footnote{These letters were originally not included in Loglan, then they were added with strange pronunciations in the 1980's and 90's, then they were largely eliminated from the dictionary in the late 90's;  after 2013, we proceeded to eliminate them completely again.   Names for these letters (usable as pronouns) will be presented later.}

The vowels are {\bf a, e, i, o, u, y}.  The first five are the regular vowels.

The consonants are   {\bf b, c, d, f, g, h, j, k, l, m, n, p, r, s, t, v, z}

The Loglan name of a vowel V has two forms (legacy and modern\footnote{The modern forms were suggested by us after 2013, but we have fully accommodated the phonetics to the original forms.}):  legacy uppercase is V{\bf ma} and lowercase is V{\bf fi}, and modern uppercase is {\bf zi}V{\bf ma} and lowercase {\bf zi}V.  The legacy forms are fully supported, but they are phonetically irregular as Loglan words, and there are contexts where the modern forms must be used.

The Loglan name of a consonant C is C{\bf ai} (uppercase) or C{\bf ei} (lowercase).

There are some other series of letter names to be introduced below.  The primary function of these words is not phonetic, but as variables (pronouns), as will be explained later.

The junctures, indicating syllable breaks or stress are {\bf -, ', *}.  The hyphen {\bf -} is a simple syllable break.\footnote{The use of the hyphen to abbreviate the phonetic hyphen {\bf y} found in earlier sources is not accepted here;  in general, we do not pronounce punctuation.}  {\bf '} is a syllable stress marker, which may appear in place of (not in addition to) a hyphen after a stressed syllable, or in final position in a word after a stressed syllable.  {\bf *} is a symbol for emphatic syllable stress, with the same grammar as {\bf '}.  A juncture is never followed by another juncture;  a hyphen can be followed only by a letter (the hyphen, unlike the stress marks, never appears in final position).

We define a phonetic block as a sequence of  letters and junctures (referred to collectively as ``characters"), with the junctures giving information about syllable breaks and stress.  A phonetic block is always intended to be pronounced without pause.

The terminal punctuation marks of Loglan are {\bf .:?!;\#}.  

The comma {\bf ,} is an especially important punctuation mark, with the phonetic meaning of a pause in the flow of speech.  A comma is always followed by whitespace followed by a phonetic block or alien text (ignoring any initial parentheses or double quotes appearing before the block or alien text).
A phonetic pause is always denoted either by whitespace or a comma followed by whitespace;  in some contexts the comma-marked pause is mandatory.  Whitespace may or may not denote a pause;  there are some contexts where whitespace must denote a pause.\footnote{Uses of a close comma without a following space in earlier versions of Loglan are entirely replaced with uses of the hyphen as a syllable break.}

The double hyphen {\bf --} is an independent punctuation mark, not a syllable break:  it represents a pause, probably longer than the pause represented by a comma, and in some cases may be used in place of a comma where a pause is required.  The ellipsis $\ldots$ is an independent punctuation mark, not terminal punctuation, similarly representing a pause and occasionally usable in place of a comma.

Parentheses and double quotes may enclose Loglan or alien text  under some circumstances described in the grammar.  These are generally ignored for phonetic purposes.  They are not pronounced:  punctuation marks are never intended to be pronounced in the version of Loglan described here, though they may dictate pauses.

The letters have lowercase forms and uppercase forms and there is a capitalization rule applying to phonetic blocks.  The formal capitalization rule is quite complex:  the basic idea is that an uppercase letter will not appear immediately following a lowercase letter unless it is the first letter of a phonetic copy of a letter name (class {\tt TAI0} to be discussed below; the letter names given above are words of this class), and a vowel may appear capitalized after {\bf z} (for malicious reasons to be explained later).    There is no restriction on capitalization resuming after a juncture.  This allows the usual sort of capitalization, and also allows all-caps, and where junctures are present individual blocks of letters may be capitalized in different styles independently.
The special treatment of letter names will be motivated in examples when these words appear.\footnote{The approach to punctuation which we have taken has been driven partially by our own design decisions and partly by the punctuation and capitalization practices in the Visit to Loglandia.}

\subsection{Pronunciation of Loglan letters}

The regular vowels have typical continental European (not English!) pronunciation.  The irregular vowel {\bf y} may be pronounced with the indistinct schwa sound.  Unstressed regular vowels do {\bf not} become schwa (English and Russian speakers note!)  A more distinct pronunciation for {\bf y} (the vowel in English ``look"\footnote{A suggestion of John Cowan.} or the vowel written bI in Cyrillic) might be preferred.

The pronunciations of {\bf b, d, f, g,  k, l, m,  p, r, s, t, v, z} require no special comment (except to note that {\bf g} is always hard).

The letters {\bf c, j} have unusual pronunciations, {\bf c} being English ``sh" in ``ship", and {\bf j} being the voiced version of the same sound, the ``z" in ``azure".  English
``ch" and ``j" are the diphthongs {\bf tc} and {\bf dj} respectively.

The letter {\bf h} has the typical English pronunciation, except in syllable final position, where it has the sound of ``ch" in Scottish ``loch".  The latter pronunciation is actually permitted in other positions as well.\footnote{Syllable-final {\bf h} did not exist in previous versions of Loglan.  The pronunciation of {\bf h} which is mandatory in final position might be preferred by speakers of some languages in other positions.}

The letter {\bf n} is pronounced as usual in English, and (as usual in English as well!) is pronounced as the ``ng" in ``sing" when it occurs before {\bf g}, {\bf k}, or the alternative pronunciation of {\bf h}.

The vowels {\bf i} and {\bf u} are sometimes pronounced as English ``y" and ``w". as will be explained below.

The consonants {\bf l, m, n, r} are sometimes syllabic (``vocalic").  When they are used syllabically, they are always doubled.\footnote{The rule that syllabic continuants must be doubled forces changes of spelling in names in legacy Loglan text in many cases.  Syllabic consonants in borrowed predicates were already doubled, and Brown suggested in Loglan 1 that this might be a good rule to adopt in general.}

\subsection{Alien text}

We describe the rules for embedding alien text in Loglan.\footnote{The model for these rules is actually the final state of the rule for Linnaean names with {\bf lao} (now foreign names in general) given in the late 90's.  We require that the occurrences of {\bf y} there suggested merely for speech also be expressed in writing.  The use of double quotes is a novelty but seems natural.  The strong quotation scheme of 1989 Loglan is abandoned, essentially by giving {\bf lie} the same phonetic grammar as {\bf lao}.}  For such text, rules of pronunciation are not supplied by Loglan.  It is required that alien text (however it is pronounced)
be preceded by a pause (regarded as part of the alien text) and followed by a pause or end of text or speech (not regarded as part of the alien text):  a pause may be expressed either by whitespace or by a comma or terminal punctuation (terminal punctuation only after the alien text, of course)  followed by whitespace, and end of text or speech simply by end of text or by terminal punctuation.  The body of the alien text between the initial pause and the final pause or end may be text not containing double quotes enclosed in double quotes, or it may consist of one or more blocks of text excluding commas, spaces and terminal punctuation marks, separated by the word {\bf y}, which must be preceded and followed by pauses in speech, which are independently expressible by whitespace or by a comma-marked pause.  Examples are {\bf``War and Peace"} and {\bf War y and y Peace}.   When alien text is enclosed in quotes, occurrences of {\bf y} between pause-separated components of the alien text may be omitted in writing but must appear in speech:  the two examples are pronounced in the same way.  Some contexts require double quoted alien text in writing.

Alien text is always preceded by one of the alien text markers {\bf hoi, hue, lie, lao, lio, sao, sue}, whose grammar and semantics will be discussed below.  Alien text marked with
{\bf hoi} or {\bf hue} must be double-quoted.   The parser identifies blocks of alien text by looking for these markers (some of the markers have multiple functions and will not always be followed by alien text).

Examples of alien text in Loglan utterances will appear when we discuss the grammatical constructions that use them.

\subsection{Vocalic diphthongs}

In this section, we describe two-letter forms which may appear as the ``vowel" component of a syllable.

The consonants {\bf l, m, n, r} we call {\em continuants}.  A doubled continuant {\bf ll, mm, nn, rr} represents a syllabic continuant, which may serve as the vocalic component of a syllable.
A syllabic continuant may not be followed  or preceded by another instance of the same continuant without an intervening pause in speech.

We now consider how to pronounce sequences of regular vowels not separated by junctures.  The issue is how to resolve such a sequence into syllables.
The irregular vowel {\bf y} is usually a single syllable, except for occurrences of syllables {\bf iy} and {\bf uy} in rare structure words, which will be discussed later.

We first consider sequences of two regular vowels.  Some of these sequences are mandatory monosyllables, some are optional monosyllables, and some cannot be read as monosyllables.

There are four mandatory diphthongs {\bf ai, ei, oi, ao}.  The diphthong {\bf ao} has the irregular pronunciation of ``ow"  in English ``cow".   The pairs of letters {\bf ai, ei, oi} are not mandatory diphthongs when followed by {\bf i} without an intervening juncture:  {\bf aii} is grouped {\bf a-ii}.

The forms {\bf a-i, e-i, o-i, a-o} (the hyphens may be replaced with other junctures), called broken monosyllables, can only occur in names (we do not regard {\bf a-ii} and its kin as containing broken monosyllables, but here we are talking about more than two letters).  One may write any other pair of regular vowels separated by a juncture, with the effect of enforcing the two syllable pronunciation (where it is optional).

There are six optional diphthongs, made up of {\bf i} or {\bf u} followed by a regular vowel.  (forms {\bf iy} and {\bf uy} also occur in special contexts, to be discussed later).  If these are pronounced as a single syllable, initial {\bf i} is pronounced as English ``y" and initial {\bf u} is pronounced as English ``w".  The disyllable pronunciation can be compelled by writing
a syllable break.   Monosyllabic {\bf iu} cannot be followed without a juncture by {\bf u} and monosyllabic {\bf ui} cannot be followed without a juncture by {\bf i} (so, for example, if
{\bf iuu} is encountered it will be read {\bf i-uu}).  As a rule, the speaker has a choice when presented with an optional monosyllable of pronouncing it as one or two syllables; sometimes the context forces one of the pronunciations.

Other pairs of adjacent vowels are pronounced as two separate syllables;  the use of a glottal stop to separate the components of a disyllabic vowel pair is permitted, but not expressed in the orthography.  The glottal stop is {\bf not} allowed as an allophone of the pause phoneme;  all required pauses must be distinct, if sometimes brief.\footnote{Previous versions of Loglan do not allow the glottal stop to appear medially in disyllables (we allow it but also allow the traditional Loglan pronunciation, a smooth glide from one vowel to the other);  previous versions of Loglan allowed the glottal stop as an allophone of pause, and we do not.  Lojban uses the h sound medially in disyllables, which would be allowed for a Loglan speaker who chose always to use the alternative pronunciation of {\bf h}.}

A pair of identical adjacent vowels not pronounced as a monosyllable has the characteristic that one of the vowels must be stressed and the other unstressed.  This always holds for
{\bf aa, ee, oo} and sometimes holds for {\bf ii}, {\bf uu} (special rules stated above are designed to encourage pronunciation of the latter two pairs as monosyllables whenever possible!)

For a three-vowel sequence appearing in a predicate or name word, the general rule is that formation of monosyllabic
{\bf ii} or {\bf uu} is the highest priority (so in {\bf aii}, forming {\bf ii} wins over forming {\bf ai}, which in this context is not a mandatory monosyllable anyway, producing {\bf a-ii}), followed by formation of a mandatory monosyllable (recalling that {\bf i}-final mandatory monosyllables are not followed by {\bf i}; {\bf aoi} is grouped {\bf ao-i} and is not considered to contain a broken monosyllable):  e.g.,  {\bf aiu} is grouped {\bf ai-u}), followed by formation of an optional monosyllable (which is often an optional preference for the speaker;  the parser does exercise this preference).  An extreme example of speaker freedom is {\bf iue}, which the parser will resolve into two syllables {\bf iu-e} (choosing to group the first two when both pairs have the same precedence)  but which the speaker can resolve into two or three syllables in any of the three possible ways.

We present a  formal rule\footnote{The formal rule for reading long sequences of vowels in names appearing in Loglan 1 is incredible, as it requires indefinite lookahead;  of course it was also really intended only for use with three or perhaps four vowels.} for reading the next syllable from a sequence of  regular vowels of any length written without junctures, which is used in resolving predicates and names into syllables.  A mandatory diphthong is read as the first syllable if it is present (recalling that if the pair of vowels ends in {\bf i} and is followed by another {\bf i} it is not a mandatory diphthong); a single vowel is read if it is not initial in a mandatory diphthong and the next two vowels form a mandatory diphthong;  if neither of the previous two cases holds an optional diphthong is read by preference by the parser (though a disyllabic reading is permitted); as the final option a single vowel is chosen, subject to the rule that {\bf i} or {\bf u} (when not part of a diphthong) cannot be followed by an intervening juncture and a consonantal occurrence of the same vowel (this situation will cause parse failure).  The process of resolution of the first syllable from a  stream of vowels is repeated until the stream of vowels is completely resolved into syllables.
This rule may look forbidding, but it should be noted that sequences of four or more vowels are quite rare in Loglan predicates or names, so the two and three vowel accounts will usually be quite enough.

There is a separate rule, used in resolving certain structure words,  in which a sequence of vowels of even length is parsed into vowel pairs, each of which is read as monosyllable or disyllable as the rules require or permit.  There is a further special rule for certain structure words with three-vowel sequences, which does not conform with the rule stated above for resolving vowel sequences in predicates and names, which will be stated when these structure words are described.

When the vowel component of a syllable is read, this will be either a syllabic continuant, or {\bf y}, or a vowel or vowel diphthong chosen using the appropriate one of the rules above.

One should note that the rules presented here are not of interest to readers and writers, speakers and listeners, very directly;  but they are certainly of interest to word makers, and might briefly be of interest to a dictionary reader encountering a word for the first time.  Such phonetic rules exist in natural languages, whose speakers are not necessarily even aware of them;  one could imagine that the native Loglander, though her speech will conform perfectly to the rules stated above, will not know much about them unless she is a grammarian!

\subsection{Consonant grouping}

There are different rules for syllable-initial and syllable-final consonant grouping.  It is worth noting that consonant grouping only occurs in regular Loglan text in predicates and names.   Syllables with final consonants also occur only in predicates and names.

These are governed by two sets of phonetic rules.  There is a list of permitted initial pairs of consonants\footnote{The initial pairs are {\bf bl}  {\bf br}  {\bf ck}  {\bf cl}  {\bf cm}  {\bf cn}   {\bf cp}  {\bf cr}   {\bf ct}   {\bf dj}   {\bf dr}   {\bf dz}  {\bf fl}  {\bf fr}   {\bf gl}   {\bf gr}   {\bf jm}   {\bf kl}  {\bf kr}   {\bf mr}   {\bf pl}  {\bf pr}   {\bf sk}  {\bf sl}   {\bf sm}  {\bf sn}  {\bf sp}   {\bf sr}  {\bf st}  {\bf sv} {\bf tc}  {\bf tr}  {\bf ts}  {\bf vl}  {\bf vr} {\bf  zb}  {\bf  zl}  {\bf zv}}.  The initial group of consonants in a syllable consists of a single consonant, or a permissible initial pair of consonants, or a triple of consonants in which each adjacent pair of consonants is an initial pair.\footnote{The rule for initial consonant groups appears in Notebook 3.}  We refer to a pair of consonants which would be a permissible initial pair if an intervening juncture were removed as a ``broken initial pair".

There is a list of forbidden medial pairs\footnote{The impermissible medial pairs consist of all doubled consonants, any pair beginning with {\bf h}, any pair both of which are taken from {\bf cjsz}, {\bf fv}, {\bf kg}, {\bf pb}, {\bf td}, any of {\bf fkpt} followed by either of {\bf jz}, {\bf bj}, and {\bf sb}.}
and a list of forbidden medial triples\footnote{{\bf cdz}, {\bf cvl}, {\bf ndj}, {\bf ndz}, {\bf dcm}, {\bf dct}, {\bf dts}, {\bf pdz}, {\bf gts}, {\bf gzb}, {\bf svl}, {\bf jdj}, {\bf jtc}, {\bf jts}, {\bf jvr}, {\bf tvl}, {\bf kdz}, {\bf vts}, and {\bf mzb}}. 
 These cannot occur even if broken by a juncture.  

There can be one or two final consonants in
a syllable, which cannot be part of a forbidden medial pair or triple whether together (if there are two of them) or combined with consonants taken from the beginning of the following syllable.  A pair of final consonants cannot be a non-continuant followed by a continuant (this appears to be pronounceable only as a separate syllable).  A final consonant cannot be followed by a regular vowel or a syllabic continuant, even with an intervening juncture (in other words, such a consonant should be read as part of the following syllable).\footnote{The rules forbidding final consonants from participating in illegal medial pairs or triples are found in our sources.  The rule forbidding a pair of final consonants from being a non-continuant followed by a continuant seems quite natural but is ours;  no word was proposed that violated it, in any case.  Other rules that we state depend on a precise definition of the syllable, which appears nowhere in Loglan sources, although the notion of syllable is important in the definition of borrowed predicates in Notebook 3.}

A consonant in either of these sorts of groups which is a continuant cannot be adjacent to another copy of the same continuant, within or without the cluster, even if separated by a juncture.
An initial consonant triple cannot be followed by a syllabic continuant at all.

\subsection{The Loglan syllable}

A Loglan syllable consists of three parts. 

 There is an optional initial group of one, two or three consonants governed by rules stated in the previous subsection.  

This is followed by the mandatory vocalic component of the syllable, which is either a pair of identical continuants, a single regular vowel, a vowel diphthong, or {\bf y} ({\bf iy}  or {\bf uy} occur only in syllables (C){\bf iy} and (C){\bf uy} which are directly allowed as units in structure words but not supported in the formal syllable definition).

This is followed optionally by one or two final consonants, for which rules are stated above, with the additional remark that unless the syllable is of the shape CVC with the vowel regular,   no final consonant in the syllable (neither of them, if there are two)  may be readable as standing at the beginning of a following syllable (in other words, except in the case of CVC syllables, the automatic placement of syllable breaks where an explicit juncture is not present  is as early as possible;  but a CVC syllable is preferred to a CV syllable where possible).  Explicit junctures will override the preferred syllable breaks, but there are subtle rules about where explicit junctures can be placed:  sometimes they will simply cause parse errors.\footnote{The subtleties have to do with the fact that a borrowed predicate cannot resolve into djifoa (see below for these terms);  an apparently legal borrowing predicate written with explicit junctures will be rejected if moving some of the junctures would create a legal complex predicate.  These are issues which mostly affect the word designer.  If you are trying to write a complex predicate from the dictionary with explicit syllable breaks, make sure that the breaks you supply conform with djifoa boundaries and these issues will not arise.}

It is worth noting that previous versions of Loglan had no official formal definition of the syllable, though the syllable did play a role in the definition of some word classes.\footnote{The lack of felt need for a formal definition of the syllable  may have come from the fact that structure words and complex predicates resolve into units which are not themselves necessarily syllables, but which are expected to conform with syllable boundaries;  it is with the introduction of borrowed predicates that a precise notion of the syllable became essential to someone who wanted to parse words, and once this notion was in hand, it became natural to require that names (which were just consonant final strings of phonemes in earlier versions of Loglan) be resolvable into syllables as well.  The accuracy of our implementation can be gauged by the fact that almost all words in the dictionary parsed correctly when we ran a test, and the ones which did not parse had recognizable errors which needed to be fixed.  It should be noted that we cannot have three final consonants in a syllable, and this is not uncommon in names.  This can usually be fixed by doubling a continuant, as in {\bf Hollmz}, {\bf Marrks}, but some names may be found to be definitely foreign.}

\subsection{Pauses and whitespace:  general principles}

A pause is always expressed as either a comma followed by whitespace (which must be followed by a phonetic block) or simply whitespace, which must be followed by a phonetic block.
The former is always a pause;  the latter may sometimes not be a pause.

Whitespace at the beginning or the end of alien text must represent an actual pause.

Whitespace after a consonant and/or before a vowel must represent an actual pause.

Names are the only consonant-final words in regular Loglan text, and they must be followed by comma-marked pauses, terminal punctuation, or end of text, or by whitespace followed by another name word or the structure word {\bf ci}.

Logical connective words of class A, some but not all of which are vowel-initial, and sentence connectives of classes I and ICA must be preceded by comma-marked pauses.  The APA and IPA logical and sentence connectives, to be discussed below, and the ICA and ICAPA sentence connectives must be followed either by the suffix {\bf fi} or a comma-marked pause.  The issues in this paragraph are handled entirely in the grammar section.

Words quoted with {\bf liu} must be followed by a comma-marked pause (or terminal punctuation or end of text).

If the final syllable of a structure word is stressed and it is followed by a predicate, it must be followed by a comma-marked pause.  This rule is of course only enforced in our orthography if we actually write explicit stress.\footnote{This rule goes back to the beginnings of Loglan, but as no earlier parser had explicit indications of stress, there was never any occasion for an earlier parser to enforce it.}

In general, certain lexicographic issues tend to force explicit comma-marked pauses.  If a pause in a sequence of structure word syllables breaks a word, it must be explicitly comma-marked as a rule, since if it were written as mere whitespace, not pausing would cause a different interpretation of the utterance.   There will be a discussion of multi-syllable structure words in the lexicography section which lays out the situations under which this issue can occur.\footnote{In Lojban, apparently all structure word syllables are separate words, but this is not the case in Loglan.}

The ``false name marker" problem creates further need of explicit pauses, which will be discussed below.

This version of Loglan supports a form of orthography known as ``phonetic transcript" in which no whitespace appears but comma-marked pauses.  This means that we require that
in every place where we can or must pause, it must be possible to replace whitespace with a comma-marked pause.  It is mostly but not entirely true that every place whitespace is written
is a place where one {\em can\/} pause:  it is possible to create situations with the APA and IPA connectives in their legacy form where a whitespace that one can write cannot represent a pause, and there is a rule that one should not pause after the structure word {\bf ci} before a consonant unless the pause is comma-marked.  This whitespace can, however, be omitted.  Whitespace which does not represent pauses can always be omitted, though in the case of whitespace after predicates, this may require the writer to insert explicit indications of stress so that the reader can tell where the predicate ends.  Whitespace which cannot be omitted can always be replaced with an explicit comma-marked pause.

Because we have phonetic transcript, we do not need a special notation for expressing pronunciation.\footnote{Brown's phonetic notation in the sources is {\em ad hoc} and reveals such things as very inconsistent notions about syllable breaks.}

\section{Phonetic word forms}

\subsection{The four forms, and general principles}

There are four basic word forms in Loglan: 

\begin{enumerate}

\item Items of alien text (with their preceding alien text markers), already described above.

\item Phonetic names (name words accompanied with their required preceding pauses or name marker words with intervening optional pause).

\item  Structure words

\item Predicates, further subdivided into complexes and borrowings.

\end{enumerate}

These classes of words have general characteristics which allow us to distinguish them  We leave aside the case of alien text which we have already analyzed.

\begin{enumerate}

\item Name words are the only consonant-final words in Loglan (other than alien text).  They are thus followed by pauses in speech (and usually by explicit pauses in writing).  This makes the right boundary of a name word easy to recognize.  One must also pause at the beginning of a name word, unless it is preceded by one of a limited class of name markers.  There are few contexts in which a name word can appear without an immediately  preceding name marker word, and if a name word happens to include a phonetic copy of a name marker word (a ``false name marker") it {\em must\/} be immediately preceded by a name marker word (an intervening pause being permitted).  Where a name marker word occurs which is not immediately followed by a name word but followed by a name word starting later, a comma-marked explicit pause (or terminal punctuation) must appear somewhere between the name marker word not serving as such and the following name word:  this prevents pronunciation of the text in a way which causes everything between the name marker word and the end of the later name word to be construed as a single longer name word.

\item Predicates end with a regular vowel (so they are not names), are penultimately stressed (with qualifications to be stated later);  this allows the right boundary of a predicate word to be recognized in speech, or in phonetic transcript), and contain adjacent consonants (in some cases the pair of consonants may be separated by {\bf y}).  The left boundary of a predicate is determined
by the fact that it must begin CC or (C)V$^n$C({\bf y})C.  in the latter case with some conditions ensuring that the (C)V$^n$ cannot be construed as a structure word.  Predicates
can more rarely begin (CVV{\bf y})$^n$((C)V$^m$)CC.

\item Structure words (Loglan {\bf cmapua}) are not names or predicates (actually some are semantically names or predicates, but this is a matter for the grammar).  In addition, we specify that they resolve into phonetic units of the shapes V, VV, CV, CVV (where the VV may be a monosyllable or a disyllable, and {\bf iy}, {\bf uy} are permitted), and the rare Cvv-V, where the vv is a monosyllable (mandatory or optional, but in any case pronounced as such).
Further, a V unit may only occur initially, and any structure word which contains a VV unit consists entirely of VV units (except that we allow words of the shapes
{\bf no}-VV and VV-{\bf noi}).  A sequence of VV units is resolved into syllables by pronouncing each unit as one or two syllables as the grammar requires or permits.  Note that the unit cmapua are not necessarily syllables, but their boundaries are syllable boundaries in a structure word.  Where a structure word is followed by a predicate beginning with CC,
stressing its last cmapua unit might create the possibility of reading the last cmapua unit and the first syllable of the intended predicate word as a predicate:  to avert this, we require that a finally stressed structure word must be separated from a following predicate word (not just a CC-initial one) by a comma-marked pause.

\end{enumerate}

It should be noted that the classes of words here should be qualified as phonetic names, phonetic predicates, and phonetic structure words, as there are cases where ``words" which are phonetically of one of these shapes are used in a way associated with one of the others.

\subsection{Phonetic Names}

We distinguish between a name word, such as {\bf Djan}, and a phonetic name, such as {\bf la Djan Braon}, which comes equipped with the name marker word or initial pause that a name word requires in its context, and may contain more than one name word after the name marker.

A name word is a phonetic block which resolves into syllables, the last of which ends in a consonant (possibly with a final stress).

A possible name word is a name word, or a name word modified by insertion of whitespace at junctures preceded by a vowel and succeeded by a consonant (so that the whitespace does not necessarily represent a pause).

A marked name is a name marker word followed by a consonant initial name word, possibly with intervening whitespace between the two.

A falsely marked name is a name word with a proper final segment which is a marked name:  that is, it is a name word with a false name marker in it.  Notice that a phonetic
occurrence of a name marker word is not a false name marker unless what follows it is a consonant-initial name word.\footnote{In early versions of Loglan, falsely marked names were simply forbidden, but {\bf la} is very common.   Later, they were admitted and some effort was made to avoid problems with them.  The idea that a falsely marked name must be marked appeared in the context of implementation of serial names (falsely marked names in a serial name had to be marked with {\bf ci}; we required after 2013 that predicate components of serial names be marked
with {\bf ci} as well to avoid the need for two pause phonemes to avoid confusion of serial names with sentences.)  We extended the idea that falsely marked names must
be marked to all contexts, and in addition reduced the distribution of unmarked names to very few contexts by forbidding unmarked vocatives.}

The name marker words are {\bf la, hoi, hue, ci, liu, gao, mue}.  A subtle point is that {\bf ci} is only a name marker when followed by a pause (an explicit comma-marked pause
or whitespace followed by a vowel):  this allows us to avoid difficult-to-predict needs for pauses after the many uses of {\bf ci}.  It does mean that when whitespace is written after {\bf ci} before a consonant, we presume that the speaker does not pause.

A phonetic name (including its name marker or preceding pause if there is one) is of one of the following kinds:

\begin{enumerate}

\item a marked name as described above (a name marker followed by possible whitespace followed by a consonant-initial name word).

\item  a vowel initial name word which is not a falsely marked name, or  a comma-marked pause followed by a name word which is not a falsely marked name.

\item a name marker followed by optional whitespace or explicit pause followed by a name word, with the additional proviso that the optional whitespace or pause must be present if the name word is vowel-initial.

\end{enumerate}

To any of these, a series of name words marked with {\bf ci} and prenames which are not falsely marked, separated by whitespace, may be appended as part of the phonetic name,
so {\bf la Djan Braon} is a phonetic name, and so is {\bf la Pierr ci, Laplas}.  In the last example one pauses both before and after {\bf ci};  the second comma must be written, and the use of {\bf ci} is necessary because {\bf Laplas} is a falsely marked name.

It is then required that this be followed either by an explicit pause, terminal punctuation, end of text, or whitespace followed by {\bf ci} followed by a predicate of class {\tt predunit}, a peek forward at the grammar.  Note that the following explicit pause or punctuation or {\bf ci} phrase is not part of the phonetic name:  this is information about the context in which a phonetic name can appear.

Names may contain explicit junctures, including ones which form broken monosyllables, and junctures may be required features of name words:  {\bf Lo-is} and {\bf Lois} are different names.\footnote{This goes back to previous versions of Loglan, but we use hyphens instead of close commas.}

\subsection{Phonetic structure words (cmapua)}

Phonetic structure words are sequences of cmapua units as sketched above;  we give more details.

Cmapua units are of the shapes V, VV, CV, CVV, Cvv-V, where vv stands for a monosyllable and VV (in VV and CVV units) includes {\bf iy}, {\bf uy}.  {\bf y} is also accepted as a V unit.\footnote{The practical reason for allowing {\bf y} to occur above is to support names of the letter {\bf y}, legacy {\bf yfi} and modern {\bf ziy} (pronounced ``zyuh"!).  It seemed more principled to install general phonetic conditions that allowed these forms than to allow them individually by fiat.}

A phonetic structure word is a string of cmapua units.  A cmapua unit not of VV form cannot be followed by a vowel, even with an intervening juncture:  this helps to enforce
the condition that vowel-initial words must be preceded by pauses in speech, represented at least by whitespace. 

Each cmapua unit is restricted by lookahead tests for other classes.  A cmapua unit cannot be an alien text marker actually followed by alien text.  A cmapua unit cannot be an occurrence of {\bf li} or {\bf kie} which actually stands at the beginning of a quotation or parenthetical free modifier (for the uses of these words, see the grammar section).  A cmapua unit cannot be a name marker followed with optional pause by a possible name word (this excludes both
actual phonetic names and strings which could be misread as phonetic names by ignoring instances of whitespace one of which should be made an explicit pause).\footnote{This is our definitive solution to the false name marker problem.  Difficulties created by the markers other than {\bf ci} should generally be easy to anticipate, by following style rules such as ``always pause after a predicate name".  The word {\bf ci} presented special difficulties as a name marker because it has a wide variety of uses some of which have nothing to do with names.  Viewing it as a name marker only when followed by a pause seems to be the final refinement of our solution.}   A cmapua unit
cannot stand at the beginning of a legal predicate (the parser  does a lookahead test which identifies strings which can only be predicates if they are grammatical and not possible phonetic names;  we describe this test below at the beginning of the discussion of predicates).  

A cmapua unit cannot be stressed and then followed by optional whitespace and the start of a consonant-initial predicate (as detected by the test above).

We then provide a phonetic test for the logical and sentence connective classes which must be preceded by a pause.  A phonetic connective starts possibly with whitespace followed by possibly by an occurrence of 
{\bf no} (not starting a predicate) followed definitely by a regular V syllable or {\bf ha}, {\bf nuu} (not starting a predicate), not followed by a vowel, and not followed by {\bf fi}, {\bf ma}, or {\bf zi}, which would make a V unit into a legacy letteral (none of these starting a predicate).

We can now describe a phonetic structure word.  It takes one of five forms.

\begin{enumerate}

\item a VV unit (here and in all clauses here including {\bf iy}, {\bf uy}) followed by {\bf noi} ({\bf noi} not starting a predicate).

\item {\bf no} (not starting a predicate) followed by a VV unit

\item a sequence of VV units

\item a regular or irregular V unit

\item an optional regular or irregular V unit followed by a sequence of one or more consonant-initial cmapua units

\end{enumerate}

A cmapua unit absorbs a following juncture.

Each cmapua unit is blocked from being followed by a vowel without intervening whitespace or by optional whitespace then a phonetic connective;  this forces explicit pauses before the logical and sentence connectives.

The phonetic structure words defined here have boundaries dictated entirely by phonetic convenience;  the actual boundaries of cmapua words in the proper sense are dictated by rules stated in the lexicography chapter.  Some words which appear in other structures, such as the name markers, alien text markers, and {\bf y}, and some others, are from a lexicographic standpoint structure words and do look like them phonetically.

\subsection{Primitive Predicates and Combining Forms (djifoa)}

The basic ``native" predicates of Loglan are of the five letter forms CCVCV and CVCCV.  The original stock of native predicates was generated by a rather {\em ad hoc\/} statistical comparison with words in major natural languages on which we have no intention of commenting, as we expect it never to be used again. 

Each of the native predicates has one or more combining forms (originally called ``affixes", a  deprecated usage;  now usually called {\em djifoa\/}, the Loglan word for these forms).

Each five-letter native predicate has a djifoa formed by replacing its final vowel with {\bf y}.  This does mean that five letter predicates which have the same final vowel must be semantically very closely related (words for animals and languages can be given fine shades of meaning by adjusting the final letter;  we do not intend to create further declensions of this kind, but we see nothing wrong with the ones we have).

In addition, many five-letter djifoa have one or more than one associated three letter djifoa, of one of the forms CVV, CVC, or CCV, which is formed by choosing three letters from 
the five letter djifoa in order of their occurrence.  The process of choosing these djifoa is not likely to be modified or extended at this point, though there is some tension about the ones with doubled vowels which force stress.

There is also a short list of three-letter djifoa built from CV cmapua, by appending {\bf r}, which we supply:
\begin{description}
\item[fer:] from {\bf fe}, five
\item[for:]  from {\bf fo}, four
\item[fur:] from {\bf fu}, 3d place passive
\item[jur:] from {\bf  ju}, 4th place passive
\item[ner:]  from {\bf  ne}, one
\item[nir:]  from {\bf ni}, zero
\item [nor:]  from {\bf  no}, logical negation
\item[nun:]  from {\bf nu}, 2nd place passive, before{\bf  r}
\item[nur:]  from {\bf nu}, 2nd place passive, not before {\bf r}
\item[por:]  from {\bf po}, state particle
\item[rar:]  from {\bf ra}, all
\item[rer:]  from {\bf re}, most of
\item[ror:]  from {\bf ro}, many of
\item[ser:] from {\bf se}, seven
\item[sor:]  from {\bf  so}, six
\item[sur:]  from {\bf  su}, at least one of, some
\item[ter:] from {\bf te}, three
\item[tor:]  from {\bf to}, two
\item[ver:]  from {\bf ve}, nine
\item[vor:] from {\bf vo}, eight
\end{description}

Further, every CV cmapua unit  has a corresponding CV{\bf h} djifoa.  CVV cmapua  may be extended with {\bf (h)y} (not with {\bf n} or {\bf r}) and used as djifoa:  thus
{\bf zaiytrena}, A-train.  This must be done with care as there may be djifoa derived from cmapua of the same shape.

Loglan complex predicates (native compound predicates) are built from sequences of these djifoa (in which the last item may be a full primitive or borrowed predicate).  There are also borrowing djifoa built from borrowed predicates, which we discuss in the next section.

It is necessary to supply additional phonetic glue so that sequences of these djifoa actually can produce predicate words.  Each three-letter djifoa has an alternative form
with suffixed {\bf y}.  CVC{\bf y}  djifoa can be broken into syllables either as CVC-{\bf y} or as {CV-C{\bf y}.  CVV djifoa in initial position will ``fall off":  so CVV{\bf r} is available as an alternative form (which will form a consonant pair with the following djifoa or predicate word), and CVV{\bf n} is available as an alternative form when it is followed by {\bf r}.  Other problems
are that CVC djifoa may not occur in final position, and CVV djifoa which have a doubled vowel can only occur in final or in penultimate position, because a predicate word can only contain one stress in a penultimate position.  We propose for CVV{\bf y} the alternative form CVV{\bf hy}:  a predicate starting this way will not be confused with any other kind of word, and this should be easier to pronounce distinctly.

A pronuncition difficulty with CVV{\bf r} extended djifoa (at least for English speakers) is fixed by a permission which will probably seldom be expressed in writing, but can be:
CVV{\bf r} can take the alternative form CVV-{\bf rr} when the VV is a mandatory monosyllable.  If this is stressed, the stress falls on the VV and the vocalic continuant is an additional unstressed syllable before the final unstressed regular syllable.

Broken djifoa forms are also available to the parser, obtained from legal djifoa by inserting junctures (C-CV or C-VV  or CV-C, for example).  These are used to enforce the condition that borrowed predicates cannot decompose into djifoa, and it should not be possible to convert a complex (especially one which is illegal for reasons peculiar to complexes) to a legal borrowed predicate by moving junctures around.

The general point about syllable breaks is that while djifoa are not syllables, the boundaries between djifoa will be syllable breaks in a legal complex.  Internal breaks are sometimes
optional:  a CVV djifoa with an optional monosyllable has two possible forms.  The CVCCV five letter predicates may admit two forms CVC-CV and CV-CCV if the medial CC is an initial pair.
The parser prefers the first form for technical reasons;  it can be coerced by writing an explicit syllable break, and the latter version is often easier to pronounce.

\subsection{Phonetic predicate words:  general principles, and recognizing the beginning of a predicate}

A phonetic predicate word ends with a regular vowel (so it is not a phonetic name), contains an adjacent pair of consonants (so it is not a structure word), and has
penultimate stress (with the exception that an additional unstressed syllable with {\bf y} or a vocalic continuant may intervene between the stressed and the final syllable), so that one can tell where it ends.

The part of the predicate before the consonant pair can be null (the predicate may start with an initial pair or triple of consonants).  This will be followed by a regular vowel,
and there are no CC(C)V(V) predicates, in which the initial consonant group is followed just by one or two vowels.

The initial segment before the consonant pair can be an optional consonant followed by one to three regular vowels\footnote{Previous versions of Loglan have allowed arbitrarily long sequences of vowels after the initial consonant in this case, but these have never been used and I like the bound on lookahead in this test obtained by forbidding more than three vowels.}   This is the only alternative which can occur in a borrowed predicate.  This can be ensured if the string starting with the consonant group doesn't satisfy the conditions given above to be the start of a predicate:  the predicate must then begin with the initial (C)V(V)(V) (it cannot begin with part of the vowel sequence because one must pause before a vowel-initial word).  It can also be ensured if the final syllable of the initial (C)V(V)(V) is stressed:  if it were the end of a cmapua, it could only be followed by a predicate,
and one cannot have a stressed cmapua followed by a predicate without pause.  There is a further technical issue, which is explained below in the discussion of borrowing djifoa:
for technical reasons, if the (C)V(V)(V) is not of one of the forms CV or CVV, the consonant group cannot be an initial pair followed by a regular vowel.

In a complex, there are other possibilities.  A complex might start with one or more CVV{\bf y} djifoa; no other sort of word can start this way.  It might start with
a CVCC{\bf y} djifoa;  no other sort of word can start this way.  It might start with a an extended CVC djifoa:  CVC{\bf y}. Again, no other Loglan word can start in this way.

We describe a  lookahead test:  a string which is already known not to be a possible phonetic name cannot be anything but a predicate (or ill-formed) if one of the following things are true (and one of these things will be true of any actual predicate):

\begin{enumerate}

\item  It begins with a permissible initial group of two or three consonants  and is followed by a regular vowel, but not by one or two vowels (with possible junctures) followed by a non-character, nor by a stressed vowel followed by a vowel not in a diphthong (short words of the shapes CC(C)V(V) are not predicates).  Junctures may appear after the vowels.  This clause of the test ignores any junctures which may appear in the initial group of consonants (to support its use in following clauses).

\item  It begins with CV(V) followed by a consonant group, with either the final syllable in the CV(V) [which might extend to a juncture in the consonant group] stressed or the part of the word beginning at the consonant group not meeting the test to start a predicate above.  Junctures may appear after the vowels, and there might be a juncture in the consonant group, which will be ignored in testing whether it begins a predicate.

\item  It begins with (C)V(V)(V) followed by a consonant group which is not an initial pair (even one broken by a juncture)  followed by a regular vowel, with either the final syllable in the (C)V(V)(V) [which might extend to a juncture in the consonant group]  stressed or the part of the word beginning at the consonant group not meeting the test to start a predicate above.  Junctures may be inserted after the vowels or there might be one in the consonant group, which will be ignored in testhing whether it begins a predicate.

\item It begins with a consonant and a regular vowel, followed by a regular vowel followed by {\bf y} [or {\bf hy}], or a consonant or pair of consonants followed by {\bf y} (with possible intervening junctures).  Both of these initial sequences
are possible beginnings for a predicate complex, and what follows the CV$^n$ could not start any legal word in that context.

\end{enumerate}

The point of including this list is making it clear that it is fairly easy to see or hear the beginning of a predicate word (though the detailed description of the cases is admittedly annoying!)

\subsection{Borrowed Predicates and Borrowing Djifoa}

After all that about djifoa, we discuss borrowings first!

A borrowing must resolve into syllables.  It must end in a regular vowel.  Any explicit stress must be on the second-to-last syllable, not counting syllables with vocalic continuants (one such unstressed syllable may intervene between the stressed syllable and the unstressed final syllable).     The end of a predicate is determined by either non-characters such as whitespace or punctuation, or by an explicit stress.  A borrowing cannot be followed without intervening whitespace or explicit pause by a vowel, nor by optional whitespace followed by a connective.  An explicit stress  may force monosyllabic pronunciation on a last syllable which could otherwise be pronounced as a disyllable.   There can be only one explicit stress in a borrowing.  The deduced stresses in doubled vowels do not play a role in parsing borrowings, as disyllabic doubled vowels are forbidden in borrowings.  It is permissible to write a borrowing with explicit junctures, but this cannot change the meaning of the word, and broken monosyllables are not permitted.  A borrowing must parse correctly in the absence of explicit junctures.

The beginning of a borrowed predicate must pass the test for beginnings of predicates given above.  Since borrowed predicates cannot contain {\bf y}, this enforces the condition that there be a pair of adjacent consonants in a borrowed predicate.  We reiterate that a borrowed predicate cannot have the shape CC(C)VV (which is of course not resolvable into djifoa, so could not be the shape of a complex predicate).

There are some restrictions on the phonetics of borrowed predicates.  Borrowed predicates may not contain {\bf y} or any doubled vowel other than monosyllabic {\bf ii} and {\bf uu}.\footnote{Forbidding doubled vowels in borrowings was an action of ours:  admitting them made reasoning about the penultimate stress in a borrowing more difficult.  Exactly one predicate {\bf alkooli} had to be changed to the better {\bf alkoholi}.}
Borrowed predicates may contain syllables with vocalic continuants.  These never follow a vowel and so far never precede a consonant, and such a syllable is always medial (not first or last).  There may not be two such syllables in succession, and such a syllable cannot be stressed. The point of such syllables is that they cannot occur in complexes (with one minor exception which cannot be confused with occurrences of syllabic continuants in borrowings:  a CVV{\bf r} djifoa may be expressed as CVV{\bf rr}, in which the syllabic continuant follows a vowel).  A borrowed predicate may not resolve into djifoa, including resolutions involving broken forms in which junctures are misplaced.

Borrowing djifoa are formed from borrowings by adding final {\bf y} and moving the stress to the final syllable of the borrowing (still penultimate in the borrowing djifoa).  It is permitted to stress the penultimate syllable in a borrowing djifoa and pause after the {\bf y}, if what follows the borrowing djifoa contains a penultimate stress.  Thus
one may pronounce {\bf bakteriyrodhopsini} as {\bf bakteri'y, rodhopsi'ni}, but one may not pronounce {\bf iglluymao} with a pause.  The stress shift is a strong signal that one
is not saying {\bf bakte'ri} but its djifoa.\footnote{Nothing in this situation is due to me!  The strange provision to pause after a borrowing djifoa is in Loglan 1.  The definition of borrowing djifoa we use was given in the 1990's.  I do not know if anyone noticed the stress shift caused by the change in the definition of borrowing djifoa, but it is all a logical consequence of the way things stood at Brown's death.}

The reason for the difference between the treatment of CV(V) and other (C)V(V)(V) prefixes in the predicate start rules  is caused by the danger that a borrowing djifoa for a (C)V$^n$CCV predicate might turn into a cmapua followed
by a stressed CCV{\bf y} djifoa when a borrowing djifoa was formed (this requires the CC to be an initial pair, of course).  This is not a danger when the consonant is present and $n = 1$ or 2 because a borrowing will not be of the shape
CV(V)CCV with the CC initial.  Thus a string (C)V$^n$CCV with the CC an initial pair is not allowed to start a predicate unless the initial consonant is present and $n$ is 1 or 2.

\subsection{Complex Predicates}

A complex predicate is formed from a sequence of djifoa in which the final element may be a full primitive or borrowed predicate (and if it is a djifoa may not be CVC, nor may it be extended with {\bf r}, {\bf n} or {\bf y}, nor may it contain {\bf y} at all).   A complex may not be formed entirely from CVV{\bf y} djifoa and a final CVV (this prevents forms without adjacent consonants;  adjacent consonants may be separated by {\bf y} as in {\bf mekykiu}).  A complex must pass the predicate start test.
CVV djifoa may need to be extended with {\bf r}, {\bf n}, or {\bf y} when they appear in initial position to keep from being read as cmapua.  CVC djifoa in initial position of a complex not of one of the six-letter forms CVCCVV or CVCCCV must
be extended with {\bf y} if the final consonant of the djifoa and the following consonant would form an initial pair (a juncture does not affect this):  this prevents the complex
from being read as a CV djifoa followed by a borrowing.\footnote{This rule superseded the historical {\bf slinkui} test, which prevented the CV from falling off the front of a CVCC... complex with the CC an initial pair by forbidding the formation of borrowed predicates obtained by extending a complex initially with a single consonant forming an admissable pair:  instead of forbidding {\bf paslinkui} in favor of {\bf pasylinkui}, {\bf paslinkui} was permitted and {\bf slinkui} was forbidden to be a borrowing.}  Thus {\bf ficynirli}, mermaid.  A CVC djifoa may need to be extended with {\bf y} to prevent formation of an illegal consonant group with the following consonant.   Thus {\bf mekykiu}, eye doctor.  Regular djifoa must be extended with {\bf y} when they appear before borrowing djifoa:  since borrowings cannot contain {\bf y}, this gives us precise information about their boundaries.

Complexes must have penultimate stress among those syllables not containing {\bf y}:  an unstressed syllable containing {\bf y} (or a syllable {\bf rr} serving as glue to a stressed CVV) may intervene between the stressed syllable and the unstressed final syllable.  The parser does know about the doubled vowel stress rule, and will not accept a complex with a CVV with a doubled vowel unless one of the vowels can carry the penultimate stress.  If a CVV with doubled vowel is followed by a CVV with optional monosyllable, the monosyllabic pronunciation is forced;  otherwise a CVV with optional monosyllable gives the speaker a choice about where to place the stress.  The parser determines where a predicate ends either by the occurrence of explicit stress or by the occurrence of a non-character from which it back-figures the location of the stress.  The only stresses in a predicate are its penultimate stress and optionally stresses in its borrowing djifoa (mandatory if the borrowing djifoa is followed by a pause).

A complex may not be followed without intervening space (a juncture doesn't help) by a vowel.  A complex may not be followed by whitespace followed by a connective.

An alternative formulation allows the formation of a complex from a sequence of cmapua units and predicate words in which the last item is a predicate word, with successive items 
separated by the ``word" {\bf zao}, optionally flanked on either side by whitespace or comma-marked pauses.  This form might be used to avoid borrowing djifoa.  I can also imagine its use to clarify the meaning of a complex by replacing its constituent djifoa with the corresponding predicates in full.\footnote{This is a proposal of John Cowan.}

Our general view is that the replacement of a djifoa in a complex by another djifoa for the same predicate or even by the full predicate linked with {\bf zao} gives another form for the same predicate word:  such forms should not appear as separate dictionary items.

\subsection{Phonetic Quotes and Parenthetical Expressions}

A phonetically valid quoted utterance begins with {\bf li} and ends with {\bf lu} with optional pauses after {\bf li} and before {\bf lu} and with the intervening text, which must be a phonetic utterance, optionally enclosed in double quotes (the quotes being between {\bf li} and {\bf lu}).  Replace {\bf li} and {\bf lu} with {\bf kie} and {\bf kiu} and optonal use of quotes with optional use of parentheses (opening with an open parenthesis, closing with a closing parenthesis, and you have the rule for spoken parenthetical expressions.\footnote{The use of punctuation here is a proposal of ours:  it is quite natural but did necessitate the phonetic parser knowing about these forms.}

We give examples:  {\bf li, la Djan, lu}; {\bf li ``la Djan", lu}; {\bf kie (ji cluva mi) kiu}.

\subsection{Phonetically valid utterances}

A phonetically valid unit phonetic utterance is a  phonetic name, phonetic structure word, marked alien text item, phonetic predicate word, quoted or parenthesized expression, hyphen, or ellipsis, possibly with initial whitespace.  A phonetic utterance is a sequence of unit phonetic utterances, explicit comma pauses, and items of terminal punctuation.  Any Loglan utterance must be a phonetic utterance:  of course it must also be grammatically correct, a matter for subsequent chapters.  \footnote{We note that CCV djifoa are alse unit phonetic utterances, strictly so that they may be quoted with {\bf liu}.  CVV djifoa are also cmapua, and CVC djifoa are also names, so they can be {\bf liu}-quoted without special ceremony.}

It is useful to be aware that the parser proceeds in effect in two passes:  it checks an entire utterance for phonetic validity, then checks whether it is grammatical in another pass.

\section{Historical and philosophical note}

Loglan phonetics is to our mind rather weird and wonderful.

James Jennings has commented that the choice to recognize word classes by patterns of consonants and vowels in the first instance was the ``original sin of the language".  Perhaps so, but once this sin was committed, it could not be undone without discarding everything and creating a different language.

1975 Loglan had a very simple procedure for resolution of words, but a method for construction of new predicates which was ultimately found unsatisfactory.  The great morphological revolution which consisted in introducing complex predicates as predicates built from djifoa and borrowings as all the phonetically acceptable predicates which were not complexes made the definition of the predicate much more complicated, and more of a challenge for the parser builder.

The problem of false name markers also led to a certain amount of excitement, once it was decided that {\bf la} was too common to ban from names.

We find Loglan phonetics charming as well as weird;  the language has a definite phonetic flavor and avoids monotonous regularity.  We would not say that it is always easy to pronounce, and its resemblance to a Romance language can be overstated:  it does allow quite a lot of consonant clustering.  A further charm is that the baroque rules, however arbitrary they may seem, actually work out as almost inevitable consequences of a fairly small number of design decisions.  We hope that we have given some hint in our discussion of what these design decisions were.

\chapter{The Grammar Proper}

In this section, we will present the grammar, pausing now and then to introduce word classes that are needed;  we will give short lists of commonly used words in the grammar text and full word lists in an appendix.

\section{Simple sentence shapes}

It is an interesting question where to start.  We will begin with the basic Loglan sentence, and work upward to more complicated utterances and other kinds of utterance fragments, and downward to sentence components.

The simplest kind of Loglan sentence is examplified by {\bf La Djan, kamla} (John comes) and {\bf La Djan, donsu le bakso, la Meris} (John gave Mary the box).  This is an S(VO) sentence:  each of these consists of a subject ({\bf la Djan} in both cases), a species of noun phrase, followed by a verb phrase, which is a verb ({\bf kamla}) in the first, {\bf donsu} in the other), followed optionally by a list of ``objects" (noun phrases, none in the first example, two, {\bf la Meris} (Mary) and {\bf le bakso} (the box) in the second.

We pause here to discuss grammatical terminology.  We have used the words ``noun" and ``verb"  though Loglan actually has no such word classes.  It does however have those functional roles, and we will use (hopefully with care) these words to help communicate what is going on in the grammar.  The words {\bf kamla}, {\bf donsu}, and {\bf bakso} all belong to the same word class, Loglan predicates, and can appear in any of the roles.  {\bf Ti bakso} (this is a box) is a sentence in which the very nounish (to the English mind) word {\bf bakso} appears...as the verb!  What we call a ``noun phrase" can also be called an ``argument" (terminology taken from logic and also already used in Loglan grammar).  Loglan grammarians have up until now used ``predicate" indifferently for predicate words and for what we call ``verbs" and ``verb phrases";  we will continue to use ``verb" and ``verb phrase" as grammatical terms.

We discuss the difference between this kind of basic sentence and atomic sentences of predicate logic.  An atomic sentence of predicate logic is of a form $Mx$ ($x$ is a man),
$Bxy$ ($x$ is bluer than $y$), $Gxyz$ ($x$ gives $y$ to $z$).  There is a predicate ($M$, $B$, $G$) and argument lists ($x$, $xy$, $xyz$).  A sentence like $Bxy$ might parse
$(B)(x)(y)$ or perhaps $(B)((x)(y))$;  the predicate and the individual arguments might be components at the same level or the predicate and the argument list might be components (the list further resolving into individual arguments).

The Loglan parse is different, in a way which brings Loglan closer to natural languages:  {\bf da mrenu}:  $xM$, {\bf da blanu de}:  $x((B)(y))$, {\bf da donsu de di}:  $(x)G((y)(z))$.  The weird thing here, from the standpoint of a logician, is the very special role of the subject:  the whole sentence breaks at the top into the noun phrase subject and the verb phrase containing the verb and  a list of the second and subsequent arguments.  The oddities of the way this breaks down become clearly important later when we discuss logically connected predicates.

\subsection{A brief review of components we use in example sentences}

To support examples, we should say something about the components of this sentence and what we are currently putting in for these components.

Any Loglan predicate word can play the role of the ``verb".  There are more complicated verbs than single predicate words, and we will see some possible additional complexities quite soon.  A name such as {\bf la Meris}, {\bf la Djan Braon} is eligible to be a noun phrase (either a subject or an object).  Pronouns such as {\bf ti}, {\bf ta} (this, that), or {\bf da, de, di, do du} (pronouns referring to recently mentioned noun phrases by a scheme we will discuss below), or letter names (referring to recently mentioned noun phrases with the given initial), or noun phrases built from predicates such as {\bf le mrenu}, (the man), are other possible arguments we may use before we have fully explained the range of possibilities for noun phrases.

\subsection{Changes of argument order and omission of arguments}

The sentence ``I am better than you" is expressed {\bf Mi gudbi tu}.  The sentence ``You are better than me" can of course be expressed {\bf Tu gudbi mi}, but it can also be expressed {\bf Mi nu gudbi tu}.  {\bf nu gudbi} is a verb, just as {\bf gudbi} is.  The effect of the particle {\bf nu} is to reverse the first and second arguments.  The particle {\bf fu}
interchanges the first and third arguments;  the particle {\bf ju} interchanges the first and fourth arguments.  Compounds are possible:  {\bf nufunu} interchanges the second and third arguments, for example:  {\bf La Meris, nufunu donsu la Djan, le bakso} (Mary gave John the box).  No one is going to carry out the transformation expressed by {\bf nufunu} in three separate steps in their head during a conversation:  it should be learned as a separate dictionary word.  But if you work it out step by step, you will find that that is what it does.  These contructions implement what would be ``passives" in other languages:  the Loglan term is ``conversion".

A variation implements ``reflexives".  {\bf nuo, fuo, juo} have the effect of eliminating the second, third, fourth argument, respectively, by supplying the subject as that argument.
{\bf La Meris, nuo donsu la Djan} (Mary gave herself to John) or {\bf la Meris, fuo donsu le bakso} (Mary gave herself the box) exemplify this transformation.  Compounds can be formed using the reflexives:  for example {\bf nufuonu} eliminates the third argument by identifying it with the second.

Any Loglan predicate word has a certain number of arguments, which have a certain order in the situation it represents, which can be seen in its dictionary entry.  Without special contrivances, you cannot supply the predicate with {\bf more} arguments.  But you can supply it with fewer arguments.  {\bf Mi gudbi tu} means ``I am better than you".
Just {\bf Mi gudbi} means ``I am good", with the underlying assertion being ``I am better than someone".  Another example {\bf La Meris, donsu le bakso}:  ``Mary gave the box away (gave it to someone)".

This can be combined with changes of argument order:  {\bf Mi nu gudbi}, ``I am bad" (I am worse than someone);  {\bf La Meris, nufunu donsu la Djan}:  ``Mary gave (something) to John.".  The argument omitted is always the last one, but if one changes the order of the arguments, one can put an argument one wishes to omit in the last position.

If the reader wonders why we introduce this transformation of verbs here, they should note two things:  as we will see later, this is one of the most tightly binding operations on verbs, and further, it acts exactly on the very simplest features of the structure of the simple Loglan sentence.

\subsection{Tenses and variations}

In this section we introduce tenses of the Loglan verb, which do not necessarily have anything to do with time.  Tense is achieved using a structure word (either {\bf ga} or a word of the PA class):  the simplest examples
are {\bf na}, {\bf pa}, {\bf fa}. the present, past, and future tenses.  A nontemporal examples is {\bf vi} (here).  There is also a null tense {\bf ga}, which is used in situations where it is grammatically useful to have a tense but we do not actually want to say anything about the time, place or conditions of the assertion.

{\bf la Meris, pa cluva la Djan} Mary loved John

{\bf Mi fa nufunu donsu tu}  I will give you something

{\bf La Ailin, vi danse}  Eileen dances here

{\bf Le mrenu ga sadji}  The man is wise (in general, no commitment to a particular time).  Here the {\bf ga} is grammatically a tense but it doesn't add anything to the semantics.  It is needed, because as we will see later, {\bf le mrenu sadji} is not a sentence, but a noun phrase, ``the wise man".

We aren't introducing tenses at length:  we actually need to introduce them in order to describe a further manipulation of basic sentences.  Notice that
{\bf nufunu donsu} is tensed, rather than {\bf fa donsu} being converted:  the tense is much more loosely attached than the conversion operator.  In fact, the tense
attaches to the verb phrase as a whole rather than to the verb.\footnote{It is actually possible to convert a sort-of-tensed verb but it is tricky:  {\bf Mi nufunu ge donsu je fa gue tu}, in which quite a lot is going on which we will not explain yet!}

\subsection{Variations in sentence order}

We can put the subject in a sentence after the verb in two ways.

  The first kind of sentence we can produce has a tensed verb phrase  with its objects (it might be tensed with {\bf ga} strictly for grammatical purposes) followed optionally by {\bf ga} then the subject:

{\bf Ga gudbi tu ga mi}  I am better than you

{\bf Ga gudbi tu}  There are better than you (here the subject is omitted!)

{\bf Ga donsu le bakso la Djan, ga la Meris}  Mary gave the box to John

{\bf Nia nu gudbi tu ga mi}  You are being better than me (combining conversion of the predicate with reordering of sentence components!)  The tense word {\bf nia} is the present progressive.

The second kind of sentence with subject delay consists of a tensed verb phrase with no objects followed by {\bf ga} then all the arguments in the sentence.

{\bf Ga donsu ga  la Meris, le bakso la Djan.}

This allows us to achieve VOS and VSO word orders.\footnote{It is a reform of ours to require that gasents (the Loglan jargon for subject-delayed sentences) must
have either exactly one argument delayed or all arguments delayed.  We want to avert listeners being forced to retroactively change their understanding of the meanings of arguments appearing earlier as in {\bf $^*$Ga donsu le bakso, ga mi tu}  in which one would reasonably start out thinking that {\bf le bakso} was the object being given (the second argument) but  the presence of two arguments after {\bf ga} (permitted in 1989 Loglan) forces the listener to revise this:  the actual meaning of the sentence would be ``I gave you to the box".}

We can put some objects before the verb, if we separate those objects from the subject with the particle {\bf gio}.

{\bf Mi gio le bakso ga donsu tu}  I give the box to you (the tense {\bf ga} is actually needed to keep from saying {\bf le bakso donsu}.the boxy giver), or

{\bf Mi gio le bakso tu ga donsu} 

This supports SOV(O) sentence order.

The particle {\bf gio} may optionally be used to set the subject apart from the objects in a VSO sentence:

{\bf Ga donsu ga  la Meris, gio  le bakso la Djan.}

There is another device for modifying sentence order by bringing objects to the front, but this device cannot be properly introduced until after we discuss logically connected sentences.

\subsection{Modifiers and tagged arguments}

Tense, location and modal operators (the same words which can decorate verb phrases as tenses) can form {\em sentence modifiers\/} which are rather like additional arguments
which can be supplied with any verb.  In English grammar, these would be ``prepositional phrases".

Relative modifiers and arguments (including the tagged arguments introduced below) are called terms.  The subject and the list of objects in a verb phrase are both
term lists (of slightly different kinds, as we will see).  The subject is an arbitrary list of terms containing at least one argument and no more than one untagged argument.  It is important to notice that if the list of terms before the verb in a sentence does not contain any arguments, the sentence will be either a gasent (if tensed) or an imperative (if not tensed).
The object list can contain no more than four untagged arguments (since there is no predicate taking more than five arguments).

Formally, a relative modifier is either a word of class PA (a tense, location or modal operator) followed by a noun phrase, followed optionally by the particle {\bf guua} or the general right closer particle {\bf gu}, or simply a PA word followed optionally by {\bf gu}.  In the last kind of sentence modifier, you should suppose that the omitted argument
of the PA word is the present situation in which the speaker is delivering their speech (the referent of the pronoun {\bf tio}).

Lists of PA words will appear in the next section.

Examples of such modifiers:

{\bf Vi la Djan, mi bleka le nirda}  Near John I watched the bird

{\bf mi bleka le nirda, vi la Djan}

{\bf Mi godzi na}  I go now (here the ``now" is not a tense but a sentence modifier).

The relationship of a sentence modifier to the sentence is exactly the same no matter where it appears in the sentence.  It modifies the verb phrase, or equivalently, the entire situation represented by the sentence, not an argument it happens to be near.

{\bf Mi bleka le nirda vi la Djan}  means that I was near John when I was watching the bird.  

{\bf Mi bleka le nirda ji vi la Djan}  means that I was watching the bird which was near John (a different grammatical construction, the subordinate clause, which we have not seen yet).

A modifier or modifiers may appear before the {\bf ga} or tense in a subject-delayed sentence.

{\bf Na la Ven, pa kamla ga la Djan}  John will come at nine

Tagged arguments are arguments which are allowed to float free in a sentence in the same way that relative modifiers do.  This can be done with numerical place tags or case tags.

The numerical place tags {\bf zua}, {\bf zue}, {\bf zui}, {\bf zuo}, {\bf zuu} are signs of the first, second, third, fourth and fifth argument of a predicate.  This allows
arguments to be freely reordered and moreover allows medial arguments to be omitted.

{\bf Zui la Djan, donsu zua la Meris}  Mary gave (something) to John.

It also allows arguments to be placed with the subject, as long as at most one argument in the subject is untagged:

{\bf La Meris, zui la Djan, pa donsu le bakso}

It also allows more than one argument to be supplied for the same place.

{\bf Zua la Meris, zua la Djan, cluva la Ailin}  Mary and John love Eileen.

Untagged arguments are taken as usual to represent the places of the argument in order, skipping places corresponding to numerical place tags (or case tags) which have already appeared earlier in the sentence.  We do not require a listener or reader to displace a sequence of arguments when a {\bf zua} is encountered at the end of a sentence.  Numerical place tags have a special effect in term lists appearing before the particle {\bf gi}, which we will describe below.

The case tags are a bizarre idea which {\em we\/} would not have installed in this language.  This is not to say that similar ideas do not occur in natural languages.  With each place of a Loglan predicate, a {\em case\/} is associated in the dictionary, and that case tag may be used to reference that argument of that particular predicate in the same way the appropriate numerical case tag would reference it.  Another possible use of a case tag is to suggest an argument whose place the speaker has forgotten, or perhaps an argument of the predicate which does not appear in the dictionary!

There is a further issue that the dictionary includes words in which distinct arguments have the same case.  To support this, we have recently provided forms which reference the first, second, third, etc. argument of a given case.

The list of case tags will appear in the next section.

We give examples of use of these tags.

{\bf La Djan, dio la Meris, cluva}  John loves Mary.  This is another way to get SOV order, and notice that {\bf gio} is not needed.

{\bf Dio la Meris, (kao) la Djan, cluva}, with the same meaning.  The nominative case tag {\bf kao} is optional:  it can be used, which illustrates the fact that the subject needs to contain at least one argument
and at most one untagged argument (two tagged arguments are all right).

{\bf La Djan, pa donsu dio la Meris, le bakso}  John gave the box to Mary.  The {\bf dio} indicates which argument {\bf la Meris} is (she is not being given as a gift) and {\bf le bakso} falls tidily into the first unused argument place.

{\bf Dio la Meris, beu le bakso, donsu la Djan}  means the same thing as the previous sentence.

\subsection{Imperatives (and observatives)}

An untensed sentence consisting of a verb followed by an object list, with possibly some modifiers before the verb, is an imperative.

{\bf Donsu le bakso la Djan!}  Give the box to John.

{\bf Na la Ven, donsu le bakso la Djan!}  At nine, give the box to John.

If a tensed verb followed by an object list (possibly preceded by modifiers) is given, this is actually a subject-delayed sentence with the subject omitted.  We call this
an {\em observative\/}:  we note it as a special form mostly to indicate that such sentences are not imperatives.

{\bf Na crina}  It is raining (literally, someone is being rained on).  This is a shortening of {\bf Na crina ga ba}

{\bf Na donsu le bakso la Djan}  Someone is giving the box to John.

As an experiment, Loglan has borrowed a concept from Lojban and installed the imperative pronoun {\bf koo}.  This is used just like {\bf tu} (you) with the extra force that the usual
referent of {\bf tu} is commanded to make the statement true.

{\bf Koo donsu le bakso la Djan!}  Give the box to John!

but also

{\bf La Meris, cluva koo!}  Make Mary love you!

{\bf Mi jupni lepo koo gudbi!}  Make me think well of you! (lit.  Make me think you are good).

The imperatives with {\bf koo} are sentences of perfectly general structure and do not belong to the imperative grammatical class which is the subject of this section, usually, though consider

{\bf Cluva koo!}  Love yourself!

\section{Logically connected sentences}

\subsection{Forethought connected sentences}

\chapter{Lexicography Appendix:  full word lists}

\section{Case tags and indirect reference particles}

The case tags, including the positional ones are listed:  

\begin{description}

\item[beu:] (patients/parts), 

\item[cau:] (quantities/amounts/values), 

\item[dio:] (destinations/receivers), 

\item[foa:] (wholes/sets/collectives), 

\item[kao:] (actors/agents/doers), 

\item[jui:] (lessers),

\item[neu:] (conditions/circumstances/fields), 

\item[pou:] (products/purposes), 

\item[goa:] (greaters), 

\item[sau:] (sources/reasons/causes), 

\item[veu:] (effects/states/effects/deeds/means/routes), 

\item[zua:] (first argument), 

\item[zue:] (second argument), 

\item[zui:] (third argument), 

\item[zuo:] (fourth argument), 

\item[zuu:] (fifth argument), 

\item[lae:] (lae X = what is referred to by X), 

\item[lue:] (lue X = something which refers to X)

\end{description}  The operators of indirect reference {\bf lae} and {\bf lue} are a different sort of creature, which originally had the same grammar as case tags, but now have somewhat different behavior.   The latter two operators can be iterated (and so can case tags, probably indicating that more than one applies to the same argument).

For each semantic case tag there are forms like {\bf beuzi, beucine} to reference the first argument with that tag,  {\bf beuza, beucito} to reference the second argument with that tag,
and {\bf beuzu, beucite} to reference the third argument with that tag.  Forms like {\bf beucifo, beucife} are theoretically possible.

\chapter{The Formal Grammar in PEG Notation}

This chapter contains the actual Parsing Expression Grammar (PEG) notation in which the formal grammar is represented:  this is the source from which the computer parser is constructed.
It is also intended to be the basis of the presentation of the grammar.

I'm hoping to do some work on the PEG parser so that this file can be included here in a way which does not run off the margins.

\begin{verbatim}

# In this file I will develop the entire Loglan grammar on top of the phonetic proposal

# Dated updates now to appear here

# I have further fine-tuning of djifoa gluing in mind.
# Allow the -r glue to be expressed as
# -rr after all mandatory monosyllables, removing the annoying pronunciation problem?
# I was thinking of allowing -hy gluing in other contexts, but it is actually a bad idea.

# 9/15/2019 installed semantic case tags with order distinctions for use with predicates with more than one argument of the same case.
# one solution is beucine, beucito...  another is beuzi, beuza, beuzu.


# 4/28/2019  Various debugging of the new predicate algorithm.  Added CVVhy as a glued form for CVV djifoa.
# added capitalization of djifoa glue!  Confirming my apparent earlier decision that a CVV(h)y djifoa must be followed
# by a full predicate complex.

# 4/26/2019:  this incorporates various revisions to the phonetics, correcting errors or clarifying rules,
# motivated by my development of the phonetics section of a new grammar document.  The one notable
# change is that <ci> is now only a name marker if followed by an explicit pause.  This only requires
# changes in writing in serial names.  In speech, it is recommended that one not pause after <ci>
# except before a name word.  The benefit is that non-serial-name related uses of <ci> no longer
# threaten mysterious needs to add explicit pauses before following name words.

# I want to add the <zao> proposal of John Cowan.  Done, 4/15/2019.  the imperative pronoun <koo> has been added though not officially.  I should also add <dao> for the dummy argument, but not today (it is in as of 4/18)

#4/25  Making note of the idea that <ci> should not be a name marker unless followed
# by a pause.  This would require that one pause before ci-marked names and it would
# remove some very confusing corrections for the false name marker problem.  If we
# required the pause to be explicit we would be imposing the expectation that whitespace
# after <ci> is not a pause.  Otherwise we could encourage writing a juncture after <ci>
# to deny presence of a pause, which is reasonable considering the meanings of <ci>.
# I am implementing the version with explicit pauses between <ci> and names
# and the directive not to pause after <ci> without explicit indication.  This solution
# involves rewriting existing text only in the rare instances where <ci> precedes a name.

# 4/25/2019  Corrected some instances of (expanded) badstress.  Now forbidding (C)VVVV initial predicates.  Probably I should use class badstress systematically in defining cmapua.

# 4/24/2019  Final consonants in syllables cannot be followed by syllabic continuants.  
# this rationalizes the definition of SyllableA.

# 4/22 I am thinking of explicitly flagging imperative sentences;  not changing
# the grammar but making this visible in the parse.  This might also have some
# effects on logical connections.  4/23 created an imperative class for atomic
# imperative sentences;  this has no actual effect on parses, just
# organizes them in a more enlightening way.

# 4/17-18 2019:  updates commented out which make sentpred linkable with forethought
# and afterthought connectives (making some uses of <guu> to share arguments 
# unnecessary).  There are subtleties.  Basically, untensed predicates without
# argument lists will be linked by A and KA series connectives.  Such a linked
# set can be tensed as a whole.  Such a linked set will share a following termset.
# This will probably change many parses in the Visit and other legacy sources.
# This required some really subtle adjustments to work right, divinable from
# the actual rules given.  Definitely experimental.

# 3/9/2019 further, extended LIU1 to handle <ainoi> and its kin
# (actual mod is to class Cmapua)  Further, fixing mismatch 
# between connective and A classes.  One does now have to pause
# before <ha> and its compounds.

# 3/9/2019 repaired bugs in negative attitudinals.  A pause
# in a negative attitudinal of the <no, ui> form will not break
# it.  <ainoi> didnt work for two reasons:  the clauses
# in the definition of NOUI were in the wrong order, and
# the connective class mistakenly included <noi> so the 
# phonetics checker was crashing!  I had to move N and NOI 
# earlier to make this work.  Not yet installed in the other 
# version.

# 1/26/2019  added <vie>, JCB's "objective subjunctive" as a PA
# class word.  I should add this to the other file as well.

# 12/22/18:  just a comment:  one does not have to pause before <ha> and its compounds.
# I do not know whether to fix this.  One did not have to in LIP either.  For the moment I will
# leave it as it is.  As a matter of style, one probably should pause.

# 10/6/18  minor adjustments, made only in this file.  Allow <sujo> (a wicked thing to say).  Do not
# allow <futo>:  suffixed conversion operators must be nu + suffix.

# 6/2 fixed LIO + alien text.  I also fixed some other glitches described in the reference grammar.

# 5/11 making version without "alternative parser" features.  This version allows GAA but it doesn't
# do anything:  the definitions of argumentA and kin are the only point of difference.  Master version:
# becomes "alternative" by reinstating alternative definitions of argumentA and kin.  Further, made changes
# recommended in the reference grammar.  ALTERNATIVE -- this is actually my master version.  Edit
# this and revise the argumentA and kin entries to make the original version.

# 4/24 discovered and repaired a bug re ci-marked names suffixed to descriptions.  Discovered a bug in numerical
# descriptions yet to be fixed:  <lio> needs to be an alien text marker, maybe taking double quotes.  The description-
# with-suffixed-name bug was actually quite gruesome.  I think it is repaired.

# 4/23 streamlined definition of descriptn.  Shouldn't change anything.  It was remarkably tricky though;  preserving the old form
# in case of further trouble.

# 4/22 I think this will be the  master grammar file, with alternative lines to turn off the 
# GAA-related features.

# 4/22 allowing general predicates in gasent1.  This removes an extreme oddity in parsing of imperatives.
# I do not see any new dangers from this.

# 4/22 I changed the final element of a keksent to be a sentence (new class uttA0), not a general sentence fragment.
# several parse errors in the Visit were uncovered by this.

# 4/22:  note that I still have the obligation to restore the <zao> construction.

# 4/9/2018 the large subject marker GAA can also be used to defend the beginnings of gasents and imperatives
# from absorbing trailing arguments into an unintended statement.  In this context <gaa> may be followed by <ga> ;-)

# 4/8/2018 this is an alternative version in which an argument which starts an SVO sentence will not be accepted
# as a trailing argument of a previous sentence.  This allows neat termination of <lepo> clauses preceding
# a subject, for example.  Unlike the previous alternative approach, this seems to involve a single fairly
# tidy change:  it is all an issue of avoiding needs for explicit closure.  Further refinement:  SVO sentences
# can be marked with GAA (which is not a tense:  it appears optionally just before the predicate, or just
# before sutori arguments marked with GIO if there are any), the "large subject marker":  an argument which 
# starts an SVO sentence *not marked with GAA* will not be accepted as a trailing argument of a previous
# sentence.  This is a sufficiently complex grammar change that it requires thought:  it is not conservative
# in my usual sense.  The fact that GAA carries a mandatory stress is virtuous.  Its resemblance to the 
# particle GA when used as a tense is not a bad thing:  it would often be used instead of GA to close
# a <lepo> clause appearing as a subject, and it is perhaps better for that purpose.  Note that GAA can
# and often will be followed by a tense.  This grammar change depends strongly on the previous ruling that the O in
# SOV(O) sentences must be marked with <gio>:  S gio O^n V (O^m).

# nuu is an atomic A core and there is no nu-affix to A connectives and their kin

# 1/20/2018 redefined CA cores to include a possible NU prefix. This allows more logically connected tenses, for example.

# 1/13/2018 reorganized the internals of class PA in a way which should allow more things and not forbid anything legal now.
# this is pursuant on an analysis of the classes NI and PA as phrases, rather than words, as I start writing a global lexicography
# proposal document.  Enforced explicit pauses after PA phrases appearing as arguments with a following modifier with an argument.

# 12/30/2017 fixed a problem with name markers in the clas NameWord and made a slight change to the new option in NI (names
# as dimensions).

# 12/27/2017 installing an alternative treatment of acronyms under which they are simply names (suffix -n to acronyms in all uses).
# supporting this requires no change at all to acronymic name usage (just use the -n versions with the usual rules for names),
# and for dimension usage requires <mue> to be a name marker and support for <mue> PreName as an alternative suffix to NI.

# 12/27/2017  Frivolously fooling with the capitalization conventions.  They ought to work better now...but I could have broken something.
# the main new idea was to require that a capitalized embedded letteral actually be followed by lowercase if it was preceded by lowercase
# (with the obvious exception for a letteral followed by a letteral).  Also changed the rules for diphthongs in cmapua to make all-caps
# legal for cmapua.  The general idea is that one can start with a capital letter and stay capitalized until one hits a lower case letter,
# at which point one can jump back up to caps only at a juncture (after which you can remain capitalized) or temporarily for a vowel
# after z- (after which lower case resumes) or an embedded literal (after which lowercase resumes).  The total effect is that this allows
# attested capitalization patterns in Loglan (including capitalization of embedded literals as in possessive articles and acronyms)
# and also allows all-caps for individual words (attested in Leith but suppressed in my version) and supports capitalization of components
# of names as in <la Beibi-Djein> (by artful use of syllable breaks:  Leith just has BeibiDjein, which does not work for me).

# 12/26/2017  Installed <niu> (quotation of phonetically legal but so far non-Loglan words).  I did not make <niu> a name marker, so if one were to
# use it with names (where it isn't really appropriate), one would have to pause initially:  <niu, Djan>.

# I note in this connection that quotation of names with li...lu remains limited, since names by themselves are not
# utterances:  one needs the <la>.  I fixed this as an exception in the previous parser;  I may do it here or I may
# not, haven't decided.  Single name words can be quoted with <liu>, of course, but not serial names.

# 12/24/2017  Refined treatment of vowel pairs for Cvv-V cmapua units.  First 12/24 version rather disastrously
# broken:  this should be fixed!

# 12/23/2017  This is now completely commented, with minor local exceptions to which I will return later.
# This document is the basis on which I will build all subsequent parsers, with due modifications to the comments.
# The Python PEG engine and preamble files contain commands for constructinging a Python parser from it directly.

# 12/22/2017 major progress on commenting the grammar

# yet later 12/20:  no change in performance of the grammar, extensive commenting in the
# grammar section.  Considerable changes in arrangement:  for example, vocatives, inverse vocatives,
# and free modifiers are moved to a much earlier point.  I'm hoping to get a genuinely almost readable
# commented grammar...

# later 12/20  starting the process of commenting and editing the grammar, starting
# at basic sentence structures.  Notably rewrote the class [keksent] more compactly,
# one hopes with no actual effect on parses.

# 12/20/2017  Do not require expression of pause after finally stressed cmapua before
# vowel initial predicate as a comma, since the initial vowel signals the pause anyway.  
# Allow final stress in names.  Fixed bug in CVVHiddenStress.  Prevented
# broken monosyllables in finally stressed CVV djifoa.  refinement of caprule

# 12/19/2017 seem to have had a versioning failure and lost the fix which requires
# CVVy djifoa to be followed by complete complexes.  Restored.

# 12/18/2017 fixed a bug in treatment of stressed syllables in recognizing predicate starts.  Also
# narrowed the generalized VCCV rule to allow more of the quite unlikely space of predicates with lots
# of vowels before the CC pair.  Probably they should be banned (and none have ever been proposed with
# more than three) but that rule is not the context in which to arbitrarily ban half of them.  Some cleanup
# of the display of parses, for which updated version of logicpreamble.py should also be uploaded. A refinement
# to class "connective" checking that apparent logical connectives are not initial segments of predicates.
# This has the effect of delaying the declaration of "connective" until after the declaration of
# "predstart".

# 12/17/2017 further refinement of the 12/16 version:  a couple of bugs spotted.

# 12/16/2017 There should be no change in parsing behavior, but the predstart ruleset is shorter
# and more intelligible, and I realized that Complex doesnt need a check for the anti-slinkui test
# (the requirement that certain initial CVC cmapua be y hypenated which replaces the slinkui test))
# at all:  the way predstart works already ensures that initial CV cmapua fall off in the excluded
# cases, the idea being that we test the front of a predicate without lookahead in all cases.  Also
# addressed the subtle point that one wasn't forced to pause after a predicate before following y
# (not likely to arise as a problem).


# 12/14/2017  Corrected vowel grouping to avoid paradoxical vowel triples which are default
# grouped in a way which becomes illegal if made explicit.  SyllableA really should contain a final
# consonant:  the previous form was messing up vowel grouping.  Serious bug where end of djifoa
# and syllable resolution of a predicate may fail to agree.  I think I blocked this by ensuring that
# final djifoa are not followed by vowels.  Other fine tuning of the complex algorithm.  Also had
# to repair the check for CVCCCV and CVCCVV predicates.


# 12/13/2017:  added kie ( utterance ) kiu to class LiQuote.  Did fine tuning to ensure
# that cmapua streams stop before <li> or <kie>, that names can stop at double quotes or close
# parentheses, and that the capitalization rule ignores opening parentheses as well as double
# quotes.  One can now adorn li lu with quotes (on the inside) in a reasonable way
# and adorn kie kiu with parentheses (on the inside) in a reasonable way.  One cannot
# *replace* these words (or any words) with punctuation in my model of Loglan.  Also,
# updates to comments, and # (end of utterance) added as a marker of terminal punctuation.

# END of dated updates

# This is now done, in a first pass.  That is, the grammar is adapted and appears to work, more or less.
# What is needed is comments on the lexicography and the grammar...Phonetics has now pretty clearly been sorted
# from the grammar (there are some places where the phonetics accept grammar information with regard to punctuation).

# Alien text is now handled somewhat differently.  Some issues to do with quoting names are not finalized and have not been tested.

# I added -iy and -uy as VV forms allowed in general in cmapua but not in other words;  they are always monosyllabic.  What this
# immediately allows me to do is to give Y a name which is not phonetically irregular!  <ziy> is supported:  <yfi> is too, now.

# capitalization is roughly back to where it was in the original, but all-caps are allowed.

# acronyms are liable to be horrible.

# Fixed the recursion problem in a way which will not be visible in ordinary parses.  Streams of cmapua will always
# be broken at name or alien text markers (instead of using lookahead to check that we do not stand at the beginning
# of a name word  or alien text word).  The next cycle will then check for a name or alien text, and also check for
# badnamemarkers;  no lookahead is happening while a stream of cmapua is being read except checking for
# the markers of names and alien text.  This will change the way phonetic parses look (streams of cmapua will
# break (and sometimes resume) at name markers or alien text markers, but it will not change any grammatical
# parses.

#Part I Phonetics

# Mod bugs, I have implemented all of Loglan phonetics as described in my proposal.  Borrowing djifoa are pretty tricky.

# I have now parsed all the words in the dictionary, and all single words of appropriate classes parse successfully.
# I have added alien text and quotation constructions which do not conform to these rules; so actually
# all Loglan text should parse,  mod some punctuation and capitalization issues.  The conventions for
# alien text here are not the same as those in the current provisional parser.

# I believe the conventions for forcing comma pauses before vowel initial cmapua and after names
# except in special contexts have been enforced.  In a full grammar, one probably would want
# to disable pauses before vowel initial letterals (done).  This grammar also does not support the lingering
# irregularities in acronyms (and won't).

# This grammar (in Part I) is entirely about phonetics:  all it does is parse text into names (with associated initial
# pauses or name markers), cmapua (qua unanalyzed streams of cmapua units), 
# borrowings and complexes, along with interspersed comma pauses and marks
# of terminal punctuation.  It does support conventions about where commas are required
# and a simple capitalization rule.  Streams of cmapua break when markers initial
# in other forms are encountered (and may in some cases resume when the markers
# are a deception).

# a likely locus for odd bugs is the group of predstartX rules which detect apparent cmapua which
# are actually preambles to predicates.  These are tricky! (and I did indeed find some lingering
# problems when I parsed the dictionary).  Another reason to watch this rule predstart
# is that it carries a lot of weight:  !predstart is used as a lightweight test
# that what follows is a cmapua (a point discussed in more detail later).

# In reviewing this, I think that very little is different from 1990's Loglan (the borrowing djifoa
# are post-1989 L1, but not my creation).  Some things add precision without making anything in 1990's Loglan incorrect.
# The requirement that syllabic consonants be doubled is new, and makes some 1990's Loglan names incorrect.
# The requirement that names resolve into syllables is new, and makes some 1990's Loglan names incorrect,
#  usually because they end in three consonants.
# The rule restricting final consonant pairs from being noncontinuant/continuant is new, but
#   does not affect any actual predicate ever proposed.
# Enhancing the VccV rule to also forbid CVVV...ccV caused one predicate to be changed
#  (<haiukre> became <haiukrre>, and haiukre was a novelty anyway, using a new name for X in X-ray)
# The exact definition of syllables and use of syllable breaks and stress marks is new (the close comma
#  was replaced with the hyphen, so Lo,is becomes Lo-is); but this does not make anything in 1990's Loglan
#  incorrect, it merely increases precision and makes phonetic transcript possible.
# Forbidding doubled vowels in borrowings was new, was already approved, and caused us to change
#  <alkooli> to <alkoholi>.
# Formally allowing the CVccVV and CVcccV predicates without y-hyphens took a proposal in 2013 because
#  Appendix H was careless in describing their abandonment of the slinkui test, but the dictionary
#  makes it evident that this was their intent all along.  The slinkui test had already been
#  abandoned in the 1990s.
# Formally abandoning qwx was already something that the dictionary workers in the 1990's were working
#  on; we completed it.
# Allowing glottal stop in vowel pairs and forbidding it as an allophone of pause is a new phonetic
#  feature in the proposal but not reflected in the parser, of course.   Alternative pronunciations of
#  y and h and allowing h in final position are invisible or do not make any 1990's Loglan incorrect.
# Permitting false name markers in names was already afoot in the 1990's and the basic outlines of our
#  approach were already in place.  The rule requiring explicit pauses between a name marker not starting
#  a name word and the beginning of the next name word is new, but reflects something which was already
#  a fact about 1990's Loglan pronunciation:  those pauses had to be made in speech
# (and in the 1990's they had no tools to do relevant computer tests)!  The requirement
#  that names resolve into syllables restricts which literal occurrences of name markers are actually
#  false name markers (the tail they induce in the name must itself resolve into syllables).
# Working out the full details of borrowing djifoa was interesting:  I'm not sure that I've done anything
#  *new* there;  explicitly noting the stress shift in borrowing djifoa might be viewed as something
#  new but it is a logical consequence of JCB's permission to pause after a borrowing djifoa, which contains
#  explicit language about how it is to be stressed, and the
#  final definition of a borrowing djifoa as simply a borrowing followed by -y.  The shift strikes
#  me as a really good idea anyway, because it marks djifoa with a pause after it as phonetically different
#  in an additional way other than ending with the very indistinct vowel y.  My rules as given here do not
#  directly enforce the rule that a borrowing djifoa must be preceded by y but I think they indirectly
#  enforce it in all or almost all cases:  the parser tries to read a borrowing djifoa before reading
#  any other kind of djifoa, so it is hard to see how to deploy a short djifoa in such a way that it would
#  fall off the head of a borrowing without using y.
# These phonetics do not support certain irregularities in acronyms.  We note that
# it is now allowed to insert <, mue> into an acronym, which would be necessary for example
# between a Ceo letteral and a following VCV letteral.

#Sounds

#all vowels

V1 <- [aeiouyAEIOUY]

#regular vowels

V2 <- [aeiouAEIOU]

#consonants

C1 <- [bcdfghjklmnprstvzBCDFGHJKLMNPRSTVZ]

# letters

letter <- (![qwxQWX] [a-zA-Z])

# a capitalization convention which allows what our current one allows and also allows all-caps.
# if case goes down from upper case to lower case, it can only go back up in certain cases.  This
# does allow capitalization of initial segments of words.  There is a forward reference to the grammar
# in that free capitalization of embedded literals is permitted, and capitalization of vowels
# guarded with z in literals as in DaiNaizA.

lowercase <- (![qwx] [a-z])

uppercase <- (![QWX] [A-Z])

caprule <- [\"(]? &([z] V1 (!uppercase/&TAI0)/lowercase TAI0 (!uppercase/&TAI0)/!(lowercase uppercase).) letter (&([z] V1 (!uppercase/&TAI0)/lowercase TAI0 (!uppercase/&TAI0)/!(lowercase uppercase).) (letter/juncture))* !(letter/juncture)

# syllable markers:  the hyphen is always medial so must be followed by a letter.
# the stress marks can be syllable final and word final.  A juncture is never followed
# by another juncture.

juncture <- (([-] &letter)/[\'*]) !juncture

stress <- ['*] !juncture

# terminal punctuation

terminal <- ([.:?!;#])

# characters which can occur in words

character <- (letter/juncture)

# to really get all Loglan text, we should add the alien text constructions and the markers of alien text,
# <lie>, <lao>, <sao>, <sue> and certain quotations which violate the phonetic rules.

# we adopt the convention that all alien text may be but does not have to be enclosed in quotes.
# it needs to be understood that in quoted alien text, whitespace is understood as <, y,>; in the unquoted
# version this is shown explicitly.  This handling of alien text is taken from the final 1990's treatment
# of Linnaeans = foreign names, and extended by us to replace the impossible treatment of strong
# quotation in 1989 Loglan.

# this is a little different from what is allowed in the previous provisional parser, but similar.
# A difference is that all the alien text markers are allowed to be followed by the same sorts of alien text.

# the forms with <hoi> and <hue> are required to have following quotes in written form to avoid
# unintended parses, which otherwise become likely in case of typos in non-alien text cases.

AlienText <- ([,]? [ ]+ [\"] (![\"].)+ [\"]/ [,]? [ ]+ (![, ]!terminal .)+ ([,]? [ ]+ [y] [,]? [ ]+ (![, ]!terminal .)+)*)

AlienWord <- &caprule ([Hh] [Oo] [Ii] juncture? &([,]? [ ]+ [\"])/[Hh][Uu] juncture? [Ee] juncture? &([,]? [ ]+ [\"]) / [Ll] [Ii] juncture? [Ee]juncture? /[Ll] [Aa] [Oo]juncture? /[Ll] [Ii] juncture? [Oo] juncture? /[Ss] [Aa] [Oo]juncture?/[Ss] [Uu] juncture? [Ee]juncture?) AlienText

# while reading streams of cmapua, the parser will watch for the markers of alien text.

alienmarker <- ([Hh] [Oo] [Ii] juncture? &([,]? [ ]+ [\"])/[Hh][Uu] juncture? [Ee] juncture? &([,]? [ ]+ [\"]) / [Ll] [Ii] juncture? [Ee] juncture? /[Ll] [Aa] [Oo] juncture? /[Ll] [Ii] juncture? [Oo] juncture?  /[Ss] [Aa] [Oo] juncture?/[Ss] [Uu] juncture? [Ee] juncture?) !V1

# 5/11/18 added <lio> as an alien text marker, to support numerals.

# the continuant consonants and the syllabic pairs they can form

continuant <- [mnlrMNLR]

syllabic <- (([mM] [mM] !(juncture? [mM]))/([nN] [nN] !(juncture? [nN]))/([rR] [rR] !(juncture? [rR]))/([lL] [lL] !(juncture? [lL])))

# the obligatory monosyllables, and these syllables when broken by a usually bad syllable juncture.
# The i-final forms are not obligatory mono when followed by another i.

MustMono <- (([aeoAEO] [iI] ![iI]) /([aA] [oO]))

BrokenMono <- (([aeoAEO] juncture [iI] ![iI])/([aA] juncture [oO]))

# the obligatory and optional monosyllables.  Sequences of three of the same letter
# are averted.  Avoid formation of doubled i or u after ui or ui.

Mono <- (MustMono/([iI] !([uU] [uU]) V2)/([uU] !([iI] [iI]) V2))

# vowel pairs of the form found in cmapua and djifoa.
# (other than the special IY, UY covered in the cmapua rules)

# The mysterious prohibition controls a permitted phonetic exception in djifoa gluing.
# compua are never followed directly by vocalic continuants in any case.

VV <- !(!MustMono V2 juncture? V2 juncture? [Rr] [Rr]) (!BrokenMono V2 juncture? V2)

# the next vocalic unit to be chosen from a stream of vowels
# in a predicate or name.  This is different than in our Sources
# and formally described in the proposal.

NextVowels <- (MustMono/(V2 &MustMono)/Mono/!([Ii] juncture [Ii] V1) !([Uu] juncture [Uu] V1) V2)

# 5/11/18 forbidding consonantal vowels to follow the same vowel.

# the doubled vowels that trigger the rule that one of them must be stressed

DoubleVowel <- (([aA] juncture? [aA])/([eE] juncture? [eE])/([oO] juncture? [oO])/([iI] juncture [iI])/([uU] juncture [uU])/[iI] [Ii] &[iI]/[Uu] [uU] &[uU])

# the mandatory "vowel" component of a syllable

Vocalic <- (NextVowels/syllabic/[Yy])

# the permissible initial pairs of consonants, and the same pairs possibly
# broken by syllable junctures.

Initial <- (([Bb] [Ll])/([Bb] [Rr])/([Cc] [Kk])/([Cc] [Ll])/([Cc] [Mm])/([Cc] [Nn])/([Cc] [Pp])/([Cc] [Rr])/([Cc] [Tt])/([Dd] [Jj])/([Dd] [Rr])/([Dd] [Zz])/([Ff] [Ll])/([Ff] [Rr])/([Gg] [Ll])/([Gg] [Rr])/([Jj] [Mm])/([Kk] [Ll])/([Kk] [Rr])/([Mm] [Rr])/([Pp] [Ll])/([Pp] [Rr])/([Ss] [Kk])/([Ss] [Ll])/([Ss] [Mm]) /[Ss] [Nn]/([Ss] [Pp])/([Ss] [Rr])/([Ss] [Tt])/([Ss] [Vv])/([Tt] [Cc])/([Tt] [Rr])/([Tt] [Ss])/([Vv] [Ll])/([Vv] [Rr])/([Zz] [Bb])/([Zz] [Ll])/([Zz] [Vv]))

MaybeInitial <- (([Bb] juncture? [Ll])/([Bb]juncture?  [Rr])/([Cc]juncture?  [Kk])/([Cc] juncture? [Ll])/([Cc]juncture?  [Mm])/([Cc]juncture?  [Nn])/([Cc]juncture?  [Pp])/([Cc]juncture?  [Rr])/([Cc]juncture?  [Tt])/([Dd]juncture?  [Jj])/([Dd]juncture?  [Rr])/([Dd]juncture?  [Zz])/([Ff]juncture?  [Ll])/([Ff]juncture?  [Rr])/([Gg]juncture?  [Ll])/([Gg]juncture?  [Rr])/([Jj]juncture?  [Mm])/([Kk]juncture?  [Ll])/([Kk] juncture? [Rr])/([Mm]juncture?  [Rr])/([Pp]juncture?  [Ll])/([Pp]juncture?  [Rr])/([Ss]juncture?  [Kk])/([Ss]juncture?  [Ll])/([Ss] juncture? [Mm]) /[Ss] juncture? [Nn]/([Ss]juncture?  [Pp])/([Ss]juncture?  [Rr])/([Ss]juncture?  [Tt])/([Ss]juncture?  [Vv])/([Tt]juncture?  [Cc])/([Tt]juncture?  [Rr])/([Tt] juncture? [Ss])/([Vv]juncture?  [Ll])/([Vv]juncture?  [Rr])/([Zz]juncture?  [Bb])/([Zz] juncture? [Ll])/([Zz] juncture? [Vv]))

# the permissible initial consonant groups in a syllable.  Adjacent consonants should be initial pairs.
# The group should not overlap a syllabic pair.  Such a group is of course followed by a vocalic unit.

# this rule for initial consonant groups is stated in NB3.

# I forbid a three-consonant initial group to be followed by a syllabic pair.  This seems obvious.

InitialConsonants <- ((!syllabic C1 &Vocalic)/(!(C1 syllabic) Initial &Vocalic)/(&Initial C1 !(C1 syllabic) Initial !syllabic &Vocalic))

# the forbidden medial pairs and triples.  These are forbidden regardless of placement
# of syllable breaks.

# each of these is actually a single consonant followed by an initial, and the idea was to identify CVC-CCV junctions which
# would be hard to pronounce.  But the placement of the syllable break is not relevant to the exclusion of the sequence.
# Notice that the continuant syllabic pairs are excluded:  this prevents final consonants from being included in such pairs.

NoMedial2 <- (([Bb] juncture? [Bb])/([Cc] juncture? [Cc])/([Dd] juncture? [Dd])/([Ff] juncture? [Ff])/([Gg] juncture? [Gg])/([Hh] juncture? C1)/([Jj] juncture? [Jj])/([Kk] juncture? [Kk])/([Ll] juncture? [Ll])/([Mm] juncture? [Mm])/([Nn] juncture? [Nn])/([Pp] juncture? [Pp])/([Rr] juncture? [Rr])/([Ss] juncture? [Ss])/([Tt] juncture? [Tt])/([Vv] juncture? [Vv])/([Zz] juncture? [Zz])/([CJSZcjsz] juncture? [CJSZcjsz])/([Ff] juncture? [Vv])/([Kk] juncture? [Gg])/([Pp] juncture? [Bb])/([Tt] juncture? [Dd])/([FKPTfkpt] juncture? [JZjz])/([Bb] juncture? [Jj])/([Ss] juncture? [Bb]))

NoMedial3 <- (([Cc] juncture? [Dd] juncture? [Zz])/([Cc] juncture? [Vv] juncture? [Ll])/([Nn] juncture? [Dd] juncture? [Jj])/([Nn] juncture? [Dd] juncture? [Zz])/([Dd] juncture? [Cc] juncture? [Mm])/([Dd] juncture? [Cc] juncture? [Tt])/([Dd] juncture? [Tt] juncture? [Ss])/([Pp] juncture? [Dd] juncture? [Zz])/([Gg] juncture? [Tt] juncture? [Ss])/([Gg] juncture? [Zz] juncture? [Bb])/([Ss] juncture? [Vv] juncture? [Ll])/([Jj] juncture? [Dd] juncture? [Jj])/([Jj] juncture? [Tt] juncture? [Cc])/([Jj] juncture? [Tt] juncture? [Ss])/([Jj] juncture? [Vv] juncture? [Rr])/([Tt] juncture? [Vv] juncture? [Ll])/([Kk] juncture? [Dd] juncture? [Zz])/([Vv] juncture? [Tt] juncture? [Ss])/([Mm] juncture? [Zz] juncture? [Bb]))

# The syllable.  

# there are no formal rules about syllables as such in our Sources, which is odd since
# the definition of predicates depends on the placement of stresses on syllables.

# The first rule enforces the special point needed in complexes that
# a CVC syllable is preferred to a CV syllable where possible; we economically apply
# the same rule for default placement of syllable breaks everywhere, which is, with
# that exception, that the break comes as soon as possible.

# the SyllableB approach is taken if the following syllable would otherwise start with a syllabic pair.

# the reason for this approach is that if one syllabizes a well formed complex in this way...
# the syllable breaks magically fall on the djifoa boundaries.  This does mean that the
# default break in <cabro> is <cab-ro>, which feels funny but is harmless.  Explicitly breaking
# it <ca-bro> will also parse correctly.

SyllableA <- (C1 V2 FinalConsonant (!Syllable FinalConsonant)?)

SyllableB <- (InitialConsonants? Vocalic (!Syllable FinalConsonant)? (!Syllable FinalConsonant)?)

Syllable <- ((SyllableA/SyllableB) juncture?)

# The final consonant in a syllable.  There may be one or two final consonants.  A pair of final
# consonants may not be a non-continuant followed by a continuant.  A final consonant may not
# start a forbidden medial pair or triple.

# The rule that a final consonant pair may not be a non-continuant followed by a continuant
# is natural and obvious but not in our Sources.  Such a pair of consonants would seem to 
# naturally form another syllable.

FinalConsonant <- !syllabic (!(!continuant C1 !Syllable continuant) !NoMedial2 !NoMedial3 C1 !(juncture? (V2/syllabic)))

# Here are various flavors of syllable we may need.

# this is a portmanteau definition of a bad syllable (the sort not allowed in a borrowing).

SyllableD <- &(InitialConsonants? ([Yy]/DoubleVowel/BrokenMono/&Mono V2 DoubleVowel/!MustMono &Mono V2 BrokenMono)) Syllable

# this (below) is the kind of syllable which can exist in a borrowed predicate:
# it cannot start with a continuant pair, it cannot have a y as vocalic unit,
# and its vocalic unit (whether it has one or two regular vowels) 
# cannot be involved in a double vowel or an explicitly broken 
# mandatory monosyllable.

BorrowingSyllable <- !syllabic (!SyllableD) Syllable

# this is the final syllable of a predicate.  It cannot be followed
# without pause by a regular vowel.

VowelFinal <- InitialConsonants? Vocalic juncture? !V2

# syllables with syllabic consonant vocalic units
# this class is only used in borrowings, and we *could* reasonably
# require it to be followed by a vowel.  But I won't for now.
# for gluing this restriction would work, but we might literally borrow predicates
# with syllabic continuant pronunciations.

SyllableC <- (&(InitialConsonants? syllabic) Syllable)

# syllables with y

SyllableY <- (&(InitialConsonants? [Yy]) Syllable)

# an explicitly stressed syllable.

StressedSyllable <- ((SyllableA/SyllableB) [\'*])

# a final syllable in a word, ending in a consonant.

NameEndSyllable <- (InitialConsonants? (syllabic/Vocalic &FinalConsonant) FinalConsonant? FinalConsonant? stress? !letter)

# the pause classes actually hang on the letter before the pause.

# whitespace which might or might not be a pause.

maybepause <- (V1 [\'*]? [ ]+ C1)

# explicit pauses:  these are whitespace before a vowel or after a consonant, or comma marked pauses.

pause <- ((C1 [\'*]? [ ]+ &letter)/(letter [\'*]? [ ]+ &V1)/(letter [\'*]? [,] [ ]+ &letter))

# these are final syllables in words followed by whitespace which might not be a pause.
# the definition actually doesnt mention the maybepause class.

MaybePauseSyllable <- InitialConsonants? Vocalic ['*]? &([ ]+ &C1)

# The full analysis of names.

# a name word (without initial marking) is resolvable into syllables and ends with a consonant.

PreName <- ((Syllable &Syllable)* NameEndSyllable)

# this is a busted name word with whitespace in it -- but not whitespace at which one has to pause.

BadPreName <- (MaybePauseSyllable [ ]+/Syllable &Syllable)* NameEndSyllable

# This is a name marker followed by a consonant initial name word without pause.

# I deployed a minimal set of name marker words; I can add the others whenever.
# I have decided (see below) to retain the social lubrication words as vocative markers
# *without* making them name markers, so one must pause <Loi, Djan>.  By not allowing
# freemods right after vocative markers in the vocative rule, I make <Loi hoi Djan> work as well,
# without pause.

# MarkedName <- &caprule ((([Ll] !pause [Aa] juncture?)/ ([Hh] [Oo] !pause [Ii] juncture?) /  ([Hh] [Uu] juncture? !pause [Ee] juncture?) / ([Cc] !pause [Ii] juncture?)/([Ll] [Ii] juncture? !pause [Uu] juncture?)/[Gg][Aa] !pause [Oo] juncture?/[Mm][Uu] juncture? !pause [Ee] juncture?) [ ]* &C1 &caprule PreName)

MarkedName <- &caprule ((([Ll] !pause [Aa] juncture?)/ ([Hh] [Oo] !pause [Ii] juncture?) /  ([Hh] [Uu] juncture? !pause [Ee] juncture?) /([Ll] [Ii] juncture? !pause [Uu] juncture?)/[Gg][Aa] !pause [Oo] juncture?/[Mm][Uu] juncture? !pause [Ee] juncture?) [ ]* &C1 &caprule PreName)


# This is an unmarked name word with a false name marker in it.

FalseMarked <- (&PreName (!MarkedName character)* MarkedName)

# This is the full definition of name words.  These are either marked consonant initial names without pause defined above,
# names without false name markers beginning with explicit pauses (either comma marked or vowel-initial) 
# and name markers followed, with or without pause, by name words.  In the latter case there must be at least
# whitespace before a vowel initial name.

# a series of names without false name markers and names marked with ci, separated by spaces, may be appended.

# there is a look ahead at the grammar: a NameWord can be followed without explicit pause (there is whitespace and 
# a pause in speech!) by another
# kind of utterance only in a serial name when what follows is of the form <ci> predunit, to be included
# in the name.

NameWord <- (&caprule MarkedName/([,] [ ]+ !FalseMarked &caprule PreName)/(&V1 !FalseMarked &caprule PreName)/&caprule ((([Ll] [Aa] juncture?)/([Hh] [Oo] [Ii] juncture?)/([Cc] &pause [Ii] juncture?)/([Ll] [Ii] juncture? [Uu] juncture?)/[Mm] [Uu] juncture? [Ee] juncture?/[Gg] [Aa] [Oo] juncture?) !V1 [,]? [ ]* &caprule PreName))([,]?[ ]+ !FalseMarked &caprule PreName/[,]?[ ]+ &([Cc] &pause [Ii]) NameWord)* &([ ]* [Cc] [Ii] predunit/&([,] [ ]+/terminal/[\")]/!.)./!.)

# this is the minimal set of name marker words we are using.  We may add more.

# I am contemplating adding the words of social lubrication as name markers, but in a more restricted
# way that in the last provisional parser, in which I made them full-fledged vocative markers.  [Actually,
# I preserved their status as vocative markers without restoring their status as name markers, in the latest version].

# adding <mue> as a name marker

namemarker <- ([Ll] [Aa] juncture?/[Hh][Oo][Ii] juncture?/([Hh] [Uu] juncture? [Ee] juncture?)/[Cc] &pause [Ii] juncture?/[Ll][Ii] juncture? [Uu] juncture?/[Gg][Aa][Oo] juncture?/[Mm] [Uu] juncture? [Ee] juncture?) !V1

# this is the bad name marker phenomenon that needs to be excluded.  This captures the idea
# that what follows the name could be pronounced without pause as a name word according to the
# orthography, but the fact that whitespace is present shows that this is not the intention.

# it is worth noting that name markers at heads of name words pass this test
# (because I omitted the test that what follows is not a PreName in the interests
# of minimizing lookahead);
# but this test is only applied to strings that have already been determined not to
# be of class NameWord.

badnamemarker <- namemarker !V1 [, ]? [ ]* BadPreName

# we test for the bad name marker condition at the beginning of each stream of cmapua,
# and streams of cmapua stop before name markers (and may resume at a name marker
# if neither a NameWord nor the bad marker condition is found).

# We have at any rate completely solved the phonetic problem of names and their markers.

# predicate start tests:  the idea is the same as class "connective" above, to recognize
# the start of a predicate without recursive appeals to the whole nasty definition of predicate.
# The reason to do it is to recognize when CV^n followed by CC cannot be a cmapua unit.

# New implementation 4/28/2019.  This allows only (C)V(V)(V) before the pair of vowels, for much less
# potential lookahead.

Vthree <- (V2 juncture?) (V2 juncture?) (V2 juncture?)

Vfour <- (V2 juncture?) (V2 juncture?) (V2 juncture?) (V2 juncture?)

# predicate starting with two or three consonants:  rules out CC(C)V(V) forms.  Junctures in
# the initial consonant group ignored.

predstartA1 <- (&MaybeInitial C1 juncture? MaybeInitial/MaybeInitial) &V2 !(V2 stress !Mono V2) !(V2 juncture? V2 !character) !(V2 juncture? !character)

# an apparent cmapua unit followed by a consonant group which cannot start a predicate -- CV(V) case

predstartA2 <- C1 V2 juncture? (V2 juncture?)? !predstartA1 C1 juncture? C1

# a stressed CV^n before a consonant group (CV(V) case)

predstartA3 <- C1 !Vthree (!StressedSyllable V2 juncture?)? &StressedSyllable V2 V2? juncture? C1 juncture? C1

# other (C)V^n followed by nonpredicate 

predstartA4 <- C1? V2 juncture? (V2 juncture?)?  (V2 juncture?)? !predstartA1 !(MaybeInitial V2) C1 juncture? C1

# other stressed (C)V^n followed by consonant group

predstartA5 <- C1? !Vfour (!StressedSyllable V2 juncture?)? (!StressedSyllable V2 juncture?)? &StressedSyllable V2 V2? juncture? !(MaybeInitial V2) C1 juncture? C1

# forms with y; implemented CVVhy alternative for CVV cmapua

predstartA6 <- C1 (V2 juncture?) (V2 juncture? [Hh]?/C1 juncture? (C1 juncture?)?) [Yy]

predstart <- predstartA1/predstartA2/predstartA3/predstartA4/predstartA5/predstartA6

# it is worth noting that in the sequel we have systematically replaced tests &Cmapua
# with !predstart.  The former involves lots of lookahead and was causing recursion crashes
# in Python.  The phonetics and the grammar are both structured so that any string
# starting with a name marker is tested for NameWord-hood before it is tested for 
# cmapua-hood; the only thing it is tested for later is predicate-hood, and predstart
# is a rough and ready test that something might be a predicate (and at any rate
# cannot be a cmapua).

# this class requires pauses before it, after all the phonetic word classes.
# what is being recognized is the beginning of a logical connective.

# To avoid horrible recursion problems, giving this a concrete phonetic definition
# without much lookahead.  This can go right up in the phonetics section if it works
# (and here it is!).

# single vowel cmapua syllables early for connectives

a <- ([Aa] !badstress juncture? !V1)

e <- ([Ee] !badstress juncture? !V1)

i <- ([Ii] !badstress juncture? !V1)

o <- ([Oo] !badstress juncture? !V1)

u <- ([Uu] !badstress juncture? !V1)


Hearly <- (!predstart [Hh])

Nearly <- (!predstart [Nn])


# these appear here for historical reasons and could be moved later


connective <- [ ]* !predstart ([Nn] [Oo] juncture?)? (a/e/o/u/Hearly a/Nearly UU) juncture? !V2 !(!predstart [Ff] [Ii]) !(!predstart [Mm] [Aa]) !(!predstart [Zz] [Ii])


# cmapua units starting with consonants.  This is the exact description from NB3.  The fancy tail in each of the 
# three cases is enforcing the rule about pausing before a following predicate if stressed.

# consonant initial cmapua units may not be followed by vowels without pause.

# I am adding <iy> and <uy> (always monosyllable, yuh and wuh) as vowel pairs permitted in VV and CVV cmapua units.
# it is worth noting that the "yuh" and "wuh" pronunciations of these diphthongs
# are surprising to the English-reading eye.
# The use for this envisaged is that the name <ziy> of Y becomes easy to introduce.  Adding word space
# is always nice, and these words seem pronounceable.  I also made <yfi> possible:  Y now has phonetically
# regular names.

CmapuaUnit <- (C1 Mono juncture? V2 !(['*] [ ]* &C1 predstart) juncture? !V1/C1 (VV/[Ii][Yy]/[Uu][Yy]) !(['*] [ ]* &C1 predstart) juncture? !V1/C1 V2 !(['*] [ ]* &C1 predstart) juncture? !V1) 

# A stream of cmapua is read until the start of a predicate or a name marker word or an alien text marker word or a quote or parenthesis marker word is encountered.
# the stream might resume with a name marker word if it does not in fact start a name word and does not potentially start a name
# word due to inexplicit whitespace (doesn't satisfy the bad name marker condition).

# we force explicit comma pauses before logical connectives, but not before vowel initial cmapua in general;
# other conditions force at least whitespace, which does stand for a pause, before such words.

# detect starts of quotes or parentheses with <li> or <kie>

likie <- ([Ll] [Ii] juncture? !V1/[Ki] [Ii] juncture? [Ee] juncture? !V1)

# a special provision is made for NO UI forms as single words.  <yfi> is supported.

Cmapua <- &caprule !badnamemarker (!predstart (VV/[Ii][Yy]/[Uu][Yy]) !(['*] [ ]* &C1 predstart) juncture? NOI/!predstart [Nn] [Oo] juncture? !predstart (VV/[Ii][Yy]/[Uu][Yy]) !(['*] [ ]* &C1 predstart) juncture?/((!predstart (VV/[Ii][Yy]/[Uu][Yy]) !(['*] [ ]* &C1 predstart) juncture?)+ / ((!predstart V1 !(['*] [ ]* &C1 predstart) juncture?)/ !predstart CmapuaUnit) (!namemarker !alienmarker !likie !predstart CmapuaUnit)*)/!predstart V2 !(['*] [ ]* &C1 predstart) juncture?) !V1 !(C1+ juncture) !([ ]* connective)

# I have apparently now completely solved the problem of parsing cmapua as well as name words.

# Now for predicates.

# the elementary djifoa (not borrowings)

# various special flavors of these djifoa will be needed.
# These are the general definitions.

# The NOY and Bad forms are for use for testing candidate borrowings for resolution
# with bad syllable break placements.  Borrowings do not contain Y...

# CVV djifoa with phonetic hyphens.

# added checks to all cmapua classes:  the vowel final ones, when not phonetically hyphenated, cannot
# be followed by a regular vowel.  This is crucial for getting the syllable analysis and the djifoa
# analysis to end at the same point.

# allowing h to be inserted before y in CVVy djifoa for a CVVhy form.

# allowing -r glue to be expressed as -rr

CVV <- C1 VV (juncture? [Hh]? [Yy] [-]? &(Complex) /juncture? [Rr] [Rr]? juncture? &C1/[Nn] juncture? &[Rr]/juncture? !V2)

CVVNoHyphen <- C1 VV juncture? !V2

CVVHiddenStress <- C1 &DoubleVowel V1 [-]? V1 ([-]? [Hh]? [Yy] [-]? &Complex /[Rr] [-]? &C1/[Nn] [-]? &[Rr]/[-]? !V2)

CVVFinalStress <- C1 VV (['*] [Hh]? [Yy] [-]? &Complex /[Rr] ['*] &C1/['*] [Rr] [Rr] juncture? &C1/[Nn] ['*] &[Rr]/['*] !V2)

CVVNOY <- C1 VV (juncture? [Rr] [Rr]? juncture? &C1/[Nn] juncture? &[Rr]/juncture? !V2) 

CVVNOYFinalStress <- C1 VV ([Rr] ['*] &C1/['*] [Rr] [Rr] juncture? &C1/[Nn] ['*] &[Rr]/['*] !V2)

CVVNOYMedialStress <- C1 !BrokenMono V2 ['*] V2 [-]? !V2

# CCV djifoa with phonetic hyphens.

CCV <- Initial V2 (juncture? [Yy] [-]? &letter/juncture? !V2)

CCVStressed <- Initial V2 (['*] [Yy] [-]? &letter/['*] !V2)

CCVNOY <- Initial V2 juncture? !V2

CCVBad <- MaybeInitial V2 juncture? !V2

CCVBadStressed <- MaybeInitial V2 ['*] !V2


# CVC djifoa with phonetic hyphens.  These cannot be final and are always followed by a consonant (well, the
# -y form may be followed by a vowel...
# an eccentric syllable break is supported if the CVC is y-hyphenated:
# <me-ky-kiu> and <mek-y-kiu> are both legal.  The default is the latter.

CVC <- (C1 V2 !NoMedial2 !NoMedial3 C1 (juncture? [Yy] [-]? &letter/juncture? &C1)/C1 V2 juncture C1 [Yy] [-]? &letter)

CVCStressed <- (C1 V2 !NoMedial2 !NoMedial3 C1 (['*] [Yy] [-]? &letter/['*] &letter)/C1 V2 ['*] C1 [Yy] [-]? &letter)

CVCNOY <- C1 V2 !NoMedial2 !NoMedial3 C1 juncture? &C1

CVCBad <- C1 V2 !NoMedial2 !NoMedial3 juncture? C1 &C1

CVCNOYStressed <- C1 V2 !NoMedial2 !NoMedial3 C1 ['*] &C1

CVCBadStressed <- C1 V2 !NoMedial2 !NoMedial3 ['*] C1 &C1

# the five letter forms (always final in complexes)

CCVCV <- Initial V2 juncture? C1 V2 [-]? !V2

CCVCVStressed <- Initial V2 ['*] C1 V2 [-]? !V2

CCVCVBad <- MaybeInitial V2 juncture? C1 V2 [-]? !V2

CCVCVBadStressed <- MaybeInitial V2 ['*] C1 V2 [-]? !V2

CVCCV <- (C1 V2 juncture? Initial V2 [-]? !V2/C1 V2 !NoMedial2 C1 juncture? C1 V2 [-]? !V2)

CVCCVStressed <- (C1 V2 ['*] Initial V2 [-]? !V2/C1 V2 !NoMedial2 C1 ['*] C1 V2 [-]? !V2)

# the medial five letter djifoa

CCVCY <- Initial V2 juncture? C1 [Yy] [-]?

CVCCY <- (C1 V2 juncture? Initial [Yy] [-]?/C1 V2 !NoMedial2 C1 juncture? C1 [Yy] [-]?)

CCVCYStressed <- Initial V2 ['*] C1 [Yy] [-]?

CVCCYStressed <- (C1 V2 ['*] Initial [Yy] [-]?/C1 V2 !NoMedial2 C1 ['*] C1 [Yy] [-]?)

# to reason about resolution of borrowings into both syllables and djifoa (we want to exclude the latter
# but we need to define it adequately) we need to recognize where to stop.  A predicate word ends either
# at a non-character (not a letter or syllable mark: whitespace, comma or terminal punctuation) or it
# has an explicit or deducible penultimate stress.  Borrowings do not contain doubled vowels, so they
# have to have explicit stress in the latter case.

# analysis:  the stressed tail consists of a stressed syllable followed by an unstressed syllable.
# identifying an unstressed final syllable is complicated by recognizing which CVV combinations can
# be one syllable.  This will either be an explicitly stressed syllable followed by a single syllable
# or a syllable suitable to be stressed followed by an explicitly final syllable.  CVV djifoa can
# contain both syllables in a tail and of course the five letter djifoa have to be tails.  A never stressed
# SyllableC (with a continuant) may intervene.

# tail of a borrowing with an explicit stress

BorrowingTail1 <- !SyllableC &StressedSyllable BorrowingSyllable (!StressedSyllable &SyllableC BorrowingSyllable)? !StressedSyllable &BorrowingSyllable VowelFinal

# tail of a borrowing or borrowing djifoa with no explicit stress

BorrowingTail2 <- !SyllableC BorrowingSyllable (!StressedSyllable &SyllableC BorrowingSyllable)? !StressedSyllable &BorrowingSyllable VowelFinal (&[Yy]/!character)

# tail of a stressed borrowing djifoa, different because stress is shifted to the end

BorrowingTail3 <- !SyllableC !StressedSyllable BorrowingSyllable (!StressedSyllable &SyllableC BorrowingSyllable)? &BorrowingSyllable InitialConsonants? Vocalic ['*] &[Yy]

BorrowingTail <- BorrowingTail1 / BorrowingTail2

# short forms that are ruled out:  CCVV and CCCVV forms.

CCVV <- (InitialConsonants V2 juncture? V2 juncture? !character / InitialConsonants V2 ['*] !Mono V2 juncture?)

# VCCV and some related forms are ruled out (rule predstartF above is about this)

# a continuant syllable cannot be initial in a borrowing and there cannot be successive continuant
# syllables.  There really ought to be no more than one!

# borrowing, before checking that it doesnt resolve into djifoa

PreBorrowing <- &predstart!CCVV!Cmapua!SyllableC(!BorrowingTail!(StressedSyllable)!(SyllableC SyllableC)BorrowingSyllable)* BorrowingTail

# ditto for an explicitly stressed borrowing

StressedPreBorrowing <- &predstart!CCVV!Cmapua!SyllableC(!BorrowingTail!(StressedSyllable)!(SyllableC SyllableC)BorrowingSyllable)* BorrowingTail1

# borrowing djifoa without explicit stress (before resolution check)

PreBorrowing2 <- &predstart!CCVV!Cmapua!SyllableC(!BorrowingTail!(StressedSyllable)!(SyllableC SyllableC)BorrowingSyllable)* BorrowingTail2

# stressed borrowing djifoa (before resolution check).

PreBorrowing3 <- &predstart!CCVV!Cmapua!SyllableC(!BorrowingTail3!(StressedSyllable)!(SyllableC SyllableC)BorrowingSyllable)* BorrowingTail3

# Now comes the problem of trying to say that a preborrowing cannot resolve into cmapua.  The difficulty is with
# recognizing the tail, so making sure that the two resolutions stop in the same place.

# we know because it is a borrowing that there is at most one explicit stress, and it has to fall
# in one of the cmapua!  This should make it doable.

# borrowing djifoa are terminated with y, so the final djifoa needs to take this into account

# the idea behind both djifoa analyses is the same.  If we end with a final djifoa followed by
# a non-character, we improve our chances of ending the syllable analysis at the same point.  We control
# this by identifying djifoa with stresses in them:  a medially stressed djifoa must be the last one
# (and the syllable analysis will find its stressed syllable and end at its final syllable, the fact
# that djifoa cannot be followed by vowels ensuring that the syllable analysis cannot overrun its end.
# When the djifoa is finally stressed, the complex analysis ends with a further djifoa guaranteed to have
# just one syllable, and the syllable analysis again will stop in the same place.  The medial five letter forms
# and borrowing djifoa of course are finally stressed mod an additional unstressed syllable which is skipped
# by the syllable analysis, because it allows one to ignore an actually penultimate syllable with y or 
# a syllabic consonant.  In the case where we never find a stress and end up at a final djifoa, the syllable
# analysis will carry right through to the same final point.

# in the attempted resolution of borrowings, our life is easier because we do not have
# borrowing djifoa or medial five letter forms to consider, or any forms with y-hyphens.

RFinalDjifoa <- (CCVCVBad/CVCCV/CVVNoHyphen/CCVBad/CVCBad) (&[Yy]/!character)

RMediallyStressed <- (CCVCVBadStressed/CVCCVStressed/CVVNOYMedialStress)

RFinallyStressed <- (CVVNOYFinalStress/CCVBadStressed/CVCBadStressed/CVCNOYStressed)

BorrowingComplexTail <- (RMediallyStressed/RFinallyStressed (&(C1 Mono) CVVNoHyphen/CCVBad)/RFinalDjifoa)

ResolvedBorrowing <- (!BorrowingComplexTail(CVVNOY/CCVBad/CVCBad))* BorrowingComplexTail

# borrowed predicates

Borrowing <- !ResolvedBorrowing &caprule PreBorrowing !([ ]* (connective))

# explicitly stressed borrowed predicates

StressedBorrowing <- !ResolvedBorrowing &caprule StressedPreBorrowing !([ ]* &V1 Cmapua)

#This is the shape of non-final borrowing djifoa.  Notice that a final stress is allowed.
#The curious provision for explicitly stressing a borrowing djifoa and pausing is supported.

# borrowing djifoa without explicit stress (stressed ones are not of this class!)
# Note that one can pause after these (explicitly, with a comma)

BorrowingDjifoa <- !ResolvedBorrowing &caprule PreBorrowing2 (['*] y [,] [ ]+/juncture? [y] [-]?)

# stressed borrowing djifoa finally implemented!

StressedBorrowingDjifoa <- !ResolvedBorrowing &caprule PreBorrowing3 [y] [-]? ([,] [ ]+)?

# We resolve complexes twice, once into syllables and once into djifoa.  We again have to ensure that
# we end up in the same place!  The syllable resolution is very similar to that of borrowings;
# the unstressed middle syllable of the tail can be a SyllableY, and can also be a
# SyllableC if the final djifoa is a borrowing.

# A stressed borrowing djifoa with the property that the tail is still a phonetic complex is
# a unit for this analysis.

# note here that I specifically rule out a complex being followed without pause by y.  I do not rule
# this out for the vowel final djifoa because they can be followed by y at the end of a borrowing
# djifoa.

PhoneticComplexTail1 <- !SyllableC !SyllableY &StressedSyllable Syllable (!StressedSyllable &(SyllableC/SyllableY) Syllable)? !StressedSyllable !SyllableY VowelFinal !V1

PhoneticComplexTail2 <- !SyllableC !SyllableY Syllable (!StressedSyllable &(SyllableC/SyllableY) Syllable)? !StressedSyllable !SyllableY VowelFinal !character

PhoneticComplexTail <- PhoneticComplexTail1 / PhoneticComplexTail2

# note the explicit predstart test here.

PhoneticComplex <- &predstart!CCVV!Cmapua!SyllableC(StressedBorrowingDjifoa &PhoneticComplex/!PhoneticComplexTail!(StressedSyllable)!(SyllableC SyllableC) Syllable)* PhoneticComplexTail

# the analysis of final djifoa and stressed djifoa differs only in details from
# what is above for resolution of borrowings.  The issues about CVV djifoa with doubled
# vowels are rather exciting.

# a stressed borrowing djifoa with the tail still a phonetic complex is a black box unit for
# this construction.

# My approach imposes the restriction on JCB's "pause after a borrowing djifoa" idea that what follows
# the pause must itself contain a penultimate stress:  <igllu'ymao> is a predicate but <igllu'y, mao> is not.
# while <iglluy', gudmao> is a predicate.

# the analysis of the djifoa resolution process is the same as above, with additional remarks
# about doubled vowel syllables:  notice that where the complex tail involved a doubled vowel syllable
# without explicit stress, we insist on that djifoa or the single-syllable next djifoa ending in
# a non-character:  in the absence of explicit stress, we always rely on whitespace or punctuation
# to indicate the end of the predicate.

# all sorts of subtleties about borrowings and borrowing djifoa are finessed by always looking for
# them first.  There are no restrictions re fronts of borrowings or borrowing djifoa looking like regular
# djifoa;  the fact that borrowing djifoa end in y and borrowings do not contain y makes it always
# possible to tell when one is looking at the head of a borrowing djifoa.  Regular djifoa just before a borrowing
# djifoa need to be y-hyphenated so as not to be absorbed into the front of the borrowing (I don't believe
# that I actually need to impose a formal rule to this effect, though I am not absolutely certain;  it would
# be difficult to formulate [and does appear in the previous version, where it is a truly unintelligible piece
# of PEG code]).

FinalDjifoa <- (Borrowing/CCVCV/CVCCV/CVVNoHyphen/CCVNOY) !character

MediallyStressed <- (StressedBorrowing/CCVCVStressed/CVCCVStressed/CVVNOYMedialStress)

FinallyStressed <-(StressedBorrowingDjifoa/CCVCYStressed/CVCCYStressed/CVVFinalStress/CCVStressed/CVCStressed)

ComplexTail <- (CVVHiddenStress (&(C1 Mono) CVVNoHyphen/CCVNOY) !character/FinallyStressed (&(C1 Mono) CVVNoHyphen/CCVNOY)/MediallyStressed/FinalDjifoa)

PreComplex <-  (!CVVHiddenStress (!ComplexTail)(StressedBorrowingDjifoa &PhoneticComplex/BorrowingDjifoa/CVCCY/CCVCY/CVV/CCV/CVC))* ComplexTail

# originally I had complicated tests here for the conditions under which an initial
# CVC cmapua has to be y-hyphenated:  I was being wrong headed, the predstart rules
# already enforce this (in the bad cases, the initial CV- falls off).  The user will
# simply find that they cannot put the word together otherwise.  The previous version
# did need this test because it actually used full lookahead to check for the start of a predicate.

Complex <- &caprule &PreComplex PhoneticComplex !([ ]* (connective))

# format for the LI quote and KIE parenthesis

LiQuote <- (&caprule [Ll][Ii]juncture? comma2? [\"] phoneticutterance [\"] comma2? &caprule [Ll][Uu]juncture? !([ ]* connective)/(&caprule [Kk][Ii]juncture?[Ee]juncture? comma2? [(] phoneticutterance [)] comma2? &caprule [Kk][Ii]juncture?[Uu]juncture? !([ ]* connective)))

# the condition on Word that a Cmapua is not followed by another Cmapua
# with mere whitespace between was used by <liu> quotation, but is now redundant,
# because I have required that <liu> quotations be closed with explicit pauses in all cases.

Word <- (NameWord / Cmapua !([ ]*Cmapua)/ Complex/CCVNOY)

# it is an odd point that all borrowings parse as complexes -- so when I parsed all the words the first time they all
# parsed as complexes.  A borrowing is a complex consisting of a single final borrowing djifoa!
# I did redesign this so that borrowings are parsed as borrowings.  (This is the class
# I used to parse the dictionary).

# Yes, CVC djifoa do get parsed as names in the dictionary, so the CVC case here is redundant.  I actually
# think that only the CCV djifoa actually get parsed as such.

SingleWord <- (Borrowing !./Complex !./ Word !./PreName !. /CCVNOY) !.

# name word appearing initially without leading spaces is important, because one type of NameWord includes a leading comma.

phoneticutterance1 <- (NameWord /[ ]* LiQuote/[ ]* NameWord/[ ]* AlienWord/[ ]*Cmapua/[ ]* '--'/[ ]* '...'/[ ]* Borrowing![y]/[ ]* Complex/[ ]* (CCVNOY))+

phoneticutterance <- (phoneticutterance1/[,][ ]+/terminal)+

# consonants and vowel groups in cmapua

# as noted above, !predstart stands in for the computationally disastrous &Cmapua

badstress <- ['*] [ ]* &C1 predstart

B <- (!predstart [Bb])

C <- (!predstart [Cc])

D <- (!predstart [Dd])

F <- (!predstart [Ff])

G <- (!predstart [Gg])

H <- (!predstart [Hh])

J <- (!predstart [Jj])

K <- (!predstart [Kk])

L <- (!predstart [Ll])

M <- (!predstart [Mm])

N <- (!predstart [Nn])

P <- (!predstart [Pp])

R <- (!predstart [Rr])

S <- (!predstart [Ss])

T <- (!predstart [Tt])

V <- (!predstart [Vv])

Z <- (!predstart [Zz])

# the monosyllabic classes may be followed by one vowel
# if they start a Cvv-V cmapua unit;  the others may never
# be followed by vowels.  Classes ending in -b are
# used in Cvv-V cmapua units.

# the single vowel classes were moved before the class
# connective in the phonetics section.


V3 <- juncture? V2 !badstress

AA <- ([Aa] juncture? [Aa] !badstress juncture? !V1)  

AE <- ([Aa] juncture? [Ee]  !badstress juncture? !V1)  

AI <- ([Aa] [Ii]  !badstress juncture? !(V1))  

AO <- ([Aa] [Oo]  !badstress juncture? !(V1)) 

AIb <- ([Aa] [Ii]  !badstress juncture? &(V2 juncture? !V1))  

AOb <- ([Aa] [Oo]  !badstress juncture? &(V2 juncture? !V1))  
 
AU <- ([Aa] juncture? [Uu]  !badstress juncture? !V1)  

EA <- ([Ee] juncture? [Aa]  !badstress juncture? !V1)  

EE <- ([Ee] juncture? [Ee]  !badstress juncture? !V1)  

EI <- ([Ee] [Ii]  !badstress juncture? !(V1)) 

EIb <- ([Ee] [Ii]  !badstress juncture? &(V2 juncture? !V1))  

EO <- ([Ee] juncture? [Oo]  !badstress juncture? !V1)  

EU <- ([Ee] juncture? [Uu]  !badstress juncture? !V1)  

IA <- ([Ii] juncture? [Aa]   !badstress juncture? !(V1))  

IE <- ([Ii] juncture? [Ee]  !badstress juncture? !(V1))  

II <- ([Ii] juncture? [Ii]  !badstress juncture? !(V1))  

IO <- ([Ii] juncture? [Oo]  !badstress juncture? !(V1))  

IU <- ([Ii] juncture? [Uu]   !badstress juncture? !(V1)) 

IAb <- ([Ii] juncture? [Aa]   !badstress juncture? &(V2 juncture? !V1))  

IEb <- ([Ii] juncture? [Ee]  !badstress juncture? &(V2 juncture? !V1))  

IIb <- ([Ii] juncture? [Ii]  !badstress juncture? &(V2 juncture? !V1))  

IOb <- ([Ii] juncture? [Oo]  !badstress juncture? &(V2 juncture? !V1))  

IUb <- ([Ii] juncture? [Uu]   !badstress juncture? &(V2 juncture? !V1))   

OA <- ([Oo] juncture? [Aa]  !badstress juncture? !V1)  

OE <- ([Oo] juncture? [Ee]  !badstress juncture? !V1)  

OI <- ([Oo] [Ii]  !badstress juncture? !(V1)) 

OIb <- ([Oo] [Ii]  !badstress juncture? &(V2 juncture? !V1)) 

OO <- ([Oo] juncture? [Oo]  !badstress juncture? !V1)  

OU <- ([Oo] juncture? [Uu]   !badstress juncture? !V1)  

UA <- ([Uu] juncture? [Aa]   !badstress juncture? !(V1))  

UE <- ([Uu] juncture? [Ee]  !badstress juncture? !(V1))  

UI <- ([Uu] juncture? [Ii]   !badstress juncture? !(V1))  

UO <- ([Uu] juncture? [Oo]  !badstress juncture? !(V1))  

UU <- ([Uu] juncture? [Uu]  !badstress juncture? !(V1)) 

UAb <- ([Uu] juncture? [Aa]   !badstress juncture? &(V2 juncture? !V1))  

UEb <- ([Uu] juncture? [Ee]  !badstress juncture? &(V2 juncture? !V1))  

UIb <- ([Uu] juncture? [Ii]   !badstress juncture? &(V2 juncture? !V1))  

UOb <- ([Uu] juncture? [Oo]  !badstress juncture? &(V2 juncture? !V1))  

UUb <- ([Uu] juncture? [Uu]  !badstress juncture? &(V2 juncture? !V1))   

# adding the new IY and UY, which might see use some time.
# they are mandatory monosyllables but do not take a possible additional
# following vowel as the regular ones do.  So far only used in <ziy>.

IY <- [Ii] [Yy] !badstress juncture? !V1

UY <- [Uu] [Yy] !badstress juncture? !V1

# this is a pause not required by the phonetics.  This is the only
# sort of pause which could in principle carry semantic freight (the
# pause/GU equivalence beloved of our Founder) but we have abandoned
# this.  There is one place, after initial <no> in an utterance, where
# a pause can have effect on the parse (but not on the meaning, I believe, 
# unless a word break is involved).

# this class should NEVER be used in a context which might follow
# a name word.  In previous versions, pauses after name words were included
# in the name word;  this is not the case here, so a PAUSE
# after a name word would not be recognized as a mandatory pause.

# in any event, as long as we stay away from pause/GU equivalence, this
# is not a serious issue!

# this class does do some work in the handling of issues surrounding the legacy
# shape of APA connectives, concerning which the less said, the better.

PAUSE <- [,] [ ]+ !(V1/connective) &caprule

# more punctuation

comma <- [,] [ ]+ &caprule

comma2 <- [,]? [ ]+ &caprule

# Part II Lexicography

# In this section I develop the grammar of words in Loglan.  I'll work by editing the original provisional PEG grammar.

# I place the start of this section exactly here, just before two final items of
# punctuation, because these items of punctuation look forward not only to lexicography
# but to the full grammar!

# the end of utterance symbol <#> should be added in the phonetics
# section as a species of terminal marker. Done.  We do *not* actually
# endorse use of this marker, but we can notionally support it and it is in 
# our sources.

end <- (([ ]* '#' [ ]+ utterance)/([ ]+ !.)/!.)

# this rule allows terminal punctuation to be followed by an inverse vocative,
# a frequent occurrence in Leith's novel, and something which makes sense.

period <- (([!.:;?] (&end/([ ]+ &caprule))) (invvoc period?)?)

# Letters with y will be special cases
# idea:  allow IY and UY (always monosyllables) as vowel combinations in cmapua only.
# done:  Y has a name now.  <yfi> is also added.

# the classes in this section after this point are the cmapua word classes of Loglan (if they begin with [ ]* or a word class).
# I suppose the alien text classes are not really word classes, but they are lexicographic items, as it were.
# Paradoxically, the PA and NI classes admit internal explicit pauses.  So of course do predicate words!

# Loglan does admit true multisyllable cmapua:  there are words made of cmapua units which have joints between
# units at which one cannot pause without breaking the word.  Lojban, I am told, does not.

# this version has the general feature that the quotation and alien text constructions are not hacked:
# they are supported by the phonetic rules (as dire exceptions, of course) and the grammatical constructions
# conform with the phonetic layer.  Alien text and utterances quoted with <li>...<lu> can be enclosed in double quotes.
# LI only supports full utterances, for the moment.  All alien text constructors take the same class as argument:
# the vocative and inverse vocative *require* quotes to avoid misreading ungrammatical expressions with typos
# as correct (inverse) vocatives.

# the names <yfi>, <ziy> for Y are supported.  The Ceo names are left as they are.  I decided that a second short series
# of letteral pronouns is actually a reasonable use of short words, and the Ceio words are there for other uses.

TAI0 <- (V1 juncture? M a/V1 juncture? F i/V1 juncture? Z i/!predstart C1 AI/!predstart C1 EI/!predstart C1 AIb u/!predstart C1 EIb (u)/!predstart C1 EO/ Z [Ii] V1 !badstress juncture? !V1 (M a)?)

# a negative suffix used in various contexts.  Always a suffix:  its use as a prefix in tenses was a mistake in NB3 and I 
# think still supported in LIP.  Ambiguities demonstrably followed from this usage (an example of how the demonstration
# of non-ambiguity of 1989 Loglan was compromised by the opaque lexicography).

NOI <- (N OI)

# the logical connectives.  [A0] is the class of core logical connectives.  [A] is the fully decorated logical connective with
# possible nu- (always in nuno- or nuu) and no- prefixes, possible -noi suffix, and possible (problematic) PA suffix, closed
# with -fi (our new proposal) or an explicit pause.

A0 <- &Cmapua (a/e/o/u/H a/N UU)

A <- [ ]* !predstart !TAI0 (N [o])? A0 NOI? !([ ]+ PANOPAUSES PAUSE) !(PANOPAUSES !PAUSE [ ,]) (PANOPAUSES ((F i)/&PAUSE))?

# 4/18 in connected sentpreds, fi must be used to close, not a pause.

# A2 <- [ ]* !predstart !TAI0 (N [o])? A0 NOI? !([ ]+ PANOPAUSES PAUSE) !(PANOPAUSES !PAUSE [ ,]) (PANOPAUSES (F i))?



# A not closed with -fi or a pause

ANOFI <- [ ]* (!predstart !TAI0 ( (N [o])? A0 NOI? PANOPAUSES?))

A1 <- A

# versions of A with different binding strength

ACI <- (ANOFI C i)

AGE <- (ANOFI G e)

# a tightly binding series of logical connectives used to link predicates
# this also includes the fusion connective <ze> when used between predicates.

CA0 <- (( (N o)? ((C a)/(C e)/(C o)/(C u)/(Z e)/(C i H a)/N u C u)) NOI?)

CA1 <- (CA0 !([ ]+ PANOPAUSES PAUSE) !(PANOPAUSES !PAUSE [ ,]) (PANOPAUSES ((F i)/&PAUSE))?)

CA1NOFI <- (CA0 PANOPAUSES?)

CA <- ([ ]* CA1)

# the fusion connective when used in arguments

ZE2 <- ([ ]* (Z e))

# sentence connectives.  [I] is the class of utterance initiators (no logical definition).
# the subsequent classes are inhabited by sentence logical connectives with various binding
# strengths.

I <- ([ ]* !predstart !TAI0 i !([ ]+ PANOPAUSES PAUSE) !(PANOPAUSES !PAUSE [ ,]) (PANOPAUSES ((F i)/&PAUSE))?)

ICA <- ([ ]* i ((H a)/CA1))

ICI <- ([ ]* i CA1NOFI? C i)

IGE <- ([ ]* i CA1NOFI? G e)

# forethought logical connectives

KA0 <- ((K a)/(K e)/(K o)/(K u)/(K i H a)/(N u K u))

# causal and comparative modifiers

KOU <- ((K OU)/(M OI)/(R AU)/(S OA)/(M OU)/(C IU))

# negative and converse forms

KOU1 <- (((N u N o)/(N u)/(N o)) KOU)

# the full type of forethought connectives, adding the causal and comparative connectives

KA <- ([ ]* ((KA0)/((KOU1/KOU) K i)) NOI?)

# the last component of the KA...KI... structure of forethought connections

KI <- ([ ]* (K i) NOI?)

# causal and comparative modifiers which are *not* forethought connectives

KOU2 <- (KOU1 !KI)

# a test used to at least partially enforce the penultimate stress rule on quantifier predicates

BadNIStress <- ((C1 V2 V2? stress (M a)? (M OA)? NI RA)/(C1 V2 stress V2 (M a)? (M OA)? NI RA))

# root quantity words, including the numerals

NI0 <- (!BadNIStress ((K UA)/(G IE)/(G IU)/(H IE)/(H IU)/(K UE)/(N EA)/(N IO)/(P EA)/(P IO)/(S UU)/(S UA)/(T IA)/(Z OA)/(Z OO)/(H o)/(N i)/(N e)/(T o)/(T e)/(F o)/(F e)/(V o)/(V e)/(P i)/(R e)/(R u)/(S e)/(S o)/(H i)))

# the class of SA roots, which modify quantifiers

SA <- (!BadNIStress ((S a)/(S i)/(S u)/(IE (comma2? !IE SA)?)) NOI?)

# the family of quantifiers which double as suffixes for the quantifier predicates
# this class perhaps should also include some other quantifier words. <re> for example ought to be handled in the same way as <ra,ri,ro>.
# No action here, just a remark.

RA <- (!BadNIStress ((R a)/(R i)/(R o)/R e/R u))

# re and ru added to class RA 5/11/18

# quantifier units consisting of a NI or RA root with <ma> 00 or <moa> 000 appended; to <moa> one can further
# append a digit to iterate <moa>:  <fomoate> is four billion, for example.  <rimoa>, a few thousand.

# a NI1 or RA1 may be followed by a pause before another NI word other than a numerical predicate;
# one is allowed to breathe in the middle of long numerals.  I question whether the pause
# provision makes sense in RA1.

NI1 <- ((NI0 (!BadNIStress M a)? (!BadNIStress M OA NI0*)?) (comma2 !(NI RA) &NI)?)

RA1 <- ((RA (!BadNIStress M a)? (!BadNIStress M OA NI0*)?) (comma2 !(NI RA) &NI)?)

# a composite NI word, optional SA prefix before a sequence of NI words or a RA word,
# or a single SA word [which will modify a default quantifier not expressed], 
# possibly negated, connected with CA0 roots to other such constructs.

NI2 <- (( (SA? (NI1+/RA1))/SA) NOI? (CA0 ((SA? (NI1+/RA1))/SA) NOI?)*)

# a full NI word with an acronymic dimension (starting with <mue>, ending with a pause) or <cu> appended.  I need to look up <cu>
# and figure out its semantics.  An arbitrary name word may now be used as a dimension, as well.

NI <- ([ ]* NI2 (&(M UE) Acronym (comma/&end/&period) !(C u)/comma2? M UE comma2? PreName !(C u))? (C u)?)

# mex is now identical with NI, but it's in use in later rules.

mex <- ([ ]* NI)

# a word used for various tightly binding constructions:  a sort of verbal hyphen.
# also a name marker, which means phonetic care is needed (pause after constructions with <ci>).

CI <- ([ ]* (C i))

# Acronyms, which are names (not predicates as in 1989 Loglan) or dimensions (in NI above).
# units in acronym are TAI0 letterals, zV short forms for vowels, the dummy unit <mue>, and NI1
# quantity units.  NI1 quantity units may not be initial. <mue> units may be preceded by pauses.
# An acronym has at least two units.

# it is worth noting that acronyms, once viewed as names, could be entirely suppressed as a feature of the
# grammar by really making them names (terminate them with -n).  I suppose a similar approach would work
# for dimensions, allowing any name word to serve as a dimension.  <mue> would be a name marker for use
# with dimensions in this case.  <temuedain>, three dollars.  Now supported.

Acronym <- ([ ]* &caprule ((M UE)/TAI0/(Z V2 !V2)) ((comma &Acronym M UE)/NI1/TAI0/(Z V2 (!V2/(Z &V2))))+)

# the full class of letterals, including the <gao> construction whose details I should look at.

TAI <- ([ ]* (TAI0/((G AO) !V2 [ ]* (PreName/Predicate/CmapuaUnit))))

# atomic non-letteral pronouns.

#4/15/2019 reserved <koo> for a Lojban style imperative pronoun, though not officially adopting it.  Also adding <dao> for a default, don't care argument, another Lojban feature.

DA0 <- ((T AO)/(T IO)/(T UA)/(M IO)/(M IU)/(M UO)/(M UU)/(T OA)/(T OI)/(T OO)/(T OU)/(T UO)/(T UU)/(S UO)/(H u)/(B a)/(B e)/(B o)/(B u)/(D a)/(D e)/(D i)/(D o)/(D u)/(M i)/(T u)/(M u)/(T i)/(T a)/(M o)/(K OO)/(D AO))

# letterals (not including <gao> constructions and atomic pronouns optionally suffixed with a digit.  One should pause after the
# suffixed forms, because <ci> is a name marker.

DA1 <- ((TAI0/DA0) (C i ![ ] NI0)?)

# general pronoun words.

DA <- ([ ]* DA1)

# roots for PA words:  tense and location words, prepositions building relative modifiers.  All can optionally be negated with -noi.  They may also be quantified.  They may also be closed with ZI class affixes.  PA cores.

PA0 <- (NI2? (N u !KOU)? ((G IA)/(G UA)/(P AU)/(P IA)/(P UA)/(N IA)/(N UA)/(B IU)/(F EA)/(F IA)/(F UA)/(V IA)/(V II)/(V IU)/(C OI)/(D AU)/(D II)/(D UO)/(F OI)/(F UI)/(G AU)/(H EA)/(K AU)/(K II)/(K UI)/(L IA)/(L UI)/(M IA)/(N UI)/(P EU)/(R OI)/(R UI)/(S EA)/(S IO)/(T IE)/ (V IE)/(V a)/(V i)/(V u)/(P a)/(N a)/(F a)/(V a)/(KOU !(N OI) !KI)) (N OI)? ZI?)

# the form used for actual prepositions and suffixes to A words, with minimal pauses allowed.
# these are built by concatenating KOU2 and PA0 units, then linking these with CA0 roots (which can take
# no- prefixes and -noi suffixes, and next to which one *can* pause), optionally suffixed with a class ZI suffix.

PANOPAUSES <- ((KOU2/PA0)+ ((comma2? CA0 comma2?) (KOU2/PA0)+)*)

# prepositional words

PA3 <- ([ ]* PANOPAUSES)

# class PA can appear as tense markers or as relative modifiers without arguments; here pauses
# are allowed not only next to CA0 units but between KOU2/PA units.  Like NI words, PA
# words are a class of arbitrary length constructions, and we think breaths within them
# (especially complex ones) are natural.

PA <- ((KOU2/PA0)+ (((comma2? CA0 comma2?)/(comma2 !mod1a)) (KOU2/PA0)+)*) !modifier

PA2 <- ([ ]* PA)

GA <- ([ ]* (G a))

# the class of tense markers which can appear before predicates.

PA1 <- ((PA2/GA))

# suffixes which indicate extent or remoteness/proximity of the action of prepositions.

ZI <- ((Z i)/(Z a)/(Z u))

# the primitive description building "articles".  These include <la> which requires special
# care in its use because it is a name marker.

LE <- ([ ]* ((L EA)/(L EU)/(L OE)/(L EE)/(L AA)/(L e)/(L o)/(L a)))

# articles which can be used with abstract descriptions:  these include some quantity words.
# this means that some abstract descriptions are semantically indefinites:  I wonder if this
# could be improved by having a separate abstract indefinite construction.

LEFORPO <- ([ ]* ((L e)/(L o)/NI2))

# the numerical/quantity article.

LIO <- ([ ]* (L IO))

# structure words for the ordered and unordered list constructions.

LAU <- ([ ]* (L AU))

LOU <- ([ ]* (L OU))

LUA <- ([ ]* (L UA))

LUO <- ([ ]* (L UO))

ZEIA <- ([ ]* Z EIb a)

ZEIO <- ([ ]* Z EIb o)

# initial and final words for quoting Loglan utterances.

LI1 <- (L i)

LU1 <- (L u)

# quoting Loglan utterances, with or without explicit double quotes (if they appear, they must
# appear on both sides).  The previous version allowed quotation of names;  likely this should
# be restored.

LI <- ([ ]* LI1 comma2? utterance0 comma2? LU1/[ ]* LI1 comma2? [\"] utterance0 [\"] comma2? LU1)

# the foreign name construction.  This is an alien text construction

LAO <- ([ ]* &([Ll] [Aa] [Oo]juncture?) AlienWord)

# the strong quotation construction.  This is an alien text construction.

LIE <- ([ ]* &([Ll] [Ii] juncture? [Ee]juncture?) AlienWord)

LIO <- ([ ]* &([Ll] [Ii] juncture? [Oo]juncture?) AlienWord)



# I am not sure this class is used at all.

LW <- Cmapua

# articles for quotation of words

LIU0 <- ((L IU)/(N IU))

# this now imposes the condition that an explicit comma pause (or terminal punctuation, or end) must appear at the end of the
# Word or PreName quoted with <liu>.  This seems like a good idea, anyway.

# this class appeals to the phonetics.  Words and PreNames can be quoted.  The ability to quote names
# here may remove the need to quote them with <li>...<lu>.  Of course, some Words are in fact phrases rather
# than single words:  we will see whether the privileges afforded are used.  The final clause allows
# use of letterals as actual names of letters.

# added <niu>:  didn't make it a name marker.

LIU1 <- ([ ]* ([Ll]/[Nn])[iI] juncture? [Uu] juncture? !V1 comma2? (PreName/Word) &(comma/terminal/end) /[ ]*(L II TAI ))

# the construction of foreign and onomatopoeic predicates.  These are alien text constructions.

SUE <- ([ ]* &([Ss] [Uu] juncture? [Ee] juncture?/[Ss] [Aa] [Oo] juncture?) AlienWord)

# left marker in a predicate metaphor construction

CUI <- ([ ]* (C UI) )

# other uses of GA

GA2 <- ([ ]* (G a) )

# ge/geu act as "parentheses" to make an atomic predicate from a complex metaphorically
# and logically connected predicates;  <ge> has other left marking uses.

GE <- ([ ]* (G e) )

GEU <- ([ ]* ((C UE)/(G EU)) )

# final marker of a list of head terms

GI <- ([ ]* ((G i)/(G OI)) )

# used to move a normally prefixed metaphorical modifier after what it modifies.

GO <- ([ ]* (G o) )

# marker for second and subsequent arguments before the predicate; NEW

GIO <- ([ ]* (G IO) )

# the generic right marker of many constructions.

GU <- ([ ]* (G u) )

# various flavors of right markers.

# It should be noted that at one point I executed a program of simplifying these to
# reduce the likelihood that multiple <gu>'s would ever be needed to close an utterance.
# first of all, I made the closures leaner, moving them out of the classes closed
# to their clients so that they generally can be used only when needed. 
# Notably, the grammar of <guu> is quite different.   Second,
# I introduced some new flavors of right marker.  All can be realized with <gu>,
# but if one knows the right flavor one can close the right structure with a single
# right closure.

# right markers of subordinate clauses (argument modifiers).
# <gui> closes a different class than in the trial.85 grammar, with
# similar but on the whole better results.

GUIZA <- ([ ]* (G UI) (Z a) )

GUIZI <- ([ ]* (G UI) (Z i) )

GUIZU <- ([ ]* (G UI) (Z u) )

GUI <- (!GUIZA !GUIZI !GUIZU ([ ]* (G UI) ))

# right markers of abstract predicates and descriptions.
# probably the forms with z are to be preferred (and the other
# two are not needed) but I preserve all five classes for now. 

GUO <- ([ ]* (G UO) )

GUOA <- ([ ]* (G UOb a/G UO Z a) )

GUOE <- ([ ]* (G UOb e) )

GUOI <- ([ ]* (G UOb i/G UO Z i) )

GUOO <- ([ ]* (G UOb o) )

GUOU <- ([ ]* (G UOb u/G UO Z u) )

# right marker used to close term (argument/predicate modifier) lists.
# it is important to note that in our grammar GUU is not a component of
# the class termset, nor is it a null termset:  it appears in other classes
# which include termsets as an option to close them.  The effects are similar
# to those in the trial.85 grammar, but there is less of a danger that
# extra unexpected closures will be needed.

GUU <- ([ ]* (G UU) )

# a new closure for arguments in various contexts

GUUA <- ([ ]* (G UUb a) )

# a new closure for sentences.  In particular, it
# may have real use in closing up the scope of a list of
# fronted terms before a series of logically connected sentences.

GIUO <- ([ ]* (G IUb o) )

# right marker used to close arguments tightly linked with JE/JUE.

GUE <- ([ ]* (G UE) )

# a new closure for descpreds

GUEA <- ([ ]* (G UEb a) )


# used to build tightly linked term lists.

JE <- ([ ]* (J e) )

JUE <- ([ ]* (J UE) )

# used to build subordinate clauses (argument modifiers).

JIZA <- ([ ]* ((J IE)/(J AE)/(P e)/(J i)/(J a)/(N u J i)) (Z a) )

JIOZA <- ([ ]* ((J IO)/(J AO)) (Z a) )

JIZI <- ([ ]* ((J IE)/(J AE)/(P e)/(J i)/(J a)/(N u J i)) (Z i) )

JIOZI <- ([ ]* ((J IO)/(J AO)) (Z i) )

JIZU <- ([ ]* ((J IE)/(J AE)/(P e)/(J i)/(J a)/(N u J i)) (Z u) )

JIOZU <- ([ ]* ((J IO)/(J AO)) (Z u) )

JI <- (!JIZA !JIZI !JIZU ([ ]* ((J IE)/(J AE)/(P e)/(J i)/(J a)/(N u J i)) ))

JIO <- (!JIOZA !JIOZI !JIOZU ([ ]* ((J IO)/(J AO)) ))

# case tags, both numerical position tags and the optional semantic case tags.

DIO <- ([ ]* ((B EU)/(C AU)/(D IO)/(F OA)/(K AO)/(J UI)/(N EU)/(P OU)/(G OA)/(S AU)/(V EU)/(Z UA)/(Z UE)/(Z UI)/(Z UO)/(Z UU)) ) (C i ![ ] NI0/ZI)?

# markers of indirect reference.  Originally these had the same grammar as case tags,
# but they are now different.

LAE <- ([ ]* ((L AE)/(L UE)) )

# <me> turns arguments into predicates, <meu> closes this construction.

ME <- ([ ]* ((M EA)/(M e)) )

MEU <- ([ ]* M EU )

# reflexive and conversion operators:  first the root forms, then those with
# optional numerical suffixes.

NU0 <- ((N UO)/(F UO)/(J UO)/(N u)/(F u)/(J u))

NU <- [ ]* (((N u/N UO) !([ ]+ (NI0/RA)) (NI0/RA)?)/NU0)+ freemod?

# abstract predicate constructors (from sentences)

# I do *not* think
# that <poia> will really be confused with <po ia>, particularly
# since we do require an explicit pause before <ia> in the latter case,
# but I record this concern:  the forms with z might be preferable.

PO1 <- ([ ]* ((P o)/(P u)/(Z o)))

PO1A <- ([ ]* ((P OIb a)/(P UIb a)/(Z OIb a)/(P o Z a)/(P u Z a)/(Z o Z a)))

PO1E <- ([ ]* ((P OIb e)/(P UIb e)/(Z OIb e)))

PO1I <- ([ ]* ((P OIb i)/(P UIb i)/(Z OIb i)/(P o Z i)/(P u Z i)/(Z o Z i)))

PO1O <- ([ ]* ((P OIb o)/(P UIb o)/(Z OIb o)))

PO1U <- ([ ]* ((P OIb u)/(P UIb u)/(Z OIb u)/(P o Z u)/(P u Z u)/(Z o Z u)))

# abstract predicate constructor from simple predicates

POSHORT1 <- ([ ]* ((P OI)/(P UI)/(Z OI)))

# word forms associated with the above abstract predicate root forms

PO <- ([ ]* PO1 )

POA <- ([ ]* PO1A )

POE <- ([ ]* PO1E )

POI <- ([ ]* PO1E )

POO <- ([ ]* PO1O )

POU <- ([ ]* PO1U )

POSHORT <- ([ ]* POSHORT1 )

# register markers 

DIE <- ([ ]* ((D IE)/(F IE)/(K AE)/(N UE)/(R IE)) )

# vocative forms:  I still have the words of social lubrication as 
# vocative markers.

HOI <- ([ ]* ((H OI)/(L OI)/(L OA)/(S IA)/(S IE)/(S IU)) )

# the verbal scare quote.  The quantifier suffix indicates how many preceding words are affected;
# this is an odd mechanism.

JO <- ([ ]* (NI0/RA/SA)? (J o) )

# markers for forming parenthetical utterances as free modifiers.

KIE <- ([ ]* (K IE) )

KIU <- ([ ]* (K IU) )

KIE2 <- [ ]* K IE comma2? [(]

KIU2 <- [ ]* [)] comma2? K IU

# marker for forming smilies.

SOI <- ([ ]* (S OI) )

# a grab bag of attitudinal words, including but not restricted to the VV forms.

UI0 <- (!predstart (!([Ii] juncture? [Ee]) VV juncture?/(B EA)/(B UO)/(C EA)/(C IA)/(C OA)/(D OU)/(F AE)/(F AO)/(F EU)/(G EA)/(K UO)/(K UU)/(R EA)/(N AO)/(N IE)/(P AE)/(P IU)/(S AA)/(S UI)/(T AA)/(T OE)/(V OI)/(Z OU)/((L OI))/((L OA))/((S IA))/(S II)/(T OE)/((S IU))/(C AO)/(C EU)/((S IE))/(S EU)/(S IEb i)))

# negative forms of the attitudinals.  The ones with <no> before the two vowel forms are a phonetic exception.  The others
# should also be (though they present no pronunciation problem) so that they are resolved as single words.

NOUI <- (([ ]* UI0 NOI)/([ ]* N [o] juncture? comma? [ ]* UI0 ))

# all attitudinals (adding the discursives nefi, tofi... etc)
# there is a technical problem with mixing UI0 roots of VV and CVV shapes.

UI1 <- ([ ]* (UI0+/(NI F i)) )

# the inverse vocative marker

HUE <- ([ ]* (H UE))

# occurrences of <no> as a word rather than an affix.

NO1 <- ([ ]* !KOU1 !NOUI (N o) !(comma2? Z AO comma2? Predicate) !([ ]* KOU) !([ ]* (JIO/JI/JIZA/JIOZA/JIZI/JIOZI/JIZU/JIOZU)) )

# a technical closure for the alternative parser approach:  the "large subject marker"

GAA <- (NO1 freemod?)* ([ ]* (G AA))


# Names, acronyms and PreNames from above.

AcronymicName <- Acronym &(comma/period/end)

DJAN <- (PreName/AcronymicName)

# predicate words which are phonetically cmapua

# "identity predicates".  Converses are provided as a new proposal.

BI <- ([ ]* (N u)? ((B IA)/(B IE)/(C IE)/(C IO)/(B IA)/(B [i])) )

# interrogative and pronoun predicates

LWPREDA <- ((H e)/(D UA)/(D UI)/(B UA)/(B UI))

# here I should reinstall the <zao> proposal.

# the predicate words defined above in the phonetics section

Predicate <- (CmapuaUnit comma2?  Z AO comma2?)* Complex (comma2? Z AO comma2? Predicate)?

# predicate words, other than the "identity predicates" of class [BI]
# these include the numerical predicates (NI RA), also cmapua phonetically.

# we are installing John Cowan's <zao> proposal here, experimentally, 4/15/2019

PREDA <- ([ ]* &caprule (Predicate/LWPREDA/(![ ] NI RA)) )

# Part 3:  The Grammar Proper

# right markers turned into classes.

guoa <- (PAUSE? (GUOA/GU) freemod?)

guoe <- (PAUSE? (GUOE/GU) freemod?)

guoi <- (PAUSE? (GUOI/GU) freemod?)

guoo <- (PAUSE? (GUOO/GU) freemod?)

guou <- (PAUSE? (GUOU/GU) freemod?)

guo <- (!guoa !guoe !guoi !guoo !guou (PAUSE? (GUO/GU) freemod?))

guiza <- (PAUSE? (GUIZA/GU) freemod?)

guizi <- (PAUSE? (GUIZI/GU) freemod?)

guizu <- (PAUSE? (GUIZU/GU) freemod?)

gui <- (PAUSE? (GUI/GU) freemod?)

gue <- (PAUSE? (GUE/GU) freemod?)

guea <- (PAUSE? (GUEA/GU) freemod?)

guu <- (PAUSE? (GUU/GU) freemod?)

guua <- (PAUSE? (GUUA/GU) freemod?)

giuo <- (PAUSE? (GIUO/GU) freemod?)

meu <- (PAUSE? (MEU/GU) freemod?)

geu <- GEU

# Here note the absence of pause/GU equivalence.

gap <- (PAUSE? GU freemod?)

# this is the vocative construction.  It can appear early because all of its components are marked.

# the intention is to indicate who is being addressed.  This can be handled via a name, a descriptive argument, a predicate or an
# alien text name (the last must be quoted).  The complexities of these grammatical constructions can be deferred until they are
# introduced.

# HOI0 <- [ ]* [Hh] [Oo] [Ii] juncture?  

# restore words of social lubrication as vocative markers but not as name markers:  <loi, Djan>  

# I do not allow a freemod to intervene between a vocative marker and the associated
# utterance, to avoid unintended grabbing of subjects by the words of social lubrication when they are used
# as vocative markers.  This lets <Loi, Djan> and <Loi hoi Djan> be equivalent.  The comma needed in the
# first because the social lubrication words are in this version not name markers.
       
HOI0 <- ([ ]* ((([Hh] OI)/([Ll] OI)/([Ll] OA)/([Ss] IA)/([Ss] IE)/([Ss] IU)))) juncture? !V1

voc <- (HOI0 comma2? name /(HOI comma2? descpred guea? namesuffix?)/(HOI comma2? argument1 guua?)/[ ]* &([Hh] [Oo] [Ii] juncture?) AlienWord)

# this is the inverse vocative.  It can appear early because all of its components are marked.

# the intention is to indicate who is speaking.  The range of ways this can be handled is similar to the range of ways it can be
# handled for the vocative;  there is the further option of a sentence (the [statement] class) and there is a strong closure option
# for the case where an argument is used (to avoid it inadvertantly expanding to a sentence).

HUE0 <- [ ]* &caprule [Hh] [Uu] juncture? [Ee] juncture? !V1

invvoc <- (HUE0 comma2? name/HUE freemod? descpred guea? namesuffix?/(HUE freemod? statement giuo?)/(HUE freemod? argument1 guu?)/[ ]* &([Hh] [Uu] juncture? [Ee] juncture?) AlienWord)


# this is the class of free modifiers.  Most of its components are head marked (those that aren't appear just above),
# and it is useful for it to appear early because these things appear everywhere in subsequent constructions.  A free modifier,
# of whatever sort, is a freely insertable gadget which modifies the immediately preceding construction, or the entire utterance
# if it is initial.

# NOUI is a negated attitudinal word.  UI1 is an attitudinal word:  these express an emotional attitude toward the 
# assertion (noting that EI marks questions (yes or no answer expected) and SEU marks utterances as answers).

# SOI creates smilies in a general sense:  <soi crano> indicates that the listener should imagine the speaker smiling;
# similarly for other predicates.

# DIE and NO DIE are register markers, communicating the social attitude of the speaker toward the one addressed:  <die> for
# example is "dear"

# KIE...KIU constructs a full parenthetical utterance as a comment, which can be enclosed in actual parentheses inside
# the marker words.

# JO is a scare quote device.

# the comma is a freemod with no semantic content:  this is a device for discarding phonetically required pauses
# and the speaker's optional pauses alike.   The pause before a non-pause marked prename is part of the NameWord and so
# is excluded.  Ellipses and dashes are fancy pauses supported as freemods.

freemod <- ((NOUI/(SOI freemod? descpred guea?)/DIE/(NO1 DIE)/(KIE comma? utterance0 comma? KIU)/(KIE2 comma? utterance0 comma? KIU2)/invvoc/voc/(comma !(!FalseMarked PreName))/JO/UI1/([ ]* '...' ([ ]* &letter)?)/([ ]* '--' ([ ]* &letter)?)) freemod?)

# the classes juelink to linkargs describe very tightly bound arguments which can be firmly attached to predicates in 
# the context of metaphorical modifications and the use of predicates in descriptive arguments.

# note that we allow predicate modifiers (prepositional phrases) to be bound with <je/jue> which is not
# allowed in 1989 Loglan, but which we believe is supported in Lojban.

juelink <- (JUE freemod? (term/(PA2 freemod? gap?)))

links1 <- (juelink (freemod? juelink)* gue?)

links <- ((links1/(KA freemod? links freemod? KI freemod? links1)) (freemod? A1 freemod? links1)*)

jelink <- (JE freemod? (term/(PA2 freemod? gap?)))

linkargs1 <- (jelink freemod? (links/gue)?)

linkargs <- ((linkargs1/(KA freemod? linkargs freemod? KI freemod? linkargs1)) (freemod? A1 freemod? linkargs1)*)

# class abstractpred supports the construction of event, property, and quantity predicates from sentences.  These are
# closable with <guo> if introduced with <po,pu,zo> and closable with suffixed variants of <guo> if introduced with suffixed
# variants of <po,pu,zo> (a NEW idea but it is clear that closure of these predicates (and of the more commonly
# used associated descriptions) is an important issue).

abstractpred <- ((POA freemod? uttAx guoa?)/(POA freemod? sentence guoa?)/(POE freemod? uttAx guoe?)/(POE freemod? sentence guoe?)/(POI freemod? uttAx guoi?)/(POI freemod? sentence guoi?)/(POO freemod? uttAx guoo?)/(POO freemod? sentence guoo?)/(POU freemod? uttAx guou?)/(POU freemod? sentence guou?)/(PO freemod? uttAx guo?)/(PO freemod? sentence guo?))

# predunit1 describes the truly atomic forms of predicate.

# PREDA is the class of predicate words (the phonetic predicate words along with the special phonetic cmapua which are predicates, listed
# above under the PREDA rule.  NU PREDA handles permutations and identifications of arguments of PREDAs.

# SUE contains the alien text constructions with <sao> and <sue>, semantically quite different but syntactically handled
# in the same way.

# <ge>...<geu/cue> (the closing optional) can parenthesize a fairly complex predicate phrase and turn it into an atomic form.  These
# forms can have conversion or reflexive operators (NU) applied.  I should look into why the class handled in the conversion case
# is different.  An important use of this is in metaphor constructions, but it has other potential uses.

# abstractpred is the class of abstraction predicates just introduced above.  These are treated as atomic in this grammar:  it should
# be noted that their privileges in the trial.85 grammar are (absurdly) limited.

# <me>...<meu> (the closing optional, but important to have available) forms predicates from arguments, the predicate being true of the
# objects to which the argument refers.  <Ti me le mrenu> :  this is one of the men we are talking about.

predunit1 <- ((SUE/(NU freemod? GE freemod? despredE (freemod? geu comma?)?)/(NU freemod? PREDA)/(comma? GE freemod? descpred (freemod? geu comma?)?)/abstractpred/(ME freemod? argument1 meu?)/PREDA) freemod?)

# <no> binds very tightly to predunit1:  a possibly multiply negated predunit1 (or an unadorned predunit1) is a predunit2.

predunit2 <- ((NO1 freemod?)* predunit1)

# an instance of NO2 is one not absorbed by a predunit.  Example:  <Da no kukra prano>  X is a slow (not-fast) runner vs
# <Da no ga kukra prano>  (X is not a fast runner, and in fact may not run at all).

NO2 <- (!predunit2 NO1)

# a predunit3 is a predunit2 with tightly attached arguments.

predunit3 <- ((predunit2 freemod? linkargs)/predunit2)

# a predunit is a predunit3 or a predunit3 converted by the short-scope abstraction operators
# <poi/pui/zoi> to an abstraction predicate.  This is the kind of predicate which can appear as
# a component in a serial name.

predunit <- ((POSHORT freemod?)? predunit3)

# a further "atomic" (because tightly packaged) form is a forethought connected pair
# of predicates (this being the full predicate class defined at the end of the process)
# possibly closed with <guu>, possibly multiply negated as well.

# the closure with guu eliminated the historic rule against kekked heads of metaphors.

kekpredunit <- ((NO1 freemod?)* KA freemod? predicate freemod? KI freemod? predicate guu?)

# there follows the construction of metaphorically modified predicates, 
# along with tightly logically linked predicates.

# CI and simple juxtaposition of predicates both represent modification of the second
# predicate by the first.  We impose no semantic conditions on this modification,
# except in the case of modification by predicates logically linked with CA,
# which do distribute logically in the expected way both as modifiers and as modified.
# We do not regard <preda1 preda2> as necessarily implying preda2:  we do regard
# it as having the same place structure as preda2.  It is very often but not always
# a qualification or kind of preda2;  in any case it is a relation analogous to preda2.

# modification with CI binds most tightly.

# we eliminated the distinction between the series of sentence and description
# predicate preliminary classes:  there seems to be no need for it even in the 
# trial.85 grammar.

despredA <- ((predunit/kekpredunit) (freemod? CI freemod? (predunit/kekpredunit))*)

# this is logical connection of predicates with the tightly binding CA
# series of logical connectives.  CUI can be used to expand the scope of
# a CA connective over a metaphor on the left.  <ge>...<geu> is used to expand
# scope on the right (and could also be used on the left, it should be noted).
# descpredC is an internal of despredB assisting the function of CUI.
# the !PREDA in front of CUI is probably not needed.

despredB <- ((!PREDA CUI freemod? despredC freemod? CA freemod? despredB)/despredA)

despredC <- (despredB (freemod? despredB)*)

# tight logical linkage of despredB's

despredD <- (despredB (freemod? CA freemod? despredB)*)

# chain of modifications of despredD's (grouping to the left)

despredE <- (despredD (freemod? despredD)*)

# the GO construction allows inverse modification:  <preda1 GO preda2> is <preda2 preda1> as it were.
#  there are profound effects on grouping.

descpred <- ((despredE freemod? GO freemod? descpred)/despredE)

# this version which appears in sentence predicates as opposed to descriptions differs
# in allowing loosely linked arguments (termsets) instead of those linked with <je/jue> for the predicate
# moved to the end by GO.

# 4/17/2019 shared argument experiment

# sentpred <- (( (KA freemod? sentpred freemod? KI freemod? sentpred !guu)/ (despredE freemod? GO freemod? barepred)/despredE) ) (A2 freemod? ((despredE freemod? GO freemod? barepred)/despredE))*

sentpred <- ((despredE freemod? GO freemod? barepred)/despredE)

# sentpred <- ((despredE freemod? GO freemod? barepred)/despredE) (A1 ((despredE freemod? GO freemod? barepred)/despredE))*

# the construction of predicate modifiers (prepositional phrases usable as terms along with arguments).

mod1a <- (PA3 freemod? argument1 guua?)

# note special treatment of predicate modifiers without actual arguments.
# the !barepred serves to distinguish these predicate modifiers from actual
# "tenses" (predicate markers).

mod1 <- ((PA3 freemod? argument1 guua?)/(PA2 freemod? !barepred gap?))

# forethought connection of modifiers.  There is some subtlety in
# how this is handled.

kekmod <- ((NO1 freemod?)* (KA freemod? modifier freemod? KI freemod? mod))

mod <- (mod1/((NO1 freemod?)* mod1)/kekmod)

# afterthought connection of modifiers

modifier <- (mod (A1 freemod? mod)*)

# the serial name is a horrid heterogenous construction!  It can involve
# components of all three of the major phonetic classes essentially!

# However, I believe I have the definition right, with all the components
# correctly guarded :-)

name <- (PreName/AcronymicName) (comma2? !FalseMarked PreName/comma2? &([Cc] [Ii]) NameWord/comma2? CI predunit !(comma2? (!FalseMarked PreName))/comma2? CI AcronymicName)* freemod? 

LA0 <- [ ]* [Ll] [Aa] juncture?  

LANAME <- (LA0 comma2? name)

# general constructions of arguments with "articles".

# the rules here have the "possessive" construction as in <lemi hasfa; le la Djan, hasfa> embedded.  These are not the same
# construction in 1989 Loglan, though speakers might think they are.  Here they are indeed the same.  The "possessor" cannot
# be "indefinite" (cannot start with a quantifier word);  the possessor can be followed by a tense, as in 
# <le la Djan, na hasfa>, "John's present house", by analogy with <lemina hasfa>, which is accepted by LIP (because
# LIP accepts <lemina> as a word).

# there are other subtleties to be reviewed.

#descriptn <- (!LANAME ((LAU wordset1)/(LOU wordset2)/(LE freemod? ((!mex arg1a freemod?)? (PA2 freemod?)?)? mex freemod? descpred)/(LE freemod? ((!mex arg1a freemod?)? (PA2 freemod?)?)? mex freemod? arg1a)/(GE freemod? mex freemod? descpred)/(LE freemod? ((!mex arg1a freemod?)? (PA2 freemod?)?)? descpred)))

descriptn <- (!LANAME ((LAU wordset1)/(LOU wordset2)/(LE freemod? ((!mex arg1a freemod?)? (PA2 freemod?)?)? (mex freemod? arg1a/mex freemod? descpred/descpred))/(GE freemod? mex freemod? descpred)))


# abstract descriptions.  Note that abstract descriptions are closed with <guo> entirely independently of abstract predicates:
# <le po preda guo> does not have a grammatical component <po preda guo>.  This avoids the double closure often apparently necessary
# in Lojban.

abstractn <- ((LEFORPO freemod? POA freemod? uttAx guoa?)/(LEFORPO freemod? POA freemod? sentence guoa?)/(LEFORPO freemod? POE freemod? uttAx guoe?)/(LEFORPO freemod? POE freemod? sentence guoe?)/(LEFORPO freemod? POI freemod? uttAx guoi?)/(LEFORPO freemod? POI freemod? sentence guoi?)/(LEFORPO freemod? POO freemod? uttAx guoo?)/(LEFORPO freemod? POO freemod? sentence guoo?)/(LEFORPO freemod? POU freemod? uttAx guou?)/(LEFORPO freemod? POU freemod? sentence guou?)/(LEFORPO freemod? PO freemod? uttAx guo?)/(LEFORPO freemod? PO freemod? sentence guo?))

# a wider class of basic argument constructions.  Notice that LANAME is always read by preference to descriptn.

namesuffix <- (&(comma2 !FalseMarked PreName/[ ]* [Cc][Ii] juncture? comma2? (PreName/AcronymicName)) ([ ]* [Cc][Ii] juncture? comma2?/comma2)? name)

arg1 <- (abstractn/(LIO freemod? descpred guea?)/(LIO freemod? argument1 guua?)/(LIO freemod? mex gap?)/LIO/LAO/LANAME/(descriptn guua? namesuffix?)/LIU1/LIE/LI)

# this adds pronouns (incl. the fancy <gao> letterals) and the option of left marking an argument with <ge>

arg1a <- ((DA/TAI/arg1/(GE freemod? arg1a)) freemod?)

# argument modifiers (subordinate clauses)

argmod1 <- ((([ ]* (N o) [ ]*)? ((JI freemod? predicate)/(JIO freemod? sentence)/(JIO freemod? uttAx)/(JI freemod? modifier)/(JI freemod? argument1)))/(([ ]* (N o) [ ]*)? (((JIZA freemod? predicate) guiza?)/((JIOZA freemod? sentence) guiza?)/((JIOZA freemod? uttAx) guiza?)/((JIZA freemod? modifier) guiza?)/(JIZA freemod? argument1 guiza?)))/(([ ]* (N o) [ ]*)? ((JIZI freemod? predicate guizi?)/(JIOZI freemod? sentence guizi?)/(JIOZI freemod? uttAx guizi?)/(JIZI freemod? modifier guizi?)/(JIZI freemod? argument1 guizi?)))/(([ ]* (N o) [ ]*)? ((JIZU freemod? predicate guizu?)/(JIOZU freemod? sentence guizu?)/(JIOZU freemod? uttAx guizu?)/(JIZU freemod? modifier guizu?)/(JIZU freemod? argument1 guizu?))))

# we improved the trial.85 grammar by closing not argmod1 but argmod with <gui>.  But the labelled argument modifier constructors
# when building an argmod1 have the argmod1 construction closed with the corresponding labelled right marker, of course.  Thus
# gui and guiza actually have different grammar.

# trial.85 did not provide forethought connected argument modifiers, and we also see no need for them,
# though they could readily be added.

argmod <- (argmod1 (A1 freemod? argmod1)* gui?)

# affix argument modifiers to a definite argument

arg2 <- (arg1a freemod? argmod*)

# build a possibly indefinite argument from an argument:  to le mrenu

arg3 <- (arg2/(mex freemod? arg2))

# build an indefinite argument from a predicate

indef1 <- (mex freemod? descpred)

# affix an argument modifier to an indefinite argument

indef2 <- (indef1 guua? argmod*)

indefinite <- indef2

# link arguments with the fusion connective <ze>

arg4 <- ((arg3/indefinite) (ZE2 freemod? (arg3/indefinite))*)

# forethought connection of arguments.  Note use of argx

arg5 <- (arg4/(KA freemod? argument1 freemod? KI freemod? argx))

# arguments with possible negations followed by possible indirect reference constructions.

argx <- ((NO1 freemod?)* (LAE freemod?)* arg5)

# afterthought connection with the tightly binding ACI connectives

arg7 <- (argx freemod? (ACI freemod? argx)?)

# afterthought connection with the usual A connectives.  Can't start with GE
# to avoid an ambiguity (to which 1989 Loglan is vulnerable) involving AGE connectives.

arg8 <- (!GE (arg7 freemod? (A1 freemod? arg7)*))

# afterthought connection (now right grouping, instead of the left grouping above)
# using the AGE connectives.  GUU can be used to affix an argument modifier at this top level.

argument1 <- (((arg8 freemod? AGE freemod? argument1)/arg8) (GUU freemod? argmod)*)

# possibly negated and case tagged arguments.  We (unlike 1989 Loglan) are careful
# to use argument only where case tags are appropriate.

argument <- ((NO1 freemod?)* (DIO freemod?)* argument1)






# an argument which is actually case tagged.

argxx <- (&((NO1 freemod?)* DIO) argument)

# arguments and predicate modifiers actually associated with predicates.

term <- (argument/modifier)

# a term list consisting entirely of modifiers.

modifiers <- (modifier (freemod? modifier)*)

# a term list consisting entirely of modifiers and tagged arguments.

modifiersx <- ((modifier/argxx) (freemod? (modifier/argxx))*)

# the subject class is a list of terms (arguments and predicate modifiers) in which all but possibly one
# of the arguments are tagged, and there is at least one argument, tagged or otherwise.

subject <- ((modifiers freemod?)? ((argxx subject)/(argument (modifiersx freemod?)?)))

# this case is identified as an aid to experimental termination of argument lists

statement1 <- (subject freemod? (GIO freemod? terms1)? predicate)

# these classes are exactly argument, but are used to signal
# which argument position after the predicate an argument occupies.
# I think the grammar is set up so that these will actually
# never be case tagged, though the grammar does not expressly forbid it.

# I am trying a simple version of the "alternative parser" approach:
# a term list will refuse to digest an argument which starts a new
# SVO sentence (statement1).

argumentA <- !statement1 argument 

# argumentA <- argument

argumentB <- !statement1 argument 

#  argumentB <- argument

argumentC <- !statement1 argument 

#  argumentC <- argument

argumentD <- !statement1 argument 

#  argumentD <- argument

# for argument lists not guarded against absorbing a following subject

argumentA1 <- argument

argumentB1 <- argument

argumentC1 <- argument

argumentD1 <- argument

# a general term list.  It cannot contain more than four untagged arguments (they will be labelled
# with the lettered subclasses given above).

terms <- ((modifiersx? argumentA (freemod? modifiersx)? argumentB? (freemod? modifiersx)? argumentC? (freemod? modifiersx)? argumentD?)/modifiersx)

# terms list not guarded against absorbing a following subject

terms1 <- ((modifiersx? argumentA1 (freemod? modifiersx)? argumentB1? (freemod? modifiersx)? argumentC1? (freemod? modifiersx)? argumentD1?)/modifiersx)

# innards of ordered and unordered list constructions.  These are something I totally rebuilt, as they were in a totally
# unsatisfactory state in trial.85.  Note the use of comma words to separate items in lists.

word <- (arg1a/indef2)

words1 <- (word (ZEIA word)*)

words2 <- (word (ZEIO word)*)

wordset1 <- (words1? LUA)

wordset2 <- (words2? LUO)

# the full term set type to be affixed to predicates.

# forethought connection of term lists

termset1 <- (terms/(KA freemod? termset2 freemod? guu? KI freemod? termset1))

# afterthought connection of term lists.  There are cunning things going on here getting <guu>
# to work correctly.  Note that <guu> is NOT a null term list as it was in trial.85.

termset2 <- (termset1 (guu &A1)? (A1 freemod? termset1 (guu &A1)?)*)

# there is an interesting option here of a list of terms followed by <go> followed by a predicate
# intended to metaphorically modify the predicate to which the terms are affixed.  Is there a reason
# why we cannot have a more complex construction in place of terms?

termset <- ((terms freemod? GO freemod? barepred)/termset2)

# this is the untensed predicate with arguments attached.  Here is the principal locus
# of closure with <guu>, but it is deceptive to say that <guu> merely closes barepred,
# as we have seen above, for example in [termset2].

# modified for 4/17/2019 shared argument experiment

barepred <- (sentpred freemod? ((termset guu?)/(guu (&termset)))?)

# barepred <- (sentpred freemod? ((termset guu?)/(guu (&termset/&A1)))?)

# tensed predicates

markpred <- (PA1 freemod? barepred)

# there follows an area in which my grammar looks different from trial.85.  Distinct parallel forms for
# marked and unmarked predicates are demonstrably not needed even in trial.85.  The behavior of the ACI
# connectives is plain weird in trial.85; here we treat ACI connectives in the same way as A connectives, but
# binding more tightly.

# units for the ACI construction following -- possibly multiply negated bare or marked predicates.

# adding shared termsets to logically connected predicates are handled differently here than in trial.85,
# which uses a very elegant but dreadfully left-grouping rule which a PEG cannot handle.  Any realistic situation
# should be manageable.

backpred1 <- ((NO2 freemod?)* (barepred/markpred))

# ACI connected predicates.  Shared termsets are added.  Notice how we first group backpred1's then recursively
# group backpreds.

backpred <- (((backpred1 (ACI freemod? backpred1)+ freemod? ((termset guu?)/(guu &termset))?) ((ACI freemod? backpred)+ freemod? ((termset guu?)/(guu &termset))?)?)/backpred1)

# A connected predicates; same comments as just above.  Cannot start with GE to fix ambiguity with AGE connectives.

predicate2 <- (!GE (((backpred (A1 !GE freemod? backpred)+ freemod? ((termset guu?)/(guu &termset))?) ((A1 freemod? predicate2)+ freemod? ((termset guu?)/(guu &termset))?)?)/backpred))

# predicate2's linked with right grouping AGE connectives (A and ACI are left grouping).

predicate1 <- ((predicate2 AGE freemod? predicate1)/predicate2)

# identity predicates from above, possibly negated

identpred <- ((NO1 freemod?)* (BI freemod? argument1 guu?))

# predicates in general.  Note that identity predicates cannot be logically connected
# except by using forethought connection (see above).

predicate <- (predicate1/identpred)


# The gasent is a basic form of the Loglan sentence in which the predicate leads.
# The basic structure is <PA word (usually a tense) or <ga>) followed optionally by terms followed optionally by
# <ga> followed by terms.  The list of terms after <ga> (if present) will either contain 
# at least one argument and no more than one untagged argument
# (a subject) [gasent1] or all the arguments of the predicate [gasent2].  We deprecate other arrangements possible in
# 1989 Loglan because they would cause unexpected reorientation of the arguments already given before <ga> as second
# and further arguments were read after <ga>.  [barepred] is an untensed predicate possibly with arguments; [sentpred]
# is "simply a verb", i.e., a predicate without arguments.

# there is a semantic change from 1989 Loglan reflected in a grammar change here:
# in [gasent1] the final (ga subject) is optional.  When it does not appear, the resulting
# sentence is an observative (a sentence with subject omitted), not an imperative.
# Imperatives for us are unmarked.

# In the alternative version, the use of the large subject marker GAA can prevent inadvertant absorption of a preceding trailing argument into a statement

# 4/22 allowing general predicates in gasent.  Otherwise the spaces of observatives and imperatives become quite confused.

#gasent1 <- ((NO1 freemod?)* (GAA? freemod? PA1 freemod? barepred (GA2 freemod? subject)?))

gasent1 <- ((NO1 freemod?)* (GAA? freemod? &markpred predicate (GA2 freemod? subject)?))

gasent2 <- ((NO1 freemod?)* (GAA? PA1 freemod? sentpred modifiers? (GA2 freemod? subject freemod? GIO? freemod? terms?)))

gasent <- (gasent2/gasent1)

# this is the simple Loglan sentence in various basic orders.  The form "gasent" is discoussed just above.  
# Predicate modifiers
# can be prefixed to the gasent.  The final form given here is the basic SVO sentence.  The "subject" class is a list of terms
#(arguments and predicate modifiers) containing at most one un-case-tagged argument.  The most general SVO form is subject, followed optionally
#by <gio> followed by a list of terms (1989 Loglan allowed more than one untagged argument before the predicate, but this leads to practical problems
#in which preceding constructions with errors in them may supply extra unintended arguments.  It should be noted in NB3 that JCB envisioned
#a single argument before the predicate, followed by the predicate, which may itself contain further arguments.  A gasent nay optionally be negated
#(even multiple times).

# re <gio> and some other changes, in his comments on the NB3 grammar  JCB often notes restrictions on appearances of term lists which he
# intends but which he thought were hard to implement in the machine grammar.  The appearance of just one argument before the "verb"
# in an SVO sentence was one of these (though later he takes it as a virtue that the actual machine grammar supports SOV:  we did not
# consider it a virtue to have unmarked SOV after observing unintended parses appearing in the Visit text).  Another example of this
# (which would not have been hard for JCB to implement, in fact) is our restriction of the form "terms gasent" to "modifiers gasent".
# His comments make it clear that he does not want arguments among those terms.

statement <- (gasent/(modifiers freemod? gasent)/(subject freemod? GAA? freemod? (GIO freemod? terms1)? predicate))

# this is a forethought connected basic sentence.  It is odd (and actual odd results can be exhibited) that the final segment in both
# of these rules is of the very general class uttA1, which includes some quite fragmentary utterances usually intended as answers.

# 12/20/2017 I rewrote the rule in a more compact form.  This rule looks ahead to the class [sentence] which we now develop;
# for the moment notice that [sentence] will include [statement].

# 4/14 tentatively allowing initial modifiers here and leaving this out of uttA0 which replaces uttA1 below.
# The intention is to eliminate weird sentence fragments.

keksent <- modifiers? freemod? (NO1 freemod?)* (KA freemod? headterms? freemod? sentence freemod? KI freemod? uttA0)

# sentence negation.  We allow this to be set off from the main sentence with a mere pause, because generally
# it does not differ in meaning from the result of negating the first argument or predicate modifier.

neghead <- ((NO1 freemod? gap)/(NO2 PAUSE))

# this class includes [statement], predicate modifiers preceding a predicate (which may contain arguments), a statement,
# a predicate, and a keksent.  Of these, the first and third are imperatives.

# in the alternative version, the large subject marker GAA can prevent inadvertant absorption of preceding trailing arguments into a statement

# 4/23/2019 added actual rule for imperative sentences.  This should not
# affect the parse in any essential way.

imperative <- ((modifiers freemod?)? GAA? !gasent predicate)

sen1 <- (neghead freemod?)* (imperative/statement/keksent)

# sen1 <- ((neghead freemod?)* ((modifiers freemod? GAA? !gasent predicate)/statement/GAA? predicate/keksent))

# the class [sentence] consists of sen1's afterthought connected with A connectives

sentence <- (sen1 (ICA freemod? sen1)*)

# [headterms] is a list of terms (arguments and predicate modifiers) ending in <gi>.  Preceding a [sen1] with these
# causes all predicates in the [sen1] to share these arguments.  We propose either that the headterms arguments be directly
# appended to the argument list of each component of the [sen1], or that there is an argument with a numbered case tag at the beginning
# of the headterms list, and the list is inserted at the appropriate position in each component sentence.  Neither of these is
# the condition described in Loglan I, which presupposes that we always know what the last argument of each predicate used is.

headterms <- (terms GI)+

# this is the previous class prefixed with a list of fronted terms.
# we think the <giuo> closure might prove useful.

uttAx <- (headterms freemod? sentence giuo?)

# weird answer fragments

uttA <- ((A1/mex) freemod?)

# a broad class of utterances, including various things one would usually only say as answers.  Notice
# that this utterance class can take terminal punctuation.

uttA0 <- sen1/uttAx

uttA1 <- ((sen1/uttAx/links/linkargs/argmod/(modifiers freemod? keksent)/terms/uttA/NO1) freemod? period?)

# possibly negated utterances of the previous class.

uttC <- ((neghead freemod? uttC)/uttA1)

# utterances linked with more tightly binding ICI sentence connectives.  Single sentences are of this class
# if not linked with ICI or ICA.

uttD <- ((sentence period? !ICI !ICA)/(uttC (ICI freemod? uttD)*))

# utterances of the previous class linked with ICA.  I went to some trouble to ensure that a freestanding
# [sentence] is actually parsed as a sentence, not a composite uttD, which was a deficiency, if not an ambiguity of
# LIP and of the trial.85 grammar.

uttE <- (uttD (ICA freemod? uttD)*)

# utterances of the previous class linked with I sentence connectives.

uttF <- (uttE (I freemod? uttE)*)

# the utterance class for use in the context of parenthetical freemods or quotations, in which it does not go to end of text.

utterance0 <- (!GE ((!PAUSE freemod period? utterance0)/(!PAUSE freemod period?)/(uttF IGE utterance0)/uttF/(I freemod? uttF?)/(I freemod? period?)/(ICA freemod? uttF)) (&I utterance0)?)

# Notice that there are two passes here:  the parser first checks that the entire utterance
# is phonetically valid, then returns and checks for grammatical validity.

# the full utterance class.  This goes to end of text, and incorporates the phonetics check.  This incorporates the only situations
# in which a freemod is initial.   The IGE connectives bind even more loosely than the I connectives and right-group instead of 
# left grouping.

utterance <- &(phoneticutterance !.) (!GE ((!PAUSE freemod period? utterance)/(!PAUSE freemod period? (&I utterance)? end)/(uttF IGE utterance)/(I freemod? period? (&I utterance)? end)/(uttF (&I utterance)? end)/(I freemod? uttF (&I utterance)? end)/(ICA freemod? uttF (&I utterance)? end)))




\end{verbatim}


\end{document}