\documentclass{article}
%test
% Dear Randall,

% I use Paul Taylor's macros for drawing proof trees. I enclose this in a 
% following mail. For sequents I use the macro:

\def\joinrelm{\mathrel{\mkern-7mu}}
\def\relbar{\mathrel{\smash-}}
\def\tailpiece{\rule{.02in}{5pt}}
\def\ltstile{\mathrel{\tailpiece\joinrelm\relbar}}

\newcommand{\ljudge}[2]{{#1}\: \ltstile   {#2}}



% Thus $\ljudge{\Gamma}{A}$ (ljudge= Logical JUDGEment) produces the appropriate 
% thing. The little squashed turnstile is a Girard-ism - indeed many people call 
% it a Girardian turnstile! The point is that it is different from $\vdash$, 
% which can then be used to denote provability.

% I hope all this works - get in touch if you have any problems.

% Cheers,
% Gavin.

\message{<Paul Taylor's Proof Trees, 2 August 1996>}
%% Build proof tree for Natural Deduction, Sequent Calculus, etc.
%% WITH SHORTENING OF PROOF RULES!
%% Paul Taylor, begun 10 Oct 1989
%% *** THIS IS ONLY A PRELIMINARY VERSION AND THINGS MAY CHANGE! ***
%%
%% 2 Aug 1996: fixed \mscount and \proofdotnumber
%%
%%      \prooftree
%%              hyp1            produces:
%%              hyp2
%%              hyp3            hyp1    hyp2    hyp3
%%      \justifies              -------------------- rulename
%%              concl                   concl
%%      \thickness=0.08em
%%      \shiftright 2em
%%      \using
%%              rulename
%%      \endprooftree
%%
%% where the hypotheses may be similar structures or just formulae.
%%
%% To get a vertical string of dots instead of the proof rule, do
%%
%%      \prooftree                      which produces:
%%              [hyp]
%%      \using                                  [hyp]
%%              name                              .
%%      \proofdotseparation=1.2ex                 .name
%%      \proofdotnumber=4                         .
%%      \leadsto                                  .
%%              concl                           concl
%%      \endprooftree
%%
%% Within a prooftree, \[ and \] may be used instead of \prooftree and
%% \endprooftree; this is not permitted at the outer level because it
%% conflicts with LaTeX. Also,
%%      \Justifies
%% produces a double line. In LaTeX you can use \begin{prooftree} and
%% \end{prootree} at the outer level (however this will not work for the inner
%% levels, but in any case why would you want to be so verbose?).
%%
%% All of of the keywords except \prooftree and \endprooftree are optional
%% and may appear in any order. They may also be combined in \newcommand's
%% eg "\def\Cut{\using\sf cut\thickness.08em\justifies}" with the abbreviation
%% "\prooftree hyp1 hyp2 \Cut \concl \endprooftree". This is recommended and
%% some standard abbreviations will be found at the end of this file.
%%
%% \thickness specifies the breadth of the rule in any units, although
%% font-relative units such as "ex" or "em" are preferable.
%% It may optionally be followed by "=".
%% \proofrulebreadth=.08em or \setlength\proofrulebreadth{.08em} may also be
%% used either in place of \thickness or globally; the default is 0.04em.
%% \proofdotseparation and \proofdotnumber control the size of the
%% string of dots
%%
%% If proof trees and formulae are mixed, some explicit spacing is needed,
%% but don't put anything to the left of the left-most (or the right of
%% the right-most) hypothesis, or put it in braces, because this will cause
%% the indentation to be lost.
%%
%% By default the conclusion is centered wrt the left-most and right-most
%% immediate hypotheses (not their proofs); \shiftright or \shiftleft moves
%% it relative to this position. (Not sure about this specification or how
%% it should affect spreading of proof tree.)
%
% global assignments to dimensions seem to have the effect of stretching
% diagrams horizontally.
%
%%==========================================================================

\def\introrule{{\cal I}}\def\elimrule{{\cal E}}%%
\def\andintro{\using{\land}\introrule\justifies}%%
\def\impelim{\using{\Rightarrow}\elimrule\justifies}%%
\def\allintro{\using{\forall}\introrule\justifies}%%
\def\allelim{\using{\forall}\elimrule\justifies}%%
\def\falseelim{\using{\bot}\elimrule\justifies}%%
\def\existsintro{\using{\exists}\introrule\justifies}%%

%% #1 is meant to be 1 or 2 for the first or second formula
\def\andelim#1{\using{\land}#1\elimrule\justifies}%%
\def\orintro#1{\using{\lor}#1\introrule\justifies}%%

%% #1 is meant to be a label corresponding to the discharged hypothesis/es
\def\impintro#1{\using{\Rightarrow}\introrule_{#1}\justifies}%%
\def\orelim#1{\using{\lor}\elimrule_{#1}\justifies}%%
\def\existselim#1{\using{\exists}\elimrule_{#1}\justifies}

%%==========================================================================

\newdimen\proofrulebreadth \proofrulebreadth=.05em
\newdimen\proofdotseparation \proofdotseparation=1.25ex
\newdimen\proofrulebaseline \proofrulebaseline=2ex
\newcount\proofdotnumber \proofdotnumber=3
\let\then\relax
\def\hfi{\hskip0pt plus.0001fil}
\mathchardef\squigto="3A3B
%
% flag where we are
\newif\ifinsideprooftree\insideprooftreefalse
\newif\ifonleftofproofrule\onleftofproofrulefalse
\newif\ifproofdots\proofdotsfalse
\newif\ifdoubleproof\doubleprooffalse
\let\wereinproofbit\relax
%
% dimensions and boxes of bits
\newdimen\shortenproofleft
\newdimen\shortenproofright
\newdimen\proofbelowshift
\newbox\proofabove
\newbox\proofbelow
\newbox\proofrulename
%
% miscellaneous commands for setting values
\def\shiftproofbelow{\let\next\relax\afterassignment\setshiftproofbelow\dimen0 
}
\def\shiftproofbelowneg{\def\next{\multiply\dimen0 by-1 }%
\afterassignment\setshiftproofbelow\dimen0 }
\def\setshiftproofbelow{\next\proofbelowshift=\dimen0 }
\def\setproofrulebreadth{\proofrulebreadth}

%=============================================================================
\def\prooftree{% NESTED ZERO (\ifonleftofproofrule)
%
% first find out whether we're at the left-hand end of a proof rule
\ifnum  \lastpenalty=1
\then   \unpenalty
\else   \onleftofproofrulefalse
\fi
%
% some space on left (except if we're on left, and no infinity for outermost)
\ifonleftofproofrule
\else   \ifinsideprooftree
        \then   \hskip.5em plus1fil
        \fi
\fi
%
% begin our proof tree environment
\bgroup% NESTED ONE (\proofbelow, \proofrulename, \proofabove,
%               \shortenproofleft, \shortenproofright, \proofrulebreadth)
\setbox\proofbelow=\hbox{}\setbox\proofrulename=\hbox{}%
\let\justifies\proofover\let\leadsto\proofoverdots\let\Justifies\proofoverdbl
\let\using\proofusing\let\[\prooftree
\ifinsideprooftree\let\]\endprooftree\fi
\proofdotsfalse\doubleprooffalse
\let\thickness\setproofrulebreadth
\let\shiftright\shiftproofbelow \let\shift\shiftproofbelow
\let\shiftleft\shiftproofbelowneg
\let\ifwasinsideprooftree\ifinsideprooftree
\insideprooftreetrue
%
% now begin to set the top of the rule (definitions local to it)
\setbox\proofabove=\hbox\bgroup$\displaystyle % NESTED TWO
\let\wereinproofbit\prooftree
%
% these local variables will be copied out:
\shortenproofleft=0pt \shortenproofright=0pt \proofbelowshift=0pt
%
% flags to enable inner proof tree to detect if on left:
\onleftofproofruletrue\penalty1
}

%=============================================================================
% end whatever box and copy crucial values out of it
\def\eproofbit{% NESTED TWO
%
% various hacks applicable to hypothesis list 
\ifx    \wereinproofbit\prooftree
\then   \ifcase \lastpenalty
        \then   \shortenproofright=0pt  % 0: some other object, no indentation
        \or     \unpenalty\hfil         % 1: empty hypotheses, just glue
        \or     \unpenalty\unskip       % 2: just had a tree, remove glue
        \else   \shortenproofright=0pt  % eh?
        \fi
\fi
%
% pass out crucial values from scope
\global\dimen0=\shortenproofleft
\global\dimen1=\shortenproofright
\global\dimen2=\proofrulebreadth
\global\dimen3=\proofbelowshift
\global\dimen4=\proofdotseparation
\global\count255=\proofdotnumber
%
% end the box
$\egroup  % NESTED ONE
%
% restore the values
\shortenproofleft=\dimen0
\shortenproofright=\dimen1
\proofrulebreadth=\dimen2
\proofbelowshift=\dimen3
\proofdotseparation=\dimen4
\proofdotnumber=\count255
}

%=============================================================================
\def\proofover{% NESTED TWO
\eproofbit % NESTED ONE
\setbox\proofbelow=\hbox\bgroup % NESTED TWO
\let\wereinproofbit\proofover
$\displaystyle
}%
%
%=============================================================================
\def\proofoverdbl{% NESTED TWO
\eproofbit % NESTED ONE
\doubleprooftrue
\setbox\proofbelow=\hbox\bgroup % NESTED TWO
\let\wereinproofbit\proofoverdbl
$\displaystyle
}%
%
%=============================================================================
\def\proofoverdots{% NESTED TWO
\eproofbit % NESTED ONE
\proofdotstrue
\setbox\proofbelow=\hbox\bgroup % NESTED TWO
\let\wereinproofbit\proofoverdots
$\displaystyle
}%
%
%=============================================================================
\def\proofusing{% NESTED TWO
\eproofbit % NESTED ONE
\setbox\proofrulename=\hbox\bgroup % NESTED TWO
\let\wereinproofbit\proofusing
\kern0.3em$
}

%=============================================================================
\def\endprooftree{% NESTED TWO
\eproofbit % NESTED ONE
% \dimen0 =     length of proof rule
% \dimen1 =     indentation of conclusion wrt rule
% \dimen2 =     new \shortenproofleft, ie indentation of conclusion
% \dimen3 =     new \shortenproofright, ie
%                space on right of conclusion to end of tree
% \dimen4 =     space on right of conclusion below rule
  \dimen5 =0pt% spread of hypotheses
% \dimen6, \dimen7 = height & depth of rule
%
% length of rule needed by proof above
\dimen0=\wd\proofabove \advance\dimen0-\shortenproofleft
\advance\dimen0-\shortenproofright
%
% amount of spare space below
\dimen1=.5\dimen0 \advance\dimen1-.5\wd\proofbelow
\dimen4=\dimen1
\advance\dimen1\proofbelowshift \advance\dimen4-\proofbelowshift
%
% conclusion sticks out to left of immediate hypotheses
\ifdim  \dimen1<0pt
\then   \advance\shortenproofleft\dimen1
        \advance\dimen0-\dimen1
        \dimen1=0pt
%       now it sticks out to left of tree!
        \ifdim  \shortenproofleft<0pt
        \then   \setbox\proofabove=\hbox{%
                        \kern-\shortenproofleft\unhbox\proofabove}%
                \shortenproofleft=0pt
        \fi
\fi
%
% and to the right
\ifdim  \dimen4<0pt
\then   \advance\shortenproofright\dimen4
        \advance\dimen0-\dimen4
        \dimen4=0pt
\fi
%
% make sure enough space for label
\ifdim  \shortenproofright<\wd\proofrulename
\then   \shortenproofright=\wd\proofrulename
\fi
%
% calculate new indentations
\dimen2=\shortenproofleft \advance\dimen2 by\dimen1
\dimen3=\shortenproofright\advance\dimen3 by\dimen4
%
% make the rule or dots, with name attached
\ifproofdots
\then
        \dimen6=\shortenproofleft \advance\dimen6 .5\dimen0
        \setbox1=\vbox to\proofdotseparation{\vss\hbox{$\cdot$}\vss}%
        \setbox0=\hbox{%
                \advance\dimen6-.5\wd1
                \kern\dimen6
                $\vcenter to\proofdotnumber\proofdotseparation
                        {\leaders\box1\vfill}$%
                \unhbox\proofrulename}%
\else   \dimen6=\fontdimen22\the\textfont2 % height of maths axis
        \dimen7=\dimen6
        \advance\dimen6by.5\proofrulebreadth
        \advance\dimen7by-.5\proofrulebreadth
        \setbox0=\hbox{%
                \kern\shortenproofleft
                \ifdoubleproof
                \then   \hbox to\dimen0{%
                        $\mathsurround0pt\mathord=\mkern-6mu%
                        \cleaders\hbox{$\mkern-2mu=\mkern-2mu$}\hfill
                        \mkern-6mu\mathord=$}%
                \else   \vrule height\dimen6 depth-\dimen7 width\dimen0
                \fi
                \unhbox\proofrulename}%
        \ht0=\dimen6 \dp0=-\dimen7
\fi
%
% set up to centre outermost tree only
\let\doll\relax
\ifwasinsideprooftree
\then   \let\VBOX\vbox
\else   \ifmmode\else$\let\doll=$\fi
        \let\VBOX\vcenter
\fi
% this \vbox or \vcenter is the actual output:
\VBOX   {\baselineskip\proofrulebaseline \lineskip.2ex
        \expandafter\lineskiplimit\ifproofdots0ex\else-0.6ex\fi
        \hbox   spread\dimen5   {\hfi\unhbox\proofabove\hfi}%
        \hbox{\box0}%
        \hbox   {\kern\dimen2 \box\proofbelow}}\doll%
%
% pass new indentations out of scope
\global\dimen2=\dimen2
\global\dimen3=\dimen3
\egroup % NESTED ZERO
\ifonleftofproofrule
\then   \shortenproofleft=\dimen2
\fi
\shortenproofright=\dimen3
%
% some space on right and flag we've just made a tree
\onleftofproofrulefalse
\ifinsideprooftree
\then   \hskip.5em plus 1fil \penalty2
\fi
}

%==========================================================================
% IDEAS
% 1.    Specification of \shiftright and how to spread trees.
% 2.    Spacing command \m which causes 1em+1fil spacing, over-riding
%       exisiting space on sides of trees and not affecting the
%       detection of being on the left or right.
% 3.    Hack using \@currenvir to detect LaTeX environment; have to
%       use \aftergroup to pass \shortenproofleft/right out.
% 4.    (Pie in the sky) detect how much trees can be "tucked in"
% 5.    Discharged hypotheses (diagonal lines).





\title{Quine's Calculus of Concepts}

\author{M. Randall Holmes}

\begin{document}

\maketitle

\section{Introduction}

The calculus of concepts of Quine is described; its semantics and
formal rules of inference are given.

\section{The Natural Semantics of {\em CC\/}}

We follow Quine's example in first presenting {\em CC\/} via the
originally intended interpretation.  It will turn out that the system
has other quite natural interpretations as well, as well as some
rather strange ones.

We begin by describing the language of {\em CC\/}.

Atomic terms of the language of {\em CC\/} are taken from a countable
supply of free variables.  Atomic constants will be introduced as
well.  Atomic terms of {\em CC\/} are terms of {\em CC\/}; if $T$ and
$U$ are terms, then $T^c$, $T\times U$, $T/U$, and $\delta T$ are
terms.  Other operations on terms may be introduced by definition.
All terms are built from atomic terms by the primitive operations and
operations defined in terms of the primitive operations below.  Unary
operations are always considered to have higher precedence than binary
operations.  Both product and quotient group to the right, and
quotient has higher precedence than product.

We now present the intended interpretation of the notion of
``concept''.
\begin{description} \item[Definition:] If ${\cal D}$ is our universe
of discourse and $n \in {\cal N}$, we define a {\em concept of degree
$n$\/} as a pair $(S,n)$ with $S \subseteq {\cal D}^n$.  (The pairing
with $n$ is needed because empty concepts of different degrees need to
be distinct.)  We identify 1-tuples of elements of ${\cal D}$ with
elements of ${\cal D\/}$ itself, and we postulate a unique 0-tuple, so
concepts of degrees 0 and 1 make sense. \end{description}

Any formula in the language of first-order logic with variables
indexed by the positive integers can be associated in a natural way
with a concept (actually, with a concept of each sufficiently large
degree):

\begin{description}

\item[Definition:] A concept $(A,n)$ is said to represent a formula
$\phi$ just in case $\phi$ has no free variable other than variables
$x_i$ for $i \leq n$ and $\{(x_1,\ldots,x_n) \mid \phi\} = A$.  Notice
that concepts of degree 0 represent sentences with no free variables
(the two concepts of degree 0 represent the truth values).

\end{description}

Each term of the language of {\em CC\/} is intended to represent a
concept.  Note that there is no explicit indication of numerical
degree in {\em CC\/}; the explicit indications of degree are a feature
of the interpretation being given.

We now introduce the operations of CC (in their intended interpretation)

We first define $x.y$, for $x \in {\cal D}^m, y \in {\cal D}^n$, as
the element of ${\cal D}^{m+n}$ obtained by concatenating $x$ and $y$.

\begin{description}

\item[complement:] The {\em complement\/} of a concept $(A,n)$,
written $(A,n)^c$ is $$({\cal D}^n-A,n).$$ (Quine would write $-(A,n)$).

\item[product:] The {\em product\/} of concepts $(A,m)$ and $(B,n)$,
written $(A,m)\times (B,n)$ is $$(\{x.y\mid x\in A \wedge y \in
B\},m+n).$$ Quine calls this the ``Cartesian product'', but this is
somewhat misleading, since the underlying operation is concatenation
rather than pairing; an important difference is that this operation is
associative and Cartesian products properly so-called are not.

\item[quotient:] The {\em quotient\/} of a concept $(A,m)$ by a
concept $(B,n)$, written $(A,m)/(B,n)$, is $$(\{x \mid \exists y \in
B. (x.y \in A)\},\max(m-n,0)\}).$$ Notice that the subtraction of
degrees natural in this theory is a subtraction on natural numbers.
Quine called this the ``image'' operation and used the double-quote
notation for image to represent it.  The definition of quotient when
the degree of $B$ exceeds the degree of $A$ is a ``don't care'' case;
under the definition given here such a concept is empty, but this will
not be provable in the formalization we present.

\item[diagonalization:] The {\em diagonalization\/} of a concept
$(A,n)$, written $\delta(A,n)$ is the concept $$(\{x.x\mid x \in
A\},2n).$$  Quine uses $I$ as the name of this operation.
\end{description}

It is quite appealing that this set of operations has the expressive
power of first-order logic.  The operations might be thought to have
some of the same intuitive appeal as the basic operations of Boolean
algebra.

\section{The expressive power of {\em CC\/}}
\begin{description}

\item[Theorem (Quine):] The expressive power of the calculus of
concepts is at least that of first-order logic with equality (if
concepts are understood via the interpretation given above).

\item[Proof:]  We use $V^1$ as the notation for the concept $({\cal D},1)$ 
whose extension is the whole domain of the theory being interpreted.
We can define $V^n$ for each positive $n$ as the product of $n$ copies
of $V^1$ and $V^0$ as $V^1/V^1$; $V^0$, the universal concept of
degree 0, is an absolute notion of the calculus, unlike the $V^i$'s
with $i>0$.

The sentence $x_i = x_j$ (for $i < j$) is represented by the concept
$V^{i-1}\times (\delta V^{j-i} / V^{j-i-1})\times V^k$ (for any $k$).

If $A$ and $B$ are concepts of the same degree, $\delta A / B$ is the
intersection $A \cap B$ of $A$ and $B$.

We proceed to represent a formula $\phi$ whose variables (free and
bound) have indices $\leq n$ using a concept of degree $n$.  We
proceed by induction on the structure of $\phi$.

If $\phi$ is an atomic sentence $R[x_{a_1}\ldots x_{a_k}]$, we let $A$
be a concept representing the formula $R[x_1,\ldots,x_k]$.  Let $E$ be the
intersection of the predicates of degree $n+k$ representing the
equality assertions $x_{a_i}=x_{n+i}$ for each $i \leq k$.  The
sentence $R[x_{a_1}\ldots x_{a_k}]$ is represented by the degree $n$
concept $((V^n \times A) \cap E)/V^{k}$
If $\phi$ is of the form $\sim\psi$, where $\psi$ is represented by a
concept $A$ of degree $n$, $\phi$ is represented by $A^c$.  If $\phi$
is of the form $\psi \wedge \chi$, where $\psi$ and $\chi$ are
represented by concepts $A$ and $B$ of degree $n$, $\phi$ is
represented by $A \cap B$.  All propositional logic operations can be
defined in terms of negation and conjunction.

Let $\phi$ be of the form $(\exists x_i.\psi)$, where $\psi$ is
represented by a concept $A$ of degree $n$.  Let $E$ be the
intersection of the concepts of degree $2n$ representing the assertions
$x_j = x_{n+j}$ for $j$ less than or equal to $n$ and not equal to
$i$.  $\psi$ is represented by the concept $((V^n\times A)\cap E)/V^n$.
The universal quantifier can be defined in terms of the existential
quantifier.

The proof of the theorem is complete.

\end{description}

\section{Other instantiations of the calculus of concepts}

In this section, we present other interpretations of the calculus of
degree, which will motivate the generality of our treatment of degree.

NOTE:  some of these examples are sketchy.

In any interpretation of the calculus of concepts, concepts are sets
of ``argument lists'' all of the same size.  Degree measures the size
of ``argument lists''.  The examples in this section illustrate how
the notion of ``argument list'' can be generalized while preserving
the usefulness of the operations of the calculus.

Multi-sorted theories are handled as readily as the single-sorted
theories in the natural interpretation.  An argument list is a
function $\alpha$ such that for each sort $\tau$ of the theory,
$\alpha(\tau)$ is a list of objects of sort $\tau$.  The degree
$|\alpha|$ of an argument list is also a function, from sorts to
natural numbers: $|\alpha|(\tau)$ is the length of the list
$\alpha(\tau)$.  The concatenation $\alpha.\beta$ of two argument
lists is defined so that $(\alpha.\beta)(\tau) =
\alpha(\tau).\beta(\tau)$.

A concept is a pair $({\cal A},|\alpha|)$ of a set ${\cal A}$ of
argument lists of degree $|\alpha|$.  The operations of the calculus
of concepts are defined in essentially the same way as they are in the
interpretation above, with attention to the ``arithmetic'' of degrees:
addition of degrees is defined in the obvious way (and a degree
$2|\alpha|$ is read $|\alpha|+|\alpha|$), and this allows the
definition of product and diagonal operations just as above.  The
degree of the quotient $A/B$, where $A$ is of degree $|\alpha|$ and
$B$ is of degree $|\beta|$, is the degree $|\alpha|-|\beta|$ defined
by $(|\alpha|-|\beta|)(\tau) = \max(|\alpha|(\tau)-|\beta|(\tau),0)$;
this subtraction operation generalizes a natural number subtraction
operation.

It is straightforward to establish that the interpretation of
multi-sorted first-order logic with equality in the calculus of
concepts goes in essentially the same way as we showed it does for
single-sorted first-order logic above.  The differences are that one
needs to have a different base concept $V^1_{\tau}$ for each sort
$\tau$ (the set of all one-term argument lists of that sort); all
degrees are then finite sums of these base concepts.  Equality is
defined (between variables of the same sort) in the same way given in
the argument above.  Negation and conjunction are handled in exactly
the same way.  The treatment of existential quantification is also
essentially the same; it is important to note that the equality
concepts making up the intersection $E$ may be products of the
equality concepts for two variables of a given sort with universal
concepts of different sorts (they need to be padded to the degree of
the concept over which we are quantifying).

This generalization is clearly useful; it will be nice to have a
formalization which unified multi-sorted and single-sorted logics.
However, there are further generalizations whose usefulness is much
less evident.  For these, we will only discuss the nature of argument
lists and degrees, as we are not seriously interested in doing logic
in these systems.

Let $X$ be a set and let a concept be a set of functions from $[0,r)$
to $X$ where $r$ is a nonnegative real number.  An argument list here
is a function from $[0,r)$ to $X$.  The degree of an argument list is
the nonnegative real number $r$.  We concatenate argument lists of
degrees $r$ and $s$ to get an argument list of degree $r+s$ in the
obvious way.  A full development of the calculus of concepts is easy
on this basis.  This is an ``unintended'' implementation of the
calculus, because ``argument lists'' here are ``nonatomic''; we see
analogues of ``blocks of variables'' here without being able to single
out individual variables.

Even worse, let $X(r)$ be a set for each element $r$ of $[0,1]$ and
let $f$ be a continuous function from $[0,1]$ to $[0,\infty)$; let a
concept of degree $f$ be a set of functions with domain $\{(r,s)\mid r
\in [0,1] \wedge f(r) < s$ sending each pair $(r,s)$ in their domain
to an element of $X(r)$.  The degrees here are determined by the
function $f$.  Addition of degrees corresponds naturally to addition
of these functions and it is fairly obvious how to define
concatenation of ``argument lists''.  Here we do not have the expected
``granularity'' of variables or even of the analogues of sorts of
object!  (this needs expansion).

A generalization of concepts which is arguably more natural than the
ones above but which suggests that we should weaken our axioms is the
following: we return to a single domain $X$, but we let ``argument
lists'' be functions from countable well-orderings to $X$, with degree
determined by the countable ordinal which is their domain.  A subtlety
is that we treat earlier elements of the countable ordinal as {\em
later\/} elements of the ``block of variables'' the argument list
represents; the concatenation of a concept with domain $\alpha$ and a
concept with domain $\beta$ is a concept of domain $\beta + \alpha$.
This implementation is of interest as handling an infinitary logic (it
is not ``unnatural'' in the way the last two are); but it suggests a
weakening of our axioms, because the operation of product is not
commutative on degrees in this implementation.  We actually do adopt
commutativity of product on degrees as an axiom of the formal system
presented here, because it simplifies the axiomatization of
subtraction of degrees, but we need to observe that a further
generalization is possible.


\section{Axioms for the calculus of degree}

NOTE: the indexing of the axioms in this section is mixed up!  I'll
fix it later.

In the formal system {\em CC\/}, concepts do not come with explicit
indications of degree.  It is necessary to verify that degree is
somehow representable in the calculus and to axiomatize its
properties.

We rely on the intended interpretation to develop the basic approach.
In the intended interpretation, there is a canonical object in each
degree which we may as well use to represent that degree: this is the
universal concept $({\cal D}^n,n)$.  In the intended interpretation,
the universal concept of the same degree as a concept $A$ can be
expressed as $(\delta A/A^c)^c$; it is straightforward to check that
this works (and that it works in our other interpretations).  In
uninterpreted {\em CC\/}, we introduce the following

\begin{description}
\item[Definition:]  

$V^A$, called the {\em degree of $A$\/}, is defined as $(\delta
A/A^c)^c$.  $\emptyset^A$ is defined as $(V^A)^c$.

\end{description}

The operations on degrees which are needed in the intended
interpretation are addition and subtraction of natural numbers; our
intention is that the degree of $A \times B$ will be the sum of the
degrees of $A$ and $B$, while the degree of $A/B$ will be the
difference of the degree of $A$ and the degree of $B$.  In the
intended interpretation, we decided that the degree of $A/B$ would be
0 when the degree of $B$ exceeded the degree of $A$; this corresponds
to a natural decision as to how to define subtraction as a complete
operation on the natural numbers.  We make the modification suggested
by the last example in the previous section that commutativity does
not hold for ``addition'' of degrees, though a special case will hold.

We adopt these axioms which reflect the effects of the operations
other than quotient on degree:

\begin{description}


\item[D1:]  $V^{V^A} = V^A$

\item[D2:]  $V^{A^c} = V^A$

\item[D3:]  $V^{\delta A} = V^A \times V^A$

\item[D4:]  $V^{A \times B} = V^A \times V^B$

\end{description}

We adopt an axiom expressing the associativity of product on all
concepts, not just degrees.
\begin{description}
\item[D5:]  $(A \times B) \times C = A \times (B \times C)$
\end{description}

We now consider the problem of the quotient operation and the
``subtraction'' operation on degrees.

We introduce the two absolutely definable concepts.

\begin{description}

\item[D7:]  $V^{A/A} = V^{B/B}$

\end{description}
Axiom {\bf D7} motivates the following

\begin{description}

\item[Definition:]

We define $V^0$ as $V^{A/A}$ and $\emptyset^0$ as $(V^0)^c$.

\end{description}
Degree 0 is the only degree which can be defined in absolute terms.
In the intended interpretation, degree 0 has the two inhabitants $V^0$
and $\emptyset^0$, which are natural representatives of the truth
values.  $V^0$ is the identity of degree addition.

\begin{description}

\item[D8:] $A \times V^0 = V^0 \times A = A/V^0 = A$

\end{description}

We tackle the general problem of axiomatizing subtraction of degrees.
Our approach is to view the degrees as forming a lattice.  The
intention is that the difference $\delta - \epsilon$ of two degrees
can be understood to be actually equal to $\delta -
\min(\epsilon,\delta)$ (which is true in the natural number case).

In the natural number case, one of $\delta - \epsilon$ and $\epsilon -
\delta$ will be 0; this is not the case in the multi-sorted
interpretations given above.  What is true in these interpretations is
that $\delta - \epsilon$ and $\epsilon - \delta$ will be ``disjoint''
in the sense that they will not share variables of any given sort.

We will not need to have max or min operations as primitives in our
calculus, because max and min are naturally definable in terms of
subtraction.  $\min(\delta,\epsilon) = \delta - (\delta - \epsilon) =
\epsilon - (\epsilon - \delta)$ is easily verified as a property of
natural number subtraction.  Similarly, $\max(\epsilon,\delta) =
\delta + (\epsilon - \delta) = \epsilon + (\delta - \epsilon)$.

We discuss the justification for this in our more general context.
The defining property of $\delta - \epsilon$ is that
$\min(\delta,\epsilon) + (\delta - \epsilon) = \delta$.  The defining
property of $\delta - (\delta - \epsilon)$ is then that
$\min(\delta,\delta - \epsilon) + (\delta - (\delta - \epsilon)) =
\delta$.  Since we expect $\min(\delta,\delta - \epsilon)$ to be
$\delta - \epsilon$ (if there is any justice), we see that this
definition depends on commutativity.  For this reason, we do adopt the
axiom

\begin{description}

\item[D6] $V^A \times V^B = V^B \times V^A$

\end{description}

in spite of the fact that it restricts the generality of the calculus.
It appears that an implementation general enough to cover the
infinitary implementation discussed in the last section would require
a primitive notion of minimum on degrees; we prefer to work in an
context where all notions can be defined in terms of Quine's basic
operations.

We note that the minimum of degrees $V^A$ and $V^B$ is $V^A/V^A/V^B$;
this motivates the right grouping of the quotient operator.

\begin{description}

\item[D10:] $V^A/B = V^{V^A/V^B}$

\end{description}

This axiom expresses the idea that the degree of a quotient is
determined by the degrees of the concepts involved.  We do not adopt
the axiom $V^A/B=V^A/V^B$, because this makes commitments about bad
quotients (those where the degree of $B$ is not less than or equal to
the degree of $A$).

\begin{description}

\item[D11:] $(V^A \times V^B)/V^B = V^A$

\item[D12:]  $(V^A/V^{A/B}) \times V^{A/B} = V^A$

\end{description}

These axioms capture the defining properties of subtraction.

\begin{description}

\item[D13:]  $V^A/V^A/V^B = V^B/V^B/V^A$

\end{description}

This axiom expresses the commutativity of the degree minimum
operation.

\begin{description}

\item[D14:]  $A/(B\times C) = (A/C)/B$

\item[D15:]  $(A \times B)/V^{B/C} = A \times (B/V^{B/C})$

\end{description}

These axioms capture technical points about the relationship between
product and quotient.  D14 is the only axiom which says anything about
``bad quotients''; it is compatible with the notion that all bad
quotients are empty.

\begin{description}

\item[D17:] $\delta A/V^A = A$

\end{description}

This might not be regarded as a degree calculation, but it is another
rule of calculation which will be needed in the following section.

NOTE:  some argument for the adequacy of these axioms is needed, though
this may turn out to be implicit in the completeness proof.

\section{Rules for sequent calculus presentation of CC}

Sequents are of the form $\Gamma \models \Delta$, where $\Gamma$ and
$\Delta$ are sets of concepts and all concepts in $\Gamma \cup \Delta$
are of the same degree.

Degree calculations need to be allowed.  We also need to recognize the
special category of ``equality concepts'' $V^A \times ((\delta
V^B)/V^{B/C})$; these represent assertions of equality between blocks
of variables.  The fact that $(\delta V^A)/V^{A \times B}$ = $V^A/V^B$
needs to be taken into account; this is subsumed under degree
calculations.

Axioms: $\Gamma,A \models A,\Delta$.

exercise:  $\models V^A$ is provable.

\begin{prooftree}
	\ljudge{\Gamma}{A,\Delta}
\justifies
	\ljudge{\Gamma,A^c}{\Delta}
\thickness=0.08em
\shiftright 2em
\using
	{\bf NEG-L}
\end{prooftree}


\begin{prooftree}
	\ljudge{\Gamma,A}{\Delta}
\justifies
	\ljudge{\Gamma}{A^c,\Delta}
\thickness=0.08em
\shiftright 2em
\using
	{\bf NEG-R}
\end{prooftree}

These rules for negation are exactly what we expect.  An additional
calculation allowed with complement is the fact that complement
commutes with left or right padding:  $V^A \times B^c = (V^A \times B)^c$
and $A^c \times V^B = (A \times V^B)^c$.

\begin{prooftree}
	\ljudge{\Gamma,V^A \times B,A \times V^B}{\Delta}
\justifies
	\ljudge{\Gamma,A \times B}{\Delta}
\thickness=0.08em
\shiftright 2em
\using
	{\bf PROD-L}
\end{prooftree}

\begin{prooftree}$
	
	$\ljudge{\Gamma}{V^A \times B,\Delta}$

	$\ljudge{\Gamma}{A \times V^B,\Delta}$

$\justifies
	\ljudge{\Gamma}{A \times B,\Delta}
\thickness=0.08em
\shiftright 2em
\using
	{\bf PROD-R}
\end{prooftree}

These rules for product are exactly what we should expect.  See the
additional rule {\bf COPY} below, which eliminates right padding; this
is also a kind of product rule.

\begin{prooftree}
	\ljudge{\Gamma^A,A,V\times B}{\Delta}
\justifies
	\ljudge{\Gamma,A/B}{\Delta}
\thickness=0.08em
\shiftright 2em
\using
	{\bf QUOT-L}
\end{prooftree}

\begin{prooftree}$

	$\ljudge{\Gamma}{C/V^B}$
	
	$\ljudge{\Gamma^A,C}{A,\Delta}$

	$\ljudge{\Gamma^A,C}{V \times B,\Delta}$

$\justifies
	\ljudge{\Gamma}{A/B,\Delta}
\thickness=0.08em
\shiftright 2em
\using
	{\bf QUOT-R}
\end{prooftree}

(where $C$ is any concept and $V$ is the appropriate universal
concept; the trick is that $C$ can always be taken to be a conjunction
of equalities, and the leftmost hypotheses in that case is proved by
calculation)

($\Gamma^A,\Delta^A$ are suitably right padded versions of $\Gamma,\Delta$)

An important fact about these rules is that they only support
``quantification'' over final ``blocks of variables''; the limitation
is circumvented by the {\bf COPY} rule.

\begin{prooftree}
	\ljudge{\Gamma_1^{\Gamma_1\times \Gamma_2},V^{\Gamma_1}\times \Gamma_2,(\delta V^{\Gamma_1 \times \Gamma_2})/V^A}{\Delta_1^{\Gamma_1\times \Gamma_2},V^{\Gamma_1}\times \Delta_2}
\justifies
	\ljudge{\Gamma_1,\Gamma_2\times V^A}{\Delta_1,\Delta_2\times V^A}
\thickness=0.08em
\shiftright 2em
\using
	{\bf COPY}
\end{prooftree}

(the notation here is tricky: $V^{\Gamma_1}$ is the universal concept
of the common degree of $\Gamma_1$, and degrees of sets of concepts
are used in several places in ways we have above used degrees of
concepts. An operation $V^A \times \Gamma$ yields a set of concepts
padded on the left with $V^A$.)

({\bf COPY} is a rule for elimination of right padded concepts; the
application is to bring quotients (including generalized equality
concepts) and concepts to which generalized equality concepts are to
be applied to the position where the quotient or diagonal rules can be
applied).

The notation in the {\bf COPY} rule is strained because we need economical
ways to describe a rule which allows manipulation of a set of concepts
of arbitrary size on each side.

\begin{prooftree}
	\ljudge{\Gamma,V^A\times D \times D,V^A \times C^D \times B^D}{V^A \times F^D \times E^D,\Delta}
\justifies
	\ljudge{\Gamma,V^A\times B^D \times C^D,V^A \times \delta D}{V^A \times E^D \times F^D,\Delta}
\thickness=0.08em
\shiftright 2em
\using
	{\bf EQ-L}
\end{prooftree}

\begin{prooftree}$
	$\ljudge{\Gamma}{V^A \times B \times B,\Delta}$

	$\ljudge{\Gamma,V^A \times C^B \times D^B}{V^A \times D^B \times C^B,\Delta}$
$\justifies
	\ljudge{\Gamma}{V^A \times \delta B,\Delta}
\thickness=0.08em
\shiftright 2em
\using
	{\bf EQ-R}
\end{prooftree}

(in {\bf EQ-R}, the concepts $C^B$ and $D^B$ are fresh concepts not
mentioned in the conclusion and of the same degree as $B$)

The following rules handle the problem which we call ``block splicing'':
an equality concept asserts the equality of two blocks of variables;
we need to be able to decompose it into the conjunction of equations
between corresponding sub-blocks.

\begin{prooftree}
	\ljudge{\Gamma,V^A \times ((\delta B^{C\times D})/V^D) \times V^D,
V^{A\times C} \times (\delta V^{D \times C})/V^C}{\Delta}
\justifies
	\ljudge{\Gamma,V^A \times \delta B^{C\times D}}{\Delta}
\thickness=0.08em
\shiftright 2em
\using
	{\bf SPLICE-L}
\end{prooftree}

\begin{prooftree}$

	$\ljudge{\Gamma}{V^A \times ((\delta B^{C\times D})/V^D) \times V^D ,\Delta}$

	$\ljudge{\Gamma}{V^{A\times C} \times (\delta V^{D \times C})/V^C ,\Delta}$
$\justifies
	\ljudge{\Gamma}{V^A \times \delta B^{C \times D},\Delta}
\thickness=0.08em
\shiftright 2em
\using
	{\bf SPLICE-R}
\end{prooftree}

\section{The Completeness Proof}

\subsection{Objects from Concepts}

The calculus of concepts is a language without nouns or pronouns;
there is nothing in the language which appears to refer directly to
``objects''.  In this subsection, we discuss how to formally interpret
talk of sorts of object and variables referring to objects from a
theory in the calculus of concepts.

We extract this information from the calculus of degrees.  Each degree
corresponds to a sort of object and degrees can further be used as
``indices'' of ``variables'' of each sort.  In the natural
interpretation of the calculus of concepts, which corresponds to
single-sorted predicate logic, degree $n$ corresponds to lists of
length $n$ of objects of the single underlying sort.  In the natural
interpretation, there are blocks of variables of length $n$ beginning
at each natural-number-indexed position; the concept $V^k \times B^n
\times V^m$ includes those $(k+n+m)$-tuples in which the $k$th block
of $n$ variables has the property represented by the degree $n$
concept $B^n$.

This needs to be generalized somewhat.  Each degree $V^A$ represents a
sort.  The ``variables'' of the sort associated with $V^A$ are
associated with concepts of the form $V^B \times V^A \times V^C$, in
which the degree $V^C$ serves strictly as padding; it is the degree
$B$ that determines which ``variable'' of the sort associated with
$V^A$ is being considered.  However, an equivalence relationship on
the $V^B$'s, determined by $V^A$, must be imposed.  That is, different
$V^B$'s may index the same ``variable'' of sort $V^A$.  This can be
seen by considering the interpretation of multi-sorted predicate logic
givne above: if $V^A$ interpreting one of the basic sorts is being
considered , only the number of variables of that sort in the ``block
of variables'' associated with the degree $V^B$ will be relevant to
which variable is being denoted.  Degrees $V^B$ and $V^C$ are taken to
be equivalent as indices for variables of sort $V^A$ just in case
$V^{A/A/B/C}$ and $V^{A/A/B/C}$ are both equal to $V^0$.

A ``variable'' is then determined by a degree $V^A$ determining its
sort and an equivalence class of degrees under the indicated
equivalence relation determining its index (HOLE: prove that it is an
equivalence relation!).  The models which we will construct in the
next subsection are sets of variables in this sense.

Variables generally represent lists of arguments of sorts taken from the
underlying logic.  The operation of concatenation on variables is easily
definable.  (HOLE:give details).

\subsection{The Proof Itself}

In this subsection, we assume that a sequent $\ljudge{\Gamma}{\Delta}$
is unprovable and proceed to construct a model on which the calculus
of concepts has an interpretation in which this sequent turns out to
be invalid.  We need to start with a complete knowledge of the
particular calculus of degree in which we are working, in order to be
able to construct ``variables'' in the manner described in the
previous subsection: concepts in our model will be represented as sets
of ``variables'' of the sort corresponding to their degree.

It will be assumed that the calculus of degrees is at most countably
infinite, which limits the number of sorts and variables of each sort
to countably infinite size.

We suppose the existence of symbols for countably many ``fresh''
concepts of each degree, not present in the unprovable sequent.

We construct a tree of sequents as in the usual completeness proofs.
At any level, we first carry out all applications of the negation,
product, and equality rules, which involve no complications and do not
change degree.  Applications of equality rules may need to be iterated
to make sure that everything we want is done.  Block splicing does not
change degree, but it involves many possible decompositions of the
degree of the equality concept into two degrees; we use all the
decompositions whose index in a suitable enumeration is bounded by our
current level in the tree (the way we handle copying will ensure that
each diagonal concept will be visited enough times that all possible
decompositions will be considered).

There remain to be considered the rules for quotient and the rule of
copying, which change degree.  The problem here is that we can't
perform many rules at the same time, because of the degree change.  We
handle these using enumerations of all quotients and enumerations of
all pairs of finite sets of concepts of uniform degree; each
enumeration is to have the property that all items occur infinitely
often.  At a given level of the tree, we apply the quotient rule if
the indexed quotient term (left padded suitably and possibly right
padded as well) is present (copying to the front if necessary), and
then apply copying if the indicated pair of finite sets of concepts
(right padded suitably) is present to be copied.  In the rule {\bf
QUOT-R}, we use all intersections of equality concepts which have the
correct degree and appear in an enumeration of such at a point before
the index of the current stage.

We weaken at all levels so that concepts present on either side
anywhere on a branch are present everywhere above that point on that
side of the branch.

Now the usual argument establishes that there is an infinite branch in
this tree of sequents consisting of unprovable sequents (if there
weren't we would have a proof)..  We claim that the infinite branch
gives a partial description of a model on which there is a sensible
interpretation of the calculus of concepts in which the original
unprovable sequent is invalid.

The elements of the model are all the ``variables'' for the given
calculus of degree defined as in the preceding subsection.  Concepts
in the model are interpreted by sets of ``variables'' of the
appropriate degree.  Variables in the model are interpreted as being
equal when they belong to the same interpreted concepts.  The variable
of sort determined by $V^A$ and index determined by $V^B$ will belong
to the interpreted concept $A$ just in case some concept $V^{B'} \times
A \times V^C$ with $B'$ equivalent to $B$ as an index appears on the
left side of some sequent in the branch.

The following can be deduced: complement and product have their
natural semantics, in the sense that if $A^c$ is asserted (denied),
$A$ will be denied (asserted), while if $A \times B$ is asserted $A$
and $B$ will be asserted of appropriate variables interpreted as
sub-blocks, and if it is denied one of these will be denied of the
appropriate sub-block.  If two variables are asserted to be equal, any
property asserted of one will also be asserted of the other.  If two
variables are denied equality, there will be a concept asserted of one
and denied of the other.  If a block is asserted to be equal to
another block, corresponding sub-blocks will be asserted to be equal;
if it is denied to be equal to another block, at least one subblock in
any decomposition will be denied equality with the corresponding
subblock.  If a quotient $A/B$ is asserted, there will be a variable
of which $A$ is asserted of whose appropriate final sub-block $B$ is
asserted and of whose appropriate initial block $A/B$ is asserted.  If
a quotient $A/B$ is denied, each block of variables of the degree
associated with $A$ must either be denied inclusion in $A$ or have its
final segment of the degree of $B$ denied inclusion in $B$.  We have
combinatorial completeness, in the sense that every combination of
subblocks into blocks is realized somewhere; yes, this is done by the
quotient rule itself (if copying actually does it, the quotient rule
can be simplified).

This is enough; the model works.

HOLE:  this is enough for me to see that it works, but needs to be
throughly rewritten for anyone else to see this!

\section{Why should we care?}

Is there any respect in which the calculus of concepts is an
improvement on standard first-order logic with equality?  It cannot be
regarded as a practical system in which to reason!  The only possible
advantage is conceptual economy, and here it does have certain
advantages.  The notion which appears essentially in a syntactical or
semantic account of standard first-order logic and which does not
appear in the calculus of concepts is that of {\em substitution\/}.
Moreover, the notion of substitution needed for first-order logic is
quite complicated, due to the appearance of variable binding
operators.  In the calculus of concepts, the notion of substitution is
replaced by a notion of copying.  The mechanics of the system become
more complex, in exactly the way that imagining an account of
substitution in the presence of variable binding in terms of copying
would suggest.  But the basic concepts and the rules associated with
those basic concepts retain a certain appealing simplicity, though
somewhat encumbered by ``calculations of degree''.

The complete disappearance of the analogues of nouns and pronouns from 
this system is somewhat startling.

\section{Notes to myself--not to be viewed as part of paper draft}

In the completeness proof, we are developing a model in which objects
are ``positions'' in ``blocks of variables'' and concepts are sets of
these positions, respecting equivalence relations on the positions imposed
by equality hypotheses.

I believe that the forms of the rules now given are the correct ones
needed for the completeness proof.

Add a description of the typed system.

Outline the completeness proof.

Think about good treatment of arbitrary situations with the degree
calculus.

Extracting sorts of objects from the degree calculus:

each degree corresponds to a sort of object (the degree $V0$ to a
trivial sort with just one element).  A product degree $V^A \times
V^B$ is inhabited by concatenations of objects of the sort associated
with $V^A$ and objects of the sort associated with $V^B$.
``Concatenation'' rather than pairing because this is an associative
operation; concatenation is the operation which builds argument lists.
Note that the structure of argument lists is not thought of as linear
here -- there is a separate linear list of arguments of each sort.

concepts of higher degree ``refer'' to many objects of any given sort:
degrees also indicate ``positions'' of blocks of a given sort.  The
concept $V^A \times B$ picks out objects in the ``position'' indicated
by $V^A$ of the sort associated with $V^B$ and which happen to fall
under $B$.  Only the part of $V^A$ which is ``commensurable'' with
$V^B$ is relevant to the position specified?  Is the part of $A$
commensurable with $B$ computable in the calculus of degrees?  No --
because no such object need exist in the continuous function example!

How do we tell whether the an object of the degree of $C$ with
position $A$ is the same as the object of the degree of $C$ with
position $B$?  Is it enough for $B/A$ and $A/B$ to be incommensurable
with $C$?  That seems right; it requires some verification.  (it might
depend on commutativity!)  This seems to express the notion that two
degrees have equivalent ``parts commensurable with'' a third; we could
add equivalence classes under this relation as virtual degrees to get
a kind of completion, maybe?

commutativity for incommensurables seems to make sense, but
commutativity of commensurable lists does not seem sufficiently
general.  $V^{A/B} \times V^{B/A} = V^{B/A}\times V^{A/B}$ ought to be
true.

any pair of degrees $A$ and $B$ can be expressed as sums of the
minimum of the two plus degrees $A'$ and $B'$.

I would like to prove that the equations of the calculus of degree
implicit in any finite sequent can be realized in a conventional
multi-sorted context; this would allow a sensible equation of the
calculus of concepts with multi-sorted first order logic with
equality, leaving aside the bizarre nonatomic calculi of degrees which
are possible.

Can we take any degree calculus and construct a completion thereof which 
will behave sensibly (allowing things like commensurable parts)?

is commutativity of degree addition really needed?  Think about reverse
well-ordered lists of arguments; these would seem to satisfy the 
conditions for a calculus of degree, but the degree calculus one
obtains is not commutative!  This can be used as an example to illustrate
why the asymmetry in the treatment of left and right ends might be
appropriate -- there might not be a left end!

Implementation:  concepts are sets of functions from countable
ordinals to a set $X$, with the domain of a concept determining its
degree, and the ``beginning'' of the domain corresponding to the
final part of the ``argument list''. (or maybe this example suggests
that we should view quotients as acting on first variables and copying
as proceeding toward the beginning?)  product, quotient, complement
and diagonal all have sensible interpretations, and product is not
commutative on degrees!  it seems that in the absence of commutativity
it might be difficult to come up with a sensible way of handling bad
quotients.

For construction of objects, suppose the calculus of degree is known
completely; we have proposed a definition for equivalence of variables
of each sort (sorts indexed by degrees and variables in each sort
indexed by degrees up to equivalence of parts commensurable with the
degree indexing the sort).  The actual objects can then be identified
with positions, with an equivalence relation determined by the equality
concepts in the sequents one is working with.

So we should assume that we know what the calculus of degree is in the
completeness proof (this seems natural anyway) and that we already
know that implicit conditions on degree are satisfied in our sequent.
Then we can use the calculus of degree itself as building material for
the elements of our ``model'': each degree codes a sort, there is a
natural way to relate objects of different degree as concatenations,
and so forth.  But we cannot do this without full commutativity:
positions are not specifiable in this way in the reverse well-ordering
example (think about a reverse well-ordering of type $\omega$; all
positions here are left padded by the same degree!  So stick with
commutativity.  But can positions still be specified in relation to
the sequents we work with?  I think so.  Noncommutativity still might
work; our method of specifying position is what is incorrect!  It is
the relation to the end of the sequent that we should look at (and fix
things in relation to earlier sequents as well; but this isnt a
problem since concepts from earlier sequents are always retained and
updated).  The correct way to describe position is to consider
concepts $V \times V^B \times V^A$, with $V^B$ determining sort and
$V^A$ determining position (automatically adjusted in each sequent);
the fact that $V$ is not necessarily uniquely determined by $V^A$ is
only a curiousity.  The equivalence between different $V^A$'s is
determined in the same way as before.  it is useful for the calculus
of degrees to be countable, since we need to limit the number of
witnesses to be handled (the number of equality concepts that need to
be used at each stage).  It is natural for this calculus to be
countable as well!  (notice that countable calculi can still exhibit
the weird incommensurability phenomena).

Would it be better to build a complete model description as in
Marcel's proof?  Then I would not need to worry about partial
valuations in quite the same way?

Am I going to have trouble with partial valuations if the sort structure
is ill-founded?  No, I don't see such a problem.  We are looking at
concrete objects at each type level; simply accept only those equations
we are forced to have, and everything should work out correctly -- maybe!!!
Consider phenomena that occur with equations between ill-founded pairing
structures?  I don't see that these are really a problem.

Just for fun, I think I should give an axiom which allows analysis of bad
quotients.  How hard would it be to express the idea that every bad quotient
is to be identified with the appropriate empty concept?

If we accept as a logical truth the assertion that each sort contains
more than one object, we can assert that $A/B$ is empty if $V^B/A$
contains more than one element.  (we also become able to express ideas
such as linearity of the type structure -- but maybe that's already
expressible?) The problem here is expressing the idea of the
nontriviality of a sort.  If we allow bad quotients to be universal
rather than empty, there might be a cleaner treatment?

another idea: $A/B$ is equivalent to $A/B$'s conjunction with the
assertion that $B$ is logically equivalent to $V^{B/A} \times B/V^{B/A}$.
This might be pretty clean!

Define left quotient $A//V^{A/B}$, the left quotient of $A$ by $V^{A/B}$;
this is a pretty easy diagramming problem.

Then $A/B = A/B//V^{B/A}$ expresses what is perhaps the nicest solution to
the bad quotient problem.

\end{document}
