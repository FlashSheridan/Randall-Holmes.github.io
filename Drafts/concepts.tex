\documentstyle{article}

\title{Quine's calculus of concepts \footnote{Supported by the U. S. Army Research Office, grant no. DAAH04-94-G-0247}}
\begin{document}
\maketitle
\section{Introduction}

In this paper, we give a complete presentation of a system, the
Calculus of Concepts, sketched by Quine in \cite{concepts} and
\cite{negative}.  Quine's concern was to present
a system with the full expressive power of first-order logic which
made no use of bound variables.  In this he succeeded, but he did not
equip the system with axioms or rules of inference, so the system as
it stands can only be described as a sketch.

We will refer to our full system as {\em CC\/} in the sequel.  Its
notation and certain decisions about default values differ
inessentially from Quine's approach.  The system is more general than
Quine's; it will handle multi-sorted first-order logic, for example.

\section{The Natural Semantics of {\em CC\/}}

We follow Quine's example in first presenting {\em CC\/} via the
originally intended interpretation.  It will turn out that the system
has other quite natural interpretations as well.

We begin by describing the language of {\em CC\/}.  {\em CC\/} is a
strictly equational theory; sentences of {\em CC\/} are of the form $T
= U$, where $T$ and $U$ are terms of the language of {\em CC\/}.

Atomic terms of the language of {\em CC\/} are taken from a countable
supply of free variables.  Atomic constants will be introduced as
well.  Atomic terms of {\em CC\/} are terms of {\em CC\/}; if $T$ and
$U$ are terms, then $T^c$, $T\times U$, $T/U$, and $\Delta T$ are
terms.  Other operations on terms may be introduced by definition.
All terms are built from atomic terms by the primitive operations and
operations defined in terms of the primitive operations below.  Unary
operations are always considered to have higher precedence than binary
operations; no other precedence convention is adopted.

We now present the intended interpretation of the notion of
``concept''.
\begin{description} \item[Definition:] If ${\cal D}$ is our universe
of discourse and $n \in {\cal N}$, we define a {\em concept of degree
$n$\/} as a pair $(S,n)$ with $S \subseteq {\cal D}^n$.  (The pairing
with $n$ is needed because empty concepts of different degrees need to
be distinct.)  We identify 1-tuples of elements of ${\cal D}$ with
elements of ${\cal D\/}$ itself, and we postulate a unique 0-tuple, so
concepts of degrees 0 and 1 make sense. \end{description}

Any formula in the language of first-order logic with variables
indexed by the positive integers can be associated in a natural way
with a concept (actually, with a concept of each sufficiently large
degree):

\begin{description}

\item[Definition:] A concept $(A,n)$ is said to represent a formula
$\phi$ just in case $\phi$ has no free variable other than variables
$x_i$ for $i \leq n$ and $\{(x_1,\ldots,x_n) \mid \phi\} = A$.  Notice
that concepts of degree 0 represent sentences with no free variables
(the two concepts of degree 0 represent the truth values).

\end{description}

Each term of the language of {\em CC\/} is intended to represent a
concept.  Note that there is no explicit indication of numerical
degree in {\em CC\/}; the explicit indications of degree are a feature
of the interpretation being given.

We now introduce the operations of CC (in their intended interpretation)

We first define $x.y$, for $x \in {\cal D}^m, y \in {\cal D}^n$, as
the element of ${\cal D}^{m+n}$ obtained by concatenating $x$ and $y$.

\begin{description}

\item[complement:] The {\em complement\/} of a concept $(A,n)$,
written $(A,n)^c$ is $$({\cal D}^n-A,n).$$ (Quine would write $-(A,n)$).

\item[product:] The {\em product\/} of concepts $(A,m)$ and $(B,n)$,
written $(A,m)\times (B,n)$ is $$(\{x.y\mid x\in A \wedge y \in
B\},m+n).$$ Quine calls this the ``Cartesian product'', but this is
somewhat misleading, since the underlying operation is concatenation
rather than pairing; for example, this operation is associative and
Cartesian products properly so-called are not.

\item[quotient:] The {\em quotient\/} of a concept $(A,m)$ by a
concept $(B,n)$, written $(A,m)/(B,n)$, is $$(\{x \mid \exists y \in
B. (x.y \in A)\},\max(m-n,0)\}).$$  Notice that the subtraction of
degrees natural in this theory is a subtraction on natural numbers.
Quine called this the ``image'' operation and used the double-quote
notation for image to represent it.  The definition of quotient when
the degree of $B$ exceeds the degree of $A$ is a ``don't care'' case,
and in fact is different under this definition than it will be under
our axioms.

\item[diagonalization:] The {\em diagonalization\/} of a concept
$(A,n)$, written $\Delta(A,n)$ is the concept $$(\{x.x\mid x \in
A\},2n).$$  Quine uses $I$ as the name of this operation.
\end{description}

It is quite appealing that this set of operations has the expressive
power of first-order logic.  The operations might be thought to have
some of the same intuitive appeal as the basic operations of Boolean
algebra.

\section{The expressive power of {\em CC\/}}
\begin{description}

\item[Theorem (Quine):] The expressive power of the calculus of
concepts is at least that of first-order logic with equality (if
concepts are understood via the interpretation given above).

\item[Proof:]  We use $V^1$ as the notation for the concept $({\cal D},1)$ 
whose extension is the whole domain of the theory being interpreted.
We can define $V^n$ for each positive $n$ as the product of $n$ copies
of $V^1$ and $V^0$ as $V^1/V^1$; $V^0$, the universal concept of
degree 0, is an absolute notion of the calculus, unlike the $V^i$'s
with $i>0$.

The sentence $x_i = x_j$ (for $i < j$) is represented by the concept
$V^{i-1}\times (\Delta V^{j-i} / V^{j-i-1})\times V^k$ (for any $k$).

If $A$ and $B$ are concepts of the same degree, $\Delta A / B$ is the
intersection $A \cap B$ of $A$ and $B$.

We proceed to represent a formula $\phi$ whose variables (free and
bound) have indices $\leq n$ using a concept of degree $n$.  We
proceed by induction on the structure of $\phi$.

If $\phi$ is an atomic sentence $R[x_{a_1}\ldots x_{a_k}]$, we let $A$
be a concept representing the formula $R[x_1,\ldots,x_k]$.  Let $E$ be the
intersection of the predicates of degree $n+k$ representing the
equality assertions $x_{a_i}=x_{n+i}$ for each $i \leq k$.  The
sentence $R[x_{a_1}\ldots x_{a_k}]$ is represented by the degree $n$
concept $((V^n \times A) \cap E)/V^{k}$
If $\phi$ is of the form $\sim\psi$, where $\psi$ is represented by a
concept $A$ of degree $n$, $\phi$ is represented by $A^c$.  If $\phi$
is of the form $\psi \wedge \chi$, where $\psi$ and $\chi$ are
represented by concepts $A$ and $B$ of degree $n$, $\phi$ is
represented by $A \cap B$.  All propositional logic operations can be
defined in terms of negation and conjunction.

Let $\phi$ be of the form $(\exists x_i.\psi)$, where $\psi$ is
represented by a concept $A$ of degree $n$.  Let $E$ be the
intersection of the concepts of degree $2n$ representing the assertions
$x_j = x_{n+j}$ for $j$ less than or equal to $n$ and not equal to
$i$.  $\psi$ is represented by the concept $((V^n\times A)\cap E)/V^n$.
The universal quantifier can be defined in terms of the existential
quantifier.

The proof of the theorem is complete.

\end{description}

\section{Rules of inference}

The rules of inference of {\em CC\/} are those appropriate for any
purely equational theory.  We present them explicitly for the sake of
completeness.  In this section only, capital letters represent general
terms of {\em CC\/}, while lower-case letters represent atomic terms
of {\em CC\/} (concpet variables).

Any theorem of {\em CC\/} is of the form $T = U$, where $T$ and $U$
are terms of {\em CC\/} as defined above.  We use the notation
$A[T/x]$ to represent the result of substituting the term $T$ for the
variable $x$ throughout the term $A$.

\begin{description}

\item[(Axiom)]  

Axioms of {\em CC\/} are theorems of {\em CC\/}.  (The axioms are
enumerated below).

\item[(Sub1)]  

If $A = B$ is a theorem of {\em CC\/} and $C[A/x] = D[A/x]$ is a
theorem of {\em CC\/}, then $C[B/x] = D[B/x]$ is a theorem of {\em
CC\/}.  This rule allows substitution of equals for equals.  It also
supports symmetry and transitivity of equality.

\item[(Sub2)]

If $A[y/x] = B[y/x]$ is a theorem of {\em CC\/} (note that $y$ as well
as $x$ is a concept variable here), then $A[T/x] = B[T/x]$ is a
theorem of {\em CC\/}, for any term $T$.  This rule supports the
intuition that variables appearing in theorems represent arbitary
concepts.

\end{description}

We introduce a single axiom in this section, to complete the
implementation of the basic properties of equality:

\begin{description}

\item[(Reflex)]  $A = A$

\end{description}

\section{Axioms for the calculus of degree}

Developing a set of axioms for {\em CC\/} turns out to be an
interesting exercise.  The first point to observe is that, since
uninterpreted terms of {\em CC\/} do not have explicit degree (as do
the atomic concepts of the intended interpretation), it is necessary
to axiomatize the properties which degree is expected to have.

We rely on the intended interpretation to develop the basic approach.
In the intended interpretation, there is a canonical object in each
degree which we may as well use to represent that degree: this is the
universal concept $({\cal D}^n,n)$.  In the intended interpretation,
the universal concept of the same degree as a concept $A$ can be
expressed as $(\Delta A/A^c)^c$; it is straightforward to check that
this works.  In uninterpreted {\em CC\/}, we introduce the following

\begin{description}
\item[Definition:]  

$V^A$, called the {\em degree of $A$\/}, is defined as $(\Delta
A/A^c)^c$.  $\emptyset^A$ is defined as $(V^A)^c$.

\end{description}

The operations on degrees which are needed in the intended
interpretation are addition and subtraction of natural numbers; our
intention is that the degree of $A \times B$ will be the sum of the
degrees of $A$ and $B$, while the degree of $A/B$ will be the
difference of the degree of $A$ and the degree of $B$.  In the
intended interpretation, we decided that the degree of $A/B$ would be
0 when the degree of $B$ exceeded the degree of $A$; this corresponds
to a natural decision as to how to define subtraction as a complete
operation on the natural numbers.

These are the axioms which we adopt for addition of degrees:


\begin{description}


\item[D1:]  $V^{V^A} = V^A$

\item[D2:]  $V^{A^c} = V^A$

\item[D3:]  $V^{\Delta A} = V^A \times V^A$

\item[D4:]  $V^{A \times B} = V^A \times V^B$

\item[D5:]  $(A \times B) \times C = A \times (B \times C)$
\end{description}
Axiom {\bf D5} is given in more generality than is needed for the
calculus of degrees alone.
\begin{description}

\item[D6:]  $V^A \times V^B = V^B \times V^A$

\item[D7:]  $V^{A/A} = V^{B/B}$

\end{description}
Axiom {\bf D7} motivates the following

\begin{description}

\item[Definition:]

We define $V^0$ as $V^{A/A}$ and $\emptyset^0$ as $(V^0)^c$.

\end{description}
Degree 0 is the only degree which can be defined in absolute terms.
In the intended interpretation, degree 0 has the two inhabitants $V^0$
and $\emptyset^0$, which are natural representatives of the truth
values.

\begin{description}

\item[D8:] $A \times V^0 = V^0 \times A = A/V^0 = A$

\item[D9:]  $A \times \emptyset^0 = \emptyset^0 \times A = \emptyset^A$

\end{description}

A more difficult proceeding was the selection of the axioms for
subtraction of degrees.  We recall having seen axiomatizations for
natural number subtraction, but we were not able to discover a
reference to such a definition while preparing this paper.  We adopted
the following axioms:

\begin{description}



\item[D10:] $(V^A \times V^B)/V^B = V^A$

\item[D11:]  $(V^A \times V^{B/A})/V^{A/B} = (V^A/V^{A/B})\times V^{B/A} = V^B$

\item[D12:]  $(V^A/V^{A/B}) \times V^{A/B} = V^A$

\item[D13:]  $A/(B\times C) = (A/C)/B$

\item[D14:] $A/B = A/(B/V^{B/A})$

\end{description}

These axioms are motivated by the fact that $m-n = m-\min(m,n)$ under
the definition of subtraction given above.  Note that $m-(m-n) =
\min(m,n)$ and $m+(n-m) = \max(m,n)$ can be defined on the natural
numbers using addition and natural number subtraction.  If the
equations just given relating natural number subtraction to maximum
and minimum are interpreted instead as relating a ``subtraction''
operation on a lattice to least upper bound and greatest lower bound,
the axioms (suitably restricted to degrees) will hold on more general
lattices with an addition operation; this will be exploited below.

Axiom {\bf D13} is stated in more generality than is needed for the
calculus of degrees alone, just as was the case for axiom {\bf D5}.

Axiom {\bf D14} implements a different approach to the problem of
``bad quotients'' than the one adopted in the description above of the
intended interpretation.  Above, we simply declared such quotients
empty; the axiom given here turns out to be more convenient.

\section{Boolean operations and axioms}

Of the operations of Boolean algebra which we might expect to have on
each degree, only complement is given to us as a primitive, and we
state its axiom
\begin{description}

\item[B1:]  $(P^c)^c = P$

\end{description}
We define the other boolean operations by first defining {\em
intersection\/} using the observation that $\Delta A/B = A \cap B$
when $A$ and $B$ have the same degree, along with a trick.

\begin{description}

\item[Definition:]  $A \cap B$, the intersection of concepts $A$ and $B$, is defined as $\Delta(A \times V^{B/A})/(B\times V^{A/B})$.

\end{description}

The trick is observing that $A \times V^{B/A}$ and $B\times V^{A/B}$
have the same degree.  In the intended interpretation, this will be
the maximum of the degrees of $A$ and $B$.  From the axioms, one can
only prove that the degree of each of the expressions will be equal to
the least upper bound of the degrees of $A$ and $B$ in a suitable
order (and it does happen in interpretations in many-sorted logic that
this may be strictly ``larger'' than both $A$ and $B$).

The rest of the boolean axioms follow:

\begin{description}

\item[B2 (a definition):]  $P \cup Q = (P^c\cap Q^c)^c$

\item[B3 (a definition):]  $(V^A)^c = \emptyset^A$
\item[B4:]  $P\cap Q = Q\cap P$

\item[B5:]  $(P\cap Q)\cap R = P\cap (Q\cap R)$

\item[B6:]  $P\cap (Q\cup R) = (P\cap Q) \cup (P\cap R)$

\item[B7:]  $P\cap \emptyset^Q = \emptyset^{P\times(Q/P)}$

\item[B8:]  $P\cap V^Q = P\times V^{Q/P}$

\item[B9:]  $P\cap P^c = \emptyset^P$

\item[B10:]  $P\cap P = P$

\end{description}

\section{Axioms of padding}

The introduction of products with universal concepts (which
corresponds to ``padding'' formulas with dummy variables in the
intended interpretation) would obstruct such simple procedures as dualizing axioms B1-10 if it were not for these (obvious) axioms:

\begin{description}

\item[P1:]  $(V^A \times B \times V^C)^c = V^A \times B^c \times V^C$

\item[P2:] $V^A \times (B \cap C) \times V^D = (V^A \times (B\times
V^{C/B}) \times V^D) \cap (V^A \times (C\times V^{B/C}) \times V^D)$

\item [P3:]  $A \times B$ = $(V^A \times B) \cap (A \times V^B)$

\end{description}

Axioms P1 and P2 tell us that ``padding'' commutes with Boolean
operations (an earlier axiom tells us that $V^0$ is left and right
identity for product, so we can convert P1 and P2 to forms suitable
for considering padding on one side).

Axiom P3 shows us that product reduces to padding and Boolean
operations.

\section{Axioms of quantification}

Quotients by arbitrary concepts can be reduced to quotients by
universal concepts and padding (similar to axiom P3).  The appearance
of $V^{B/A}$ (which will be $V^0$ in all situations we care about) is
a degree arithmetic fix for bad quotients.

\begin{description}

\item[Q1:]  $A / B = (A \cap ((V^{A/B} \times B)/V^{B/A}))/V^B$

\end{description}

\begin{description}

\item[Q2:]  $(A/V^B)^c = (A/V^B)^c \cap (A^c/V^B)$

\item[Q3:]  $(A\cap B)/V^C = (A\cap B)/V^C \cap ((A\times V^{B/A})/V^C)$

\item[Q4:] $(A\cup B)/V^C = ((A\times V^{B/A}) /V^C)\cup ((B\times
V^{A/B})/V^C)$

\item[Q5:]  $(A \times B)/V^{B/(B/C)} = A \times (B/V^{B/(B/C)})$

\item[Q6:]  $A = A \cap ((A/V^{A/(A/B)})\times V^{A/(A/B)})$

\end{description}

To understand the forms of axioms Q5 and Q6, note that $V^{B/(B/C)}$
will be the minimum degree of $V^B$ and $V^C$; we are ensuring that we
do not take a quotient by a universal concept with degree greater than
that of $B$.

\section{Axioms of equality}

There now remains the diagonalization operation.  Like the padding and
quotient operations, it can be simplified using Boolean operations.

\begin{description}

\item[E1:] $\Delta A = \Delta V^A \cap (A \times V^A) = \Delta V^A
\cap (V^A \times A)$

\end{description}

It is an easy consequence of E1 and the distribution axioms for
padding that diagonalization distributes over Boolean operations
relative to $\Delta V^A$.

We introduce a definition to make it easier to talk about general
equality concepts:

\begin{description}

\item[Definition:] The concept Eq$(A,B,C)$ is defined as
$V^A\times(\Delta V^{B\times C}/V^C)$.  

\end{description}

Note that this definition depends only on the degrees of its
arguments.  Eq$(A,B,C)$ contains all those sequences of variables with
an initial segment in $V^A$, a second segment in $V^B$, a third
segment in $V^C$ and a fourth segment equal to the second
segment. Eq$(A,B,C)$ is the general form of a concept capturing all
those sequences with a final segment in $V^B$ equal to an earlier
non-overlapping segment in a given position.
We now state the further axioms of equality:

\begin{description}

\item[E2:] (Eq$(A,B,C) \times V^{D\times B}$) $\cap$ Eq$(A,B,C\times B
\times D)$ $\cap$ Eq$(A\times B \times C,B,D)$ = (Eq$(A,B,C) \times
V^{D\times B}$) $\cap$ Eq$(A,B,C\times B \times D)$ = Eq$(A,B,C\times
B \times D)$ $\cap$ Eq$(A\times B \times C,B,D)$ = (Eq$(A,B,C) \times
V^{D\times B}$) $\cap$ Eq$(A\times B \times C,B,D)$

\item[E3:]  (Eq$(A,B,C) \cap D^c)/V^{B/(B/E)}$ = ((Eq$(A,B,C) \cap D)/V^{B/(B/E)})^c$

\item[E4:]  (Eq$(A,B,C) \cap (D\cap E))/V^{B/(B/F)} =
(($Eq$(A,B,C) \cap (D\times V^{E/D}))/V^{B/(B/F)}) \cap (($Eq$(A,B,C) \cap (E\times V^{D/E}))/V^{B/(B/F)})$

\item[E5:] Eq$(A,B\times C,D) = ($Eq$(A,B,C\times D)\times V^C)\cap$
Eq$(A\times B,C,D \times B)$

\end{description}
Axiom E2 is the transitive law of equality.  In this context, where
``variables'' (positions in argument lists of concepts) cannot be
permuted or duplicated (these effects are simulated using
diagonalization), we should not expect to find analogues of the
reflexive or symmetric properties of equality.

Axioms E3 and E4 assert that existential quantification distributes
over all Boolean operations when the variable quantified over is
asserted to be equal to a variable which remains free.  (It also
applies to blocks of variables).  Axiom E5 allows equality between
blocks of variables to be analyzed into equalities between sub-blocks.

\section{Why these axioms?  Showing that they work!}

Reasoning in the calculus of concepts with these axioms corresponds to
a regimented style of reasoning in which any primitive predicate
appears applied to a consecutive block of variables (shifting blocks
of variables causes no problems because of the degree arithmetic and
padding axioms) and in which all quantifications are applied to the
last free variable available (which means that we need to reason with
labelled formulas; this is also handled by the degree and padding
axioms and related features of the boolean axioms). These features
together eliminate the need for bound (or any) variables.

Modulo these restrictions, the boolean and quantification axioms are
easily seen to be adequate mod adequacy of our extended boolean
algebra.

The factor which needs to be considered is the possibility of
permuting or identifying variables.  This is handled via the expedient
of introducing new variables, declaring them equal to earlier
variables (using concepts Eq$(A,B,C)$), asserting desired properties
of the sequence of variables, then quantifying them out (as we did in
the interpretation of first-order logic in CC).  Axioms E3 and E4
ensure that this process commutes with Boolean operations (as it
should).  Axiom E1 allows facts about equal blocks of variables to be
transferred from one block to another. Axiom E5 ensures that equality
between blocks is related correctly to equality between corresponding
sub-blocks.  Axiom E2 allows us to deduce any additional equalities
between blocks from given equalities between blocks (where equalities
between overlapping blocks are involved, the use of E5 may be needed).
As noted earlier, no analogue of symmetry or reflexivity is needed (or
possible!)

\end{document}









